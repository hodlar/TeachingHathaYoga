\chapter{Preface}
"Cuando estás inspirado por algún gran propósito, algun proyecto extraordinario, todos tus pensamientos rompen sus ataduras: Tu mente traciende limitaciones, tu conciencia expande en todas direcciones y tu te encuentras en un nuevo, grande y maravilloso mundo. Fuerzas dormidas, facultades y talentos cobran vida, y tu descubres que eres una persona mucho mejor de lo que jamás soñaste que serías."

-Patanjali, los Yoga Sutras

\newpage
Acerca de poseer el Yoga

Los materiales presentados en este manual representan una composición e interpretación personal del yoga. El yoga en occidente es tanto una ciencia como un arte de profundo entendimiento de la condición humana. En mi investigación en este tema, he tenido la buena fortuna de conocer grandes maestros, cuya sabidiría ha derramado un hambre de conocimiento en mi propia búsqueda por un entendimiento mayor. Tal acercamiento, con el tiempo, espero pueda beneficiar tanto a estudiantes como a maestros. Cualquier estilo que sea tu práctica, yoga es una práctica de revelación tanto física como mental, y esta revelación es nuestra verdadera naturaleza como una fuente de infinita sabiduría.

Esta fuenta, como el océano, subyace en todas nuestras cualidades individuales. Una de las cualidades del ego individual es la noción de pertenencia, la cuál es manifestada en la forma de: escrituras de tierras que han estado aquí milenios antes de que pusieramos una cerca alrededor de ellas; el deseo de acumular bienes en nuestro nombre; o el patentar ideas. Ultimamente, dejamos todo atrás excepto el entendimiento que cultivamos, y es nuestro privilegio como maestros compartir ese conocimiento.

El yoga no puede ser poseido, porque el yoga como forma de conciencia adquirida, es una parte intrínseca de nuestra naturaleza. Cada vez que tomas una profunda y conciente respiración, has experimentado el yoga.

He decidido escribir este texto como un intento de proporcionar las herramientas necesarias a quienes quieren profundizar sus conocimientos en la disciplina del yoga, idealmente se recomienda al lector tener como mínimo una práctica de yoga de un año con algún instructor certificado, para que de tal forma, la información aquí escrita sea más fácil de comprender. De igual manera se invita a cualquier profesional del yoga a utilizar este material didácticamente (por supuesto, dando crédito al autor) e incluso a darme alguna retroalimentación en el material, ya que no pretendo tener un conocimiento absoluto y será un honor agregar o modificar la información aquí presentada con el apoyo de cualquier experto en el tema.

Namaste,

Carlos Rodríguez

\newpage
