%	First file
\documentclass[a4paper]{article}

\usepackage{mathtools}
\usepackage[]{graphicx}
\usepackage{booktabs}

\title{Enseñando Hatha Yoga}
\date{2016-01-27}
\author{Daniel Clement with Naomi Clement}

\begin{document}
\pagenumbering{gobble}
\maketitle

\newpage
Copyright c 2007 Daniel Clement
Todos los derechos reservados. Sin limitaciones de derechos bajo 
derechos de autor, ninguna parte de esta publicación debe ser 
reproducida, almacenada o introducida en un sistema de 
retribución, o transmitida, en cualquier forma o por cualquier 
medio (electrónico, mecánico, fotocopiado, grabado o de otra 
forma), sin el previo consentimiento escrito del propietario de 
los derechos, excepto para cortas reseñas.
Primer impresión Octubre 2007, segunda impresión 2008, tercer 
impresión 2009, cuarta impresión 2010, quinta impresión 2011.
Contacta al publicador en la web www.opensourceyoga.ca

ISBN: 978-0-9735820-9-3

\newpage
\tableofcontents
\newpage


\begin{table}[h!]
	\centering
	\begin{tabular}{l|c||r}
		1 & 2 & 3\\
		\hline
		a & b & c\\
	\end{tabular}
	\caption{Caption for the table}
	\label{tab:table}
\end{table}


\begin{table}[h!]
	\centering
	\begin{tabular}{ccc}
		\toprule
		Some & actual & content\\
		\midrule
		prettifies & the & content\\
		as & well & as \\
		using & the & booktabs package\\
		\bottomrule
	\end{tabular}
	\caption{The other table}
	\label{tab:table2}
\end{table}

\section{FI LOSOFIA, ESTILO DE VIDA Y ETICA}
texto

\subsection{Por que el yoga pudo haber pasado}
texto

\subsection{Los upanishads}
\subsection{El bagavad Gita}
\subsection{Los Yoga Sutras de Patanjali}

\pagenumbering{arabic}

\begin{equation*} 
	f(x)=x^2
\end{equation*}

$\begin{matrix}
 1 & 0\\
 0 & 1
\end{matrix}$

\begin{align*}
	1 &+ 2 = 3\\
	1 = 3 &- 2\\
	\int^a_b \frac{1}{3}x^3
 \end{align*}


\section{Section}

Hola

\subsection{Subsection}

Probando punto sin comando. Y ahora con comando\@. Ahora con minuscula sin. ahora con.\ fin.
\LaTeX
\today

\subsubsection{Subsubsection}
woah

\paragraph{Paragraph}
Mas texto D=

\subparagraph{Subparagraph}

Mas texto D=!!!!
 
\section{Another section}

\end{document}
