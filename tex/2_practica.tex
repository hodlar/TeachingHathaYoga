\chapter{Técnicas de Entrenamiento y Práctica}
\section{Técnicas de Asana}
\subsection{Alineamientos Fundamentales}
El cuerpo humano tiene una alineación óptima. Cuando el cuerpo se mueve hacia o en un alineamiento óptimo, hay un incremento en la libertad de movimiento de las articulaciones y más energía disponible en el cuerpo, dado que el cuerpo no está peleando consigo mismo y le es pocible moverse libremente. El dolor es tambi\'en reducido o eliminado cuando los huesos y tejidos del cuerpo se encuentran en una relación cooperativa.

La habilidad para realizar una asana puede variar enormemente, dependiendo de facotes como la estructura osea, lesiones previas, edad, nivel de enería y empeño del estudiante. Comenzar una práctica de asana como disciplina de consciencia y no violencia (especialmente a los practicantes) es un comienzo fundamental para una práctica libre de dolor.


\subsection{Biomecánicas Holísticas}
\begin{itemize}
	\item Holistico - ``visto como un todo, integrar.''
	\item Biomecánico - ``el estudio y la aplicación de fuerzas físicas benficas en seres sensibles.''
	\item O: ``No uses el cuerpo en esa postura\ldots usa la postura para entrar en el cuerpo''
\end{itemize}

\textbf{Biomecánizas holísticas en pocas palabras:}
\begin{itemize}
	\item Reconoce que cada parte del cuerpo humano (cuerpo, mente y emociones) es involucrada en cualquier actividad, sea una postura de yoga, una discusión o comer una comida.
	\item Reconoce el potencial para la salud a trav\'ez de la aplicación apropiada de conciencia y forza la reación de relaciones armoniosas entre articulaciones, músculos, huesos, mente y emociones.
	\item Reconoce las limitaciones estructurales del cuerpo y trabaja dentro de esos límites.
	\item Reconoce que cada persona es diferente (músculos, huesos, mente y corazón). La aplicación de cualquier t\'ecnica debe ser adaptativa a las necesidades individuales de esa persona.
\end{itemize}

Una obstrucción primaria para los estudiantes de Hatha Yoga (yoga físico) es intentar interpretar el lenguaje de un maestro conforme dirige el movimiento de ciertas partes de tu cuerpo en formas específicas. El distinto acercamiento a las posturas de yoga puede dar algunas veces conflicto verbal de instrucciones a elos estudiantes. El siguiente trabajo intenta ayudarte como estudiante a entender el trabajo de tu cuerpo en acción.

Desde el trabajo del maestro T. Krishamacharya y otrs de 1930 en adelante, el potencial de curar al cuerpo y obtener un completo rango de mobilidad y vibración física a partir de la práctica de Hatha Yoga ha sido reconocida. El buen alineamiento y la aplicación de fuerza apropiada en el cuerpo físico es la llave. Sincronizando la mente con el cuerpo e incluso emociones es lo que yo llamo ``Biomecánicas holísticas''.

Las biomecánicas holísticas toman la perspectiva del yoga llamada ``Tantra'', y acepta que el cuerpo no es simplemente un vehículo inherte para un espíritu separado, sino un universo inteligente en miniatura. Fuerzas en nuestro universo interactúan en ciertas formas predecibles que son reflejadas en nuestro cuerpo, mente y emociones. La aparente separación de cuerpo, mente y el mundo exterior es una ilusión.

Los movimientos y direcciones de energía o prana en el cuerpo han sido descritos anteriormente en los Upanishads, una antigua compilación de sabiduría yoguica obtenida por varios autores. Muchas escuelas modernas de yoga utilizan antiguos entendimientos de las energías sutiles en el cuerpo, y las describen en diferentes formas.

La perspectiva oriental de las energías sutiles y de cómo el prana se mueve benficamente dentro de el cuerpo junto con algunos modernas (generalmente occidentales) entendimientos de las relaciones de músculos y huesos, conforman la biomecánica holística.

\textbf{Methodology}
Cuando se realizan los estramientos terrenales que llamamos ``asana'', cualquier forma de nuestro cuerpo tiene el potencial de beneficiar la elasticidad y fuerza, o causar lesiones. Si no hacemos nada, sólo nos sentamos en el sofá, por ejemplo, hay poca probabilidad de algún daño inmediato en el cuerpo.

Existe tambin una posibilidad muy alta de eventualmente perder salud debido al atrofiamiento de tejidos del cuerpo, y correspondiendo contraindicaciones de la mente y el cuerpo emocional. La salud se encuentra entre hacer nada y hacer mucho. Cuando el cuerpo se mueve con habilidad y armonía, la intensidad del ejercicio puede ser incrementado con seguridad y los beneficios incrementan tambin. Un ejercicio hecho sin cuidado tiene más alto potencial de lesiones y menos potencial de beneficio, poque mejoramosen las cosas que practicamos.

Las siguientes decuencias de acciones son una síntesis, tomada del trabajo de maestros de yoga, fisicoterapeutas occidentales, observación de animales y práctica personal. Estas tcnicas no pertenecen a nadie, son parte de nuestra cultura e intelectual común. Cómo puedes saber si la t\'ecnica de un fisioterapeuta está funcionando? Se siente bien. Ese reconocimiento es innato, como el deseo de estirarse y descansar, ese conocimiento es intrínseco en todos los seres. (Opuesto de sedentario y sobre-activo)

\textbf{Samasasthiti (pronunciado, sama stee tee hee)}
Samasthiti es la palabra en sánscrito que significa ``esparcir la luz de la consciencia a travs del cuerpo.'' Samasthiti es el estado de consciencia dentro de la postura inicial de levantarse listo frente al tapete, y en cualqueier postura, demostrar un estado que permite la concentración luminosa y compasiva.

El resultado emocional e intelectual de esta ación es una receptividad, físicamente una falta de armonía exterior y sin embargo una fuerte base de las partes de tu cuerpo que tocan la tierra.

\textbf{Integración}
Como "condensaci\'on de consciencia", el practicante dirije los tejidos del cuerpo como uno hacia el centro, localizado a lo largo del eje de la columna vertebral, entre los hombros y las caderas.

Los hombros y caderas se mueven hacia atras en su "hogar" estructural:
\begin{itemize}
	\item los muslos se mueven hacia los izquiotiviales, suavizando el frente de las ingles; y
	\item los hombros descanzan c\'omoda y fuertemente hacia la parte trasera del cuerpo.
\end{itemize}
(Tadasana de nuevo, energ\'ia moviendose hacia adentro)

\textbf{Expansi\'on}
Una vez que la integración se ha llevado a cabo, la energia sublime del cuerpo que se encuentra en distintas formas, se basa en previa integración muscular, fluye del centro del cuerpo a los ejes de la espina, desde abajo de las piernas como las raíces de un árbol, hasta la cima de la espina, brazos y cabeza, como ramas de un árbol.

(energía movi\'endose hacia afuera)

En pocas palabras, la práctica de esas tres acciones: Suavizar, Flexionar y Estirar. Estas acciones pueden ser realizadas en cualquier postura, posición o actividad dentro de una práctica formal de yoga o en actividades diarias incluso, como lavar los platos.

Los beneficios de una fuerza incrementada, un mayor rango de movilidad y una sensación de serenidad aparecen porque el cuerpo, la mente y las emociones son conducidad en la fábrica de nuestro mundo, de nuestro universo. No estamos separados de todo lo que vemos, así que cuando reconozcamos nuetra conección fundamental, y nos comportemos como todas las cosas en el universo lo hacen (pulsando con expansión y contracción) entonces encontraremos esta relación armoniosa. No una relación de inactividad o separación, sino una de participación con la vida.

\textbf{Dolor contra intensidad:}
Para un practicante principiante, la sensación experimentada durante una práctica de asana es con frecuencia desconocida. Con más experiencia, una diferenciación puede ser encontrada entre dolor e intensidad. Dolor repentino, sensaciones desagradables, especialmente alrededor de las articulaciones no debe ser ignorado. El cuerpo envía una señal de desalineación o desconexión que podría ser dañina.

Sensaciones de estiramientos intensos a un músculo puede ser interpretado como dolor, pero la sensación es muy distinta. Con frecuencia un practicante se encuentra en control de la cantidad de sensaciones experimentadas, como en una flexión sentado hacia el frente. Estas sensaciones de intensidad son intrínsecamente parte de la práctica. Generalmente respirar en las posturas intensas recoge la mente del cuerpo y la resistencia decrementa.

\subsection{Terapia Estructural}

La terapia estructural es la aplicación consciente de las holisticas biomecanicas de un practicante a su cliente, con una completa participación del cliente. Tocar con vigor establece resonancia entre el practicante y el cliente, antes y durante los ajustes realizados. Con práctica, el practicante aprende a diferenciar entre tipos de resistencia: compresión, tensión, músculo y tejido conectivo. Junto con la resistencia física, pueden ser encontrados patrones de resistencia en los cuerpos emocionales y mentales.

La realización de ajustes será de poco valor a largo plazo a menos que se ilustre al cliente ómo realizar una buena alineación, con apropiada integración y expansión. La idea es que el cliente gane consciencia kinestésica de cómo crear alineación curativa para ellos mismos.

\textbf{Qué tanta presión?}
Cuando se aplica un ajuste, primero debes conectar con la parte del cuerpo que vas a ajustar con seguridad, incluso presión. Moverás piel, músculo y hueso como una unidad. Poca presión será inefectiva. El ajuste debe comenzar desde la más baja intensidad e ir incrementando hasta el máximo en un período de 3 o 4 segundos, dl ajuste mismo puede durar desde 5 segundos hasta 1 minuto.

Mantén tu atención en la cara del cliente buscando signos de inconformidad, dolor o alivio. Cuando trabajes con piernas o cadera, tus manos y brazos son usualmente menos fuertes que con lo que estás trabajando. Cada cuerpo es distinto y la sensitividad o sensación es distinta de persona a persona. Mantente comunicado con tu cliente, revisando si está experimentando dolor o alivio.

\textbf{Areas clave de imbalance}

Desalineación ocurre donde los huesos se encuentran. La desalineación habitual ocurre en el estilo de vida: sentado por largos períodos, movimientos repetitivos que crean imbalance muscular, y posiblemente nuestra forma de estar erguidos y caminar como tales.

Es posible que desde una perspectiva evolutiva, aún nos encontremos en evolución física y que no estemos por completo adaptados a una posición erguida. La parte superior de nuestro fémur encaja mejor en el hueco de la cadera en una posición inclinada en lugar de en una orientación vertical, la parte exterior de nuestras piernas se encuentran mas rígidas cuando nos encontramos de pie igualmente. Nuestros muslos interiores son difíciles de involucrar muscularmente para rebalancear la rotación común del femur debido a la rigidez de las piernas exteriores
éáíñúó
\subsection{Técnicas de Asana: Categorías de Posturas}
\subsection{Forma y Acción}
\subsection{Navegando el Tapete}
\subsection{Sacro}
\subsection{La Práctica de Asana}
\subsection{Yin y Yang}
\subsection{Polaridades de la Energía Física}
\subsection{Fuerzas Opuestas}
\section{Téecnicas de Purificación}
\section{Meditación}
\section{Pranayama}
\subsection{Nadi Shodhana}
