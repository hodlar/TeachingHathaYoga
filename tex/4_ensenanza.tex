\chapter{Metodología de Enseñanza}
\section{Enseñando a dirigir}
Antes de hablar acerca de la filosofía de yoga dentro de una clase o incluso enseñar una postura, debes ser capaz de dirigir el movimiento de un estudiante con claridad y unas cuantas palabras.

\textbf{Ejercicio:}

Elige diariamente una actividad como abrir una puerta, quitarte un zapato, o rascar tu pierna. Escribe un guión para esa acción de tal manera que pueda ser llevada a cabo sin interpretación. Un ejemplo de una instrucción con escasez de claridad sería:

``Camina y toma la perilla de la puerta y abre la puerta.''

Esta instrucción funcionaría, pero sólo porque la persona a qui\'en instruyes ha abierto muchas puertas antes y sabe como, una mejor instrucción sería:

``Colocado a un brazo de distancia de la puerta, coloca tu pie izquierdo al frente y toma la perilla de la puerta con tu mano derecha. Rota la perilla en el sentido del reloj hasta que se detenga, luego colocando un poco más de peso en tu pi\'e derecho, suavemente jala la perilla hacia ti, abriendo la puerta.''

Enseña a un amigo a usar distintos objetos y movimientos diariamente, asegurándote de que tu amigo no interprete las instrucciones a su manera, sino que haga exactamente lo que instruíste. La experiencia puede ilustrar qu\'e tan difícil puede ser mantener tu voz con claridad y fuerza.

\section{Agregando Contenido}
Comienza con sencillez. Simple es claro. Claro es bueno, Practica tus enseñanzas simplemente instruyendo la respirazión de tu propia práctica del Saludo al Sol, una respiración por movimiento. Desde ahí, utiliza instrucciones fundamentales que est\'en tan bien establecidas que te permitan tener libertad creativa.

\subsection{Nivel 1 - Respiración}
Conforme instruyes verbalmente la respiración en tu propia práctica, escucho el tono de tu voz. Observa la sincronización y el paso de la instrucción simple de inhalar, exhalar. Crea un flujo tranquilo en tu cuerpo y tus palabras. Di la palabra ``inhala'' justo un poco antes de iniciar el movimiento en Urdhva Hastasana, y ``exhala'' justo antes del segundo movimiento hacia Uttasana. Observa tus propias tendencias a acortar la respiración.

Cuando te sientas cómodo dirigiendo tu propia respiración, visualiza  la secuencia de movimientos de Surya Namaskara en lugar de realizarlos, y dirige la respiración audiblemente. Cuando te encuentres cómodo con ese nivel de instrucción, camina alrededor del cuarto y dirige la respiración, manteniendo un paso tranquilo de movimiento sincronizando con tu propia inhalación y exhalación. Incorpora las enseñanzas de la respiración tan completamente de tal forma que si tu congelaras un instante en cualquier parte del saludo al sol, sabrías que parte de la respiración (inhalación o exhalación) se conecta.

\subsection{Nivel 2 - Movimiento del Cuerpo Exterior}
Lo siguiente es enseñar el movimiento del cuerpo exterior. La duración dentro de una postura es dependiente del estilo de clase que est\'es enseñando (como alentadora, restaurativa, meditativa, etc). Como punto inicial, cuenta tu número de respiraciones en un minuto, utilizando tus respiraciones para marcar la duración. Enseña posturas por 45 segundos por lado para una clase de paso medio, un minuto para un paso más lento.

Necesitarás escribir un guión para instruír la entrada a una postura. Comenzando tu entrada a la mayoría de las posturas de pie desde un desplante te proporciona un punto de referencia ``base''. Aquí está un ejemplo de instrucción meramente física a Parsvakonasana (postura de lado) entrando desde un desplante:

``Desde un desplante (realizado con el pi\'e derecho al frente, pi\'e izquierdo atrás) rotando tu pi\'e trasero 90 grados y presionando por completo las cuatro esquinas del pi\'e a la tierra. Coloca tu antebrazo derecho en tu muslo derecho y tu mano izquierda en la cadera. Rota tu torso hacia la izquierda.''

Esta es una instrucción muy básica, libre de palabras innecesarias. Esta instrucción tarda alrededor de quince segundos, permitiendo a los estudiantes realizar cada parte de la instrucción. Una vez que ellos han tomado la forma básica, la duración es cinco respiraciones o aproximadamente un minuto. La postura es repetira de nuevo hacia el otro lado con la misma duración. Escribe un guión básico para el cuerpo exterior para todas las posturas de pie que enseñarás y practica decirlas conforme realizas el asana, utilizando los movimientos de tu propio cuerpo como guía para la sincronización. Habla primero, despu\'es mu\'evete.

Cuando te sientes cómodo a este nivel, instruye tu propio cuerpo, intenta levantarte como si estuvieras dirigiendo una clase al frente del cuarto y dirigiendo oralmente una postura de pi\'e a la vez. Luego sincronízate, eliminando cualquier cosa poco clara o innecesaria.

\subsection{Nivel 3 - Alineamiento Fisico/Movimiento Energético}
Construyendo sobre la respiración y la forma básica de la postura, alineando el cuerpo óptimamente es lo siguiente. La alineación básica del cuerpo exterior (longitud de la postura, posición del cuerpo) debe ser tomada en cuenta en tu instrucción de movimiento del cuerpo exterior. Ahora puedes comenzar a describir el movimiento del prana, conectado a la inhalación y exhalación para alinear al estudiante con el pulso de la naturaleza. Conectando la inhalación a energía condensada, la exhalación a energía expansiva. Aquí está un ejemplo, basado en las previas instrucciones a Parsvakonasana:

``Desde un desplante (realizado con el pi\'e derecho al frente, pi\'e izquierdo atrás) rotando tu pi\'e trasero 90 grados y presionando por completo las cuatro esquinas del pi\'e a la tierra. Inhala conforme recoges tu energía de la tierra hacia tu centro. Coloca tu antebrazo derecho en tu muslo derecho y tu mano izquierda en la cadera. Rota tu torso hacia la izquierda. En tu siguiente exhalación, envía la energía hacia atrás a trav\'es de las piernas hacia la tierra''

Esta instrucción mejorada ahora toma alrededor de 25 segundos en ser expresada. A este punto, estás comenzando a introducir la intención filosófica. Al simplemente describir el flujo de la energía la atención del estudiante es recogida a esta pulsación universal de opuestos. Dependiendo del estudiante, esta incorporación física puede ser un momento ``Ah,Ha!'', o puede no resonar del todo. Continúa enseñando, Practica esto con todas las posturas de pie de nuevo, conforme las realizas, y luego mantente quieto.

\subsection{Nivel 4 - Incorporando Intención}
Aquí, tu enseñanza de clase debe referirse a la intención inicial de la clase introducida brevemente al comienzo de la clase. Puedes usar un poema, compartir una experiencia personal, recitar un yoga \textit{sutra} o utilizar cualquier otro material inspirador para introducir una intención. Cualquiera que sea, entrelaza la intención a travs de las instrucciones de postura en tu clase. Debido a lo que experimentamos físicamente en la práctica de asana se encuentra relacionado a una pulsasión mayor de la vida, la esencia de la enseñanza se vuelve evidente, encaminando a una profundización de conciencia en el estudiante. Una enseñanza creativa y hábil eleva la práctica del asana de una rutina física a su potencial como eperiencia que revela la integridad. Aquí está un ejemplo, de nuevo construyendo en previas instrucciones. La intención para esta clase, reflejado en las instrucciones de posturas, es el ``no apego.''

``Desde un desplante (realizado con el pi\'e derecho al frente, pi\'e izquierdo atrás) rotando tu pi\'e trasero 90 grados y presionando por completo las cuatro esquinas del pi\'e a la tierra, recon\'ectate con ella. Inhala conforme recoges tu energía de la tierra hacia tu centro, haciendo conciencia que lo que es tomado debe ser regresado. Coloca tu antebrazo derecho en tu muslo derecho y tu mano izquierda en la cadera. Rota tu torso hacia la izquierda, y con gratitud expande completamente en la inhalación. En tu siguiente exhalación, envía la energía hacia atrás a trav\'es de las piernas hacia la tierra, y permite este pulso de energía moverse a trav\'es de ti como un río, este flujo se mantiene limpio, sin inactividad.''

Este nivel de instrucción puede ser muy inspirador si viene naturalmente de un lugar de verdadera experiencia en un maestro. Debes decidir con qu\'e nivel de instrucción te encuentras cómodo. Asegurate de que tienes los fundamentos de instrucción absolutamente sólidos antes de avanzar. No tiene sentido intentar transmitir aspectos sutiles de nuestra verdadera naturaleza cuando media clase se encuentra en otra postura.

Tu intención funcionará más efectivamente si es una intención que puede ser incororada. El ejemplo anterior funciona en una práctica de asana porque liberanr la respiración es claramente una forma física de no apego. Tu puedes tener una gran intención de clase que es difícil de incorporar. Una intención como ``estudiar una escritura para incrementar la sabiduría'' es una gran cosa para dedicar el esfuerzo personal, pero es difícil desmenuzar este tipo de intención en el cuerpo durante la práctica de asanas.

\subsubsection{Dirigiendo posturas específicas}
Ser un buen maestro de yoga es como ser un buen mesero. Diriges a los estudiantes para tomar su asiento, les describes qu\'e hay en el menú, modificas probablemente alguno que otro platillo para servirlo mejor, y ocacionalmente vas a revisar cómo van las cosas. Cada tiempo de la comida necesita ser llevado y presentado, y entonces el mesero permite a los comenzales disfrutar.

Al utilizar un lenguaje efectivo y claro, los estudiantes a quienes enseñas deberán ser instruídos para entrar y salir de la forma básica de cada postura que enseñas. Los estudiantes avanzados pueden estar familiarizados con los nombres de muchas posturas, pero los principiantes requerirán instrucción de cómo aproximarse a cada postura. Las instrucciones básicas son esenciales antes de detallados o alineaciones que se enseñen. Para hacer esto, necesitarás escribir un guión para cada postura que pretendas enseñar.

Este trabajo se vuelve cada vez más sencillo dado que muchas posturas comparten similaridades. La longitud de apertura para muchas posturas de pie es la misma. Muchas posturas de piso comparten tambin similaridades en su forma general. La maestría de enseñar una vez que una forma básica de una postura es realizada por tus estudiantes consiste en ilustrar las diferencias entre las posturas.

\subsubsection{Algunos ejemplos de posibles guiones:}

\underline{Surya Namaskara - Saludo al Sol}

``Inhala y eleva tus brazos, exhala y flexionate al frente, tocando el suelo.
Inhala y mira al frente.
Exhala, camina hacia atrás y baja a plancha baja, llamada Chataronga Dandasana.
Inhala, desliza tu cuerpo hacia el frente y arriba en Urdhva Mukha Svanasana (perro boca arriba).
Exhala y muevete hacia atrás en una V invertida, o Adho Mukha Svanasana.
Mant\'en esta postura por cinco respiraciones, siente la conexión con la tierra mediante de tus dedos.
Flexiona las rodillas ligeramente, estira hacia atrás desde las manos para alargar la columna.
En tu sexta inhalación, caminas al frente y mira al frente.
Exhala y flexiona al frente de nuevo.
Inhala colócate de pie, eleando tus brazos al cielo.
Exhala manos al pecho y sigues de pie.''

Debido a la velocidad de Surya Namaskara, moverse a una postura por respiración, no es posible instruir mucho más que la entrada básica a cada postura. Surya Namaskara es un lugar excelente para comenzar a instruír posturas con presición debido a esto.

\subsubsection{Ejercicio:}
Escribe un guión para Surya Namaskara en tus propias palabras. Tu propio uso de gramática, el lenguaje coloquial y la elección de palabras es como una firma. Tu forma natural de interpretar la práctica es única de ti. Has la enseñanza propia, no una copia de nadie más. Al mismo tiempo, mantn el número de palabras al mínimo mientras sigues dando instrucciones claras.

\subsubsection{Continuando los ejemplos de guines\ldots}
\underline{Parsvakonasana}\\
``Desde perro boca abajo, da un paso al frente entre las manos. Colocando tu antebrazo derecho en tu rodilla, rota tu torso hacia la izquierda conforme rotas tu pie trasero 90 grados hacia el frente de tu tapete. MAnteniendo tu espalda fuerte, dirige tu rodilla derecha justo sobre tu tobillo. Eleva el brazo izquierdo 45 grados hacia el techo, con la palma mirando hacia abajo. Respira profundo, relajando la mirada. Mantente ahí por cinco respiraciones. En tu siguiente inhalación, mant\'en ambas palmas en el tapete, exhala y camina hacia atrás en perro boca abajo.''
AQUI VA UNA IMAGEN

\underline{Janu Sirsasana}\\
Desde Dandasana,lleva tu pierna izquierda hacia el frente y coloca la planta del pie contra tu muslo. Mant\'en los dedos en el suelo y mant\'en firmes los músculos de tus piernas. Desde la base, extiende la columna hacia arriba y la pelvis hacia el frente. Toma tu pierna, tobillo o pie con la mano derecha. Usando la conexión con la tierra a trav\'es de los dedos izquierdos, inhala y cuadra tus hombros, exhala y flexiona hacia el frente en la postura. Cada inhalación, alarga la columna. Cada exhalación, entr\'egate más profundamente en Janu Sirsasana. Toma cinco largas respiraciones aquí. Mant\'en tu pierna extendida firme hacia el suelo y en tu siguiente inhalación, el\'evate de nuevo de la flexión y coloca las piernas en Dandasana de nuevo.``
AQUI VA UNA IMAGEN

\underline{Rajakapotasana}\\
Desde perro boca abajo, desliza tu pie derecho al frente, cruzándolo por el frente del tapete de tal forma que tu rodilla dercha se encuentre más abierta que tu cadera, y la espinilla exterior y el tobillo traseros descancen en el tapete. Mantn tu pierna izquierda activa mientras apoyas las uñas de los pies en el tapete, estirándose hacia atrás. Los dedos o palmas debajo de los hombros, inhala y alarga la columna, manteniendo tu base firme en ambas piernas. Exhala y arquea hacia atrás, llevando la punta de la garganta atrás. Continúa activando las piernas y llena las costillas con aire. Despus de cinco respiraciones, coloca las palmas hacia abajo y empuja firmememnte en perro boca abajo.``
AQUI VA UNA IMAGEN

\underline{Savasana}\\
``Ahora el esfuerzo de la práctica de asana ha terminado. Acu\'estate en tu tapete, permitiendo que tus palmas se abran y las piernas y pi\'es se relajen. Toma otra respiración profunda por la naríz y exhala por la boca. Permitiendo que los ojos se relajen hacia atrás en sus cavidades con los párpados descanzando sobre ellos. Siente la conexión entre tu cuerpo y el suelo y entra más profundamente en la experiencia de la relajación.
AQUI VA UNA IMAGEN

Cada palabra que utilizes debe ser específica y no fácilmente mal interpretada. Recuerda, tus estudiantes se estarán esforzando físicamente, experimentando alguna incomodidad e intentando escuchar tu voz por encima de su propio diálogo. Utilize un diálogo activo y directo para dirigir las formas básicas de las posturas.

Esos ejemplos son simples y directos, y constituyen las ``tuercas y tornillos'' de instruír posturas. Tu propia energía, palabras e intención para la clase que estás enseñando florecerán de una instrucción clara de fundamentos básicos, y te permitirá verdaderamente llevar tu pripia voz a la clase.

Apegarte a las instrucciones básicas te permitirá tambi\'en relajarte completamente conforme enseñas. Esta fluidez y confianza tangible motivará a tus estudiantes a relajarse y tambi\'en aceptar las enseñanzas más completamente. Cuando ya dominas instruír con sencillez, entonces puedes tambi\'en comenzar a enseñar lo que observas, agregando los ajustes apropiados físicos y verbales cuando sean necesarios.

\subsubsection{Ejercicio:}
Escribe un guión para cinco posturas de pie y cinco posturas de piso. Diríge la entrada en posturas de pie desde perro boca abajo. Dirige las posturas de piso desde Dandasana. Se tan corto como sea posible por ahora, recordando que el estudiante necesita encontrar la postura y luego pasar cinco largas respiraciones experimentándola.

Despu\'es de escribir tu guión, leelo en voz alta en una voz natural y observa si suena como algo que dirías. Evita descripciones anatómicas no familiares o jerga que la población general no entendería.

\subsection{Enseñando lo que Observas}
\subsection{Temporizando una Clase}
\subsection{Saludando/Centrando}
\subsection{Secuencia}
\subsection{Trabajo en Equipo}
\section{Tematizando}
\subsection{Tematizando a una Postura Especifíca}
\section{Organización del Salón de Clase}
\subsection{Líneas Visuales}
\subsection{Diseño de Clase}
\subsection{Ofreciendo Props}
\section{Demostración}
\subsection{Demostración Silenciosa}
\section{Cuestiones de Salud}
\subsection{Estudiantes Lesionados}
\subsection{Usando Props}
\subsection{Cuidados Específicos de Salud}
\subsection{Yoga para Artritis}
\subsection{Fibromalgia}
\section{Lenguaje}
\subsection{Tono de Voz}
\subsection{Que Tanto Decir?}
\subsection{Se Conciso}
\subsection{Volumen/Contenido}
\section{Modificación de Postura}
\section{Observación: Estudiante Individual}
\subsection{Fundación}
\subsection{Estado de Animo General}
\section{El Rol del Maestro}
\section{Fundamentos de Secuenciado}
\subsection{Secuenciado Variable y Establecido}
\subsection{Principios de Secuenciado}
\subsection{Entretenimiento - Centrando la Clase}
\subsection{Secuenciando la Clase a Nivel Mixto}
\section{Creando Intención}
\subsection{La Intención de "Liberar Tensión"}
\section{La Práctica y el Servicio de Enseñar Yoga}


