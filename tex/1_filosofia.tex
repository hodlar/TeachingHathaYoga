\chapter{FILOSOFÍA, ESTILO DE VIDA Y ÉTICA}
texto
\section{Por qué el yoga pudo haber pasado}
\subsection{Los upanishads}
\subsection{El bagavad Gita}
\subsection{Los Yoga Sutras de Patanjali}
\subsection{Vedanta - La filosofía No-Dual del Sankara}
\subsection{Kashmir Saivism y los Tattvas}
\subsection{Yoga Moderno}
\subsection{Cronología del Yoga}
\section{Dieta y Estilo de Vida}
\subsection{Sueño}
\subsection{Práctica}
\subsection{Relaciones}
\section{El principio de atracción}
\subsection{Yoga y el principio de atracción}
\subsection{Reciviendo}
\subsection{Práctica}
\section{Ética}
\subsection{Yamas y Niyamas}
\subsection{Mas de Ética de un maestro}
\section{Luz y Oscuridad}
\section{Mantras}
\section{Visión general de los Estilos de Yoga}
Una variedad de caminos del yoga continúan entrelazándose e informándose entre sí. Algunos ejemplos son:
\begin{itemize}
	\item \textbf{Anusara Yoga:} Desarrollado por John Friend en 1997; unirica la filosofía enfocada al corazón Tántrica con principios de alineación bio-mecánicos.
	\item \textbf{Ashtanga Vinyasa Yoga:} Desarrollado por T.Krishnamacharya y su estudiante Pattabhi Jois; acercamiento sistemático y secuencial a la práctica de los asanas donde las posturas son separadas en series. Vinyasa, una conexión energetica de un asana a otro, es usado para crear y mantener calor y un movimiento de estado meditativo.
	\item \textbf{Ashtanga Yoga:} Llamado así por Baba Hari Dass, despus del camino de ocho pasos de Patanjali, no confundirlo con Ashtanga Vinyasa Yoga.
	\item \textbf{Bikram Yoga:} Una secuencia de veintiseis posturas en un cuarto calentado a 100 grados Fahrenheit.
	\item \textbf{Iyengar Yoga:} B.K.S. Iyengar, otro de los estudiantes de Krishnamacharya, refinó el aprendizaje de su gurú despus de mudarse a Pune. Avandonó el estilo Vinyasa y se enfocó en las enseñanzas de la salud, alineación estructural y beneficios terapeuticos de las posturas.
	\item \textbf{Kundalini Yoga:} Despertar energía, Kundalini yoga llegó al oeste en 1969, cuando Sikh Yogi Bhajan desafió la tradición y comenzó a enzeñarlo públicamento. Esta práctica es designada para despertar la energía Kundalini, la cuál es almacenada en la base de la espina y en ocasiones representada como una serpiente enroscada. Kundalini mezcla cantos, prácticas de respiración y ejercicios de yoga. El enfasis no es en los asanas, sino en los cantos y respiraciones.
	\item \textbf{Mysore Style:} Nombrado así en honor a la ciudad en india donde Pattabhi Jois enseña el metodo Ashtanga Vinyasa; una práctica autoencaminada con supervición y ajustes físicos de un instructor.
	\item \textbf{Vijnana Yoga:} Una práctica de Hatha Yoga desarrollada por Donna Holleman y Orit Sen-Gupta, basada en siete \"principios vitales\" diseñados para usar el cuerpo para explorar los más profundos niveles de nuestro ser.
	\item \textbf{Viniyoga:} Esta forma gentil de flow yoga pone gran \'enfasis en la respicación y cordina respiración con movimiento. El movimiento fluido de Viniyoga o Vinyasa es similar a la dinámica de la serie de poses de Ashtanga, pero ejecutado con una gran reducción de paz y nivel de estr\'es. Las posturas y secuencias son elejidas para encajar en las habilidades del estudiante. Se enseña al estudiante cómo aplicar las herramientas de yoga: asana, cántos, pranayama (control de la respiración), y meditación, en una práctica individual. Desarrollado por T.K.V. Desikachar, el hijo de Krisnamacharya (maestro de algunos grandes instructores de yoga incluyendo Iyengar y Pattabhi Jois), Viniyoga pone menos estres en uniones y rodillas manteniendo las posturas con una ligera flexión en las rodillas. Viniyoga es considerado excelente para principiantes, y es incrementalmente usado en ambientes rerap\'euticos.
	\item \textbf{Yin Yoga:} Un t\'ermino creado por Paul Grilley para describir una forma de práctica con un \'enfasis en posturas mantenidas por mucho tiempo, usualmente sentados, boca abajo o boca arriba. Yin Yoga se enfoca en fortalecer y alargar el tejido conectivo, que en turno, a partir de líneas meridianas, tiene un efecto óptimo en el funcionamiento de los órganos.
\end{itemize}

\subsection{Viendo la Imagen Completa}
ga}


\section{Sencillez Voluntaria}
\subsection{Principios de Sencillez}
\subsection{Acercamiento a Sencillez}
\subsection{Un Acercamiento al Lado Financiero}

lakoae dwomcsdkjfcnerdIJERKLMDSCIOEJDM
