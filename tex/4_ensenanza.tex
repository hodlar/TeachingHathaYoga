\chapter{Metodología de Enseñanza}
\section{Enseñando a dirigir}
\section{Agregando Contenido}
\subsection{Nivel 1 - Respiración}
\subsection{Nivel 2 - Movimiento del Cuerpo Exterior}
\subsection{Nivel 3 - Alineamiento Fisico/Movimiento Energético}
\subsection{Nivel 4 - Incorporando Intención}
\subsection{Enseñando lo que Observas}
\subsection{Temporizando una Clase}
\subsection{Saludando/Centrando}
\subsection{Secuencia}
\subsection{Trabajo en Equipo}
\section{Tematizando}
\subsection{Tematizando a una Postura Especifíca}
\section{Organización del Salón de Clase}
\subsection{Líneas Visuales}
\subsection{Diseño de Clase}
\subsection{Ofreciendo Props}
\section{Demostración}
\subsection{Demostración Silenciosa}
\section{Cuestiones de Salud}
\subsection{Estudiantes Lesionados}
\subsection{Usando Props}
\subsection{Cuidados Específicos de Salud}
\subsection{Yoga para Artritis}
\subsection{Fibromalgia}
\section{Lenguaje}
\subsection{Tono de Voz}
\subsection{Que Tanto Decir?}
\subsection{Se Conciso}
\subsection{Volumen/Contenido}
\section{Modificación de Postura}
\section{Observación: Estudiante Individual}
\subsection{Fundación}
\subsection{Estado de Animo General}
\section{El Rol del Maestro}
\section{Fundamentos de Secuenciado}
\subsection{Secuenciado Variable y Establecido}
\subsection{Principios de Secuenciado}
\subsection{Entretenimiento - Centrando la Clase}
\subsection{Secuenciando la Clase a Nivel Mixto}
\section{Creando Intención}
\subsection{La Intención de "Liberar Tensión"}
\section{La Práctica y el Servicio de Enseñar Yoga}


