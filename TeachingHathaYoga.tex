\documentclass[a4paper]{book}



\usepackage[T1]{fontenc}
\usepackage[utf8]{inputenc}
\usepackage{booktabs}

\title{Enseñando Hatha Yoga}
\date{2016-01-27}
\author{Daniel Clement with Naomi Clement}

\begin{document}
\pagenumbering{gobble}
\maketitle

\newpage
Copyright c 2007 Daniel Clement
Todos los derechos reservados. Sin limitaciones de derechos bajo derechos de autor, ninguna parte de esta publicación debe ser reproducida, almacenada o introducida en un sistema de
retribución, o transmitida, en cualquier forma o por cualquier medio (electrónico, mecánico, fotocopiado, grabado o de otra forma), sin el previo consentimiento escrito del propietario de los derechos, excepto para cortas reseñas.
Primer impresión Octubre 2007, segunda impresión 2008, tercer impresión 2009, cuarta impresión 2010, quinta impresión 2011.
Contacta al publicador en la web www.opensourceyoga.ca

ISBN: 978-0-9735820-9-3

\newpage
\tableofcontents

\newpage
"Cuando estás inspirado por algún gran propósito, algun proyecto extraordinario, todos tus pensamientos rompen sus ataduras: Tu mente traciende limitaciones, tu conciencia expande en todas direcciones y tu te encuentras en un nuevo, grande y maravilloso mundo. Fuercas dormidas, facultades y talentos cobran vida, y tu descubres que eres una persona mucho mejor de lo que jamás soñaste que serías."

-Patanjali, los Yoga Sutras

\newpage
Acerca de poseer el Yoga

Los materiales presentados en este manual representan una composición e interpretación personal del yoga. Yoga es tanto una ciencia como un arte de profundo entendimiento de la condición humana. En mi investigación en este tema, he tenido la buena fortuna de conocer grandes maestros, cuya sabidiría ha derramado luz en mi propia búsqueda por un entendimiento mayor. Tal acercamiento, con el tiempo, beneficiará tanto a estudiantes como a maestros. Cualquier estilo que sea tu práctica, yoga es una práctica de revelación. Lo que es revelado es nuestra verdadera naturaleza como un aspecto de Fuente.

Esta fuenta, como el océano, subyace en todas nuestras cualidades individuales. Una de las cualidades del ego individual es la noción de pertenencia. Pertenencia es manifestada en la forma de: escrituras de tierras que han estado aquí milenios antes de que pusieramos una cerca alrededor de ellos; el deseo de acumular bienes en nuestro nombre; o el patentar ideas. Ultimamente, dejamos todo atrás excepto el entendimiento que cultivamos, y es nuestro privilegio como maestros compartir ese conocimiento.

El yoga puede no ser poseido, porque el yoga como forma de conciencia adquirida, es una parte intrínseca de nuestra naturaleza. Cada vez que tomas una profunda y conciente respiración, has experimentado el yoga.

Namaste,

Dan Clement

\newpage

\chapter{FILOSOFÍA, ESTILO DE VIDA Y ÉTICA}
texto
\section{Por qué el yoga pudo haber pasado}
\textbf{\textit{Imaginate a ti mismo hace 10,000 años\ldots}}
Te despiertas con el olor a tierra conforme el sol calienta y comienza a evaporar la humedad. Tu vivienda simple provee algo de refugio de los elementos. Tu tribu se levanta al amanecer; los niños juegan alrededor. Conoces a todos en tu tribu, y ellos te conocen. Más tarde llega la el resultado de la caza para ser cocinada y compartida por todos. Cuando el sol se pone, se hace un fuego y observas la madera convertirse en caloe, luz y ceniza. El humo se eleva en el cielo donde las estrellas son tan claras debido a la ausencia de luz ambiental que puedes reconocer las constelaciones como viejos amigos. Cuando es tiempo de dormir, el sueño llega fácil al cuerpo en armonía con el resto de la naturaleza. Cuando la tribu se mueve, volteas atrás y todo lo que queda detrás de ti es el cesped doblado de donde los refugios se encontraban, y un pozo de fuego. En unas cuantas semanas incluso esto se volverá invisible.

La idea de ``tener relaciones'' con otros en tu tribu no es un concepto que nadie comprenda. Conoces a estas personas de toda la vida. Algunos te agradan más que otros, pero no se comienzan y terminan relaciones con ellos. Ellos son literalmente tus relaciones, tanto como los animales y plantas. Los niños alrededor son responsabilidad de todos, y obtienen diferentes habilidades de otros en tu tribo conforme se mueven libremente alrededor.

Este es un regalo económico. La moneda no existe, y la respuesta natural a la abundancia de la vida en la tierra es la gratitud de uno. Nadie tiene la idea de que la vida humana deba ser prolongada, o que la juventud es mejor que la madurez. Los espíritus humanos y animales habitan la tierra.

El escenario anterior puede sonar utópico. Eso es mayoritariamente porque a trav\'es de proyecciones y algunos estudios poco cienfíficos de el siglo 19 en europa, se nos ha hecho creer que la vida de nuestros ancestros es ``brutal y corta''. Este no era el caso. Ni lo era acerca de que nuestros ancestros eran menos saludables que nosotros. Comiendo una dieta ``orgánica'', azúcar en cualquier forma dado que era difícil encontrar, y caminando kilómetros diariamente, nuestros predecesores eran generalmente atl\'eticos, saludables y competentes en diversas áreas.

La conexión profunda de nuestros ancestros nómadas con el mundo natural hace viible la continuidad de todas las cosas en su primer naturaleza. Justo cuando comenzamos a forzar la naturaleza para proveernos con más de lo que podemos consumir al instante, a trav\'es de la agricultura, removimos la idea del despíritu de la naturaleza, por un regalo nunca pedido, y una vez pedido, no es más un regalo.

Dejando atrás el estado de reconocimiento de la naturaleza sagrada de las cosas, la jerarquía espiritual evoluciona. Una vez removido de la tierra, el espíritu se movió a las montalas (dominio de los dioses antiguos) y entonces a los cielos. Abajo es mejor que aquí, abajo es peor. Subir es bueno, mientras que decender es malo. Este ``verticalismo'' tambi\'en disminuye nuestra conexión horizontal del uno con el otro. Cuando observa la historia de las religiones organizadas, la representación de dios es ``en lo más alto'' (en una plataforma, un trono o si está caminando, es muy alto).

Lo pantanoso, la tierra f\'ertil de la vida diaria no era sagrada, y junto con el modelo vertical del espíritu, llegó otro modelo, pureza: blanca, brillante, inmaculada. Nuestro instinto, la naturaleza crnal humana se volvió menos espiritual y finalmente pecado, y a su vez, la idea abstractca del cielo deseada. El propósito del hombre en la vida estaba ahora no solo restrinjida de la naturaleza a trav\'ez de cultivos de la tierra y domesticación de animales, sino tambi\'en restringido de su propia naturaleza. Esto es, convertirse en un cultivo.

La creación del mito de Adan y Eva arrojados del jardín del Eden había sido invertida. El jardín natural del mundo, cultivado por el hombre. Adan y eva fueron arrojados \textit{en} un jardín, donde tenían aue labrar la tierra con el sudor de su frente.

Los orígenes del yoga son algo misteriosos, con la tradición siendo largamente oral en naturaleza. Esculturas encontradas en las civilizaciones de valles de ríos Indús de Harrapa y Mohenjo-daro representando una figura sentada en lo que parece ser una postura de yoga data del 2500AC.

Siendo los orgígenes de la agricultura aproximadamente 5,000 años antes de esto. La práctica del yoga comienza \textit{despus\'es} de la revolución agrícola. La mayoría, mas no todos los cazadores se convirtieron en granjeros cultivando la tierra. Los granjeros deben protejer sus cultivos, construir cercas, guardar la comida restante, y ser capáz de intercambiar ese restante. Por lo tanto la agricultura arroja ideas como posesión, control, cambio, reglas y ley. Los granjeros comen una dieta más monótona de cultivos plantados y animales domsticos, opuestamente a la dieta variada que se encuentra en los cazadores y recolectores de temporada. Los efectos negativos en la salud y esperanza de vida en muchas culturas fue enorme. La práctica de yoga puede haberse acentuado como cura para este nuevo estulo de vida: una forma de reconectarse con las fuerzas naturales y ritmos que fueron olvidándose.

Las herramientas del yoga son las que ya conocemos: cuerpo, respiración y mente. Algunas de las perspectivas posteriores del yoga del cuerpo son muy distintas a la idea de la autodivisión del espíritu que se ha inculcado profundamente en nuestra cultura, actitud y comportamiento.

La palabra yoga puede significarse ``unión'' o una aplicación de significados; en este caso, significa la reconexión con algo. Recordemos que \textit{algo} ya está. No hay separación del espíritu de la naturaleza, excepto en nuestra mente. Cualquier búsqueda interior nos llevará de nuevo al sentimiento de ser. Las ideas que la mente genera y que pone en el cuerpo, al igual que cualquier cosa que se encuentre en la tierra, se encuentran conectadas a qu\'e tanto se toma de nuestra conexión con los elementos primarios de los que nos generamos.

\section{Un resumen histórico de la filosofía del yoga}
El origen específico del yoga sigue siendo un misterio. Existe evidencia que indica que el yoga pudo existir desde 2500-1500AC, en la región de la India ``Indus Valley''. Esculturas de figuras sentadas en lo que parece postura de loto de esta era han sido encontradas, pero debido a que la leyenda que acompaña estas figuras es desconocido, no es posible determinar con ninguna certeza si las esculturas son representaciones de una postura de yoga, o simplemente una forma de sentarse en el suelo. Lo que está claro es que desde tiempos antiguis ha existido el conocimiento de que la conciencia humana es vasta, puede ser explorada, y desde la exploración interior ha sido revelada sabiduría sobre la condición humana, el universo, y nuestro lugar en el.

En los primeros siglos de los primeros milenios AC, dos ramas de cultura existieron en India: v\'edicos y no v\'edicos. Los \textit{Vedas} contenían textos sagrados de sabiduría revelada, o Sruti, que significa ``lo que es escuchado de una fuente superior''. Los cuatro \textit{Vedas} componen las escrituras más antigual de la f\'e Hindu. La cultura india no v\'edica incluía Jainismo y Budismo, ninguno de los cuales aceptaban la autoridad de los \textit{Vedas}, y consecuentemente evolucionaron en creencias separadas. Es importante recordar que, dentro de la cultura india, el conocimiento era transmitido de Guru a estudiante: el Guru transformaba su sabiduría en algo consistente y apropiado para ese estudiante. Dado este medo de transmisión de conocimiento, diferentes escuelas de filosofía se relacionaban e influenciaban la una a la otra en una forma mucho menos rígida de lo que imaginamos.

Es poco claro si el yoga evolucionó de la cultura v\'edica o no v\'edica. Los acadm\'emicos han observado que durante el período Sramanas (literalmente, ``aquellos que se esfuerzan'') estuvieron involucrados en austeridades (actividades practicadas por individuos que eran renunciantes y ascetas de la cultura no v\'edica).

El primer milenio AC fue un período de cambio dramático social y cultural en India. Cerca del s\'eptimo siglo AC, largos centros urbanos fueron tomando forma al norte de la India. Los centros urbandos crecieron donde había abundancia de comida y forma de guardarla. No enteramente dependiente de agricultura, y otros bienes comnzaron a ser producidos, el comercio evolucionó junto con las rutas de comercio, y las idea tambi\'en eran bienes intercambiables. Durante este período de rápido cambio, la filosofía evolucionó tambi\'en. Posiblemente como resultado de epidemias esparci\'endose de pueblos aislados a centros urbanos mayores, que resultaron en muerte esparcida, los filósofos comenzaron a cuestionarse el mero significado de la vida y la naturaleza de la existencia. Cerca de el s\'eptimo siglo AC el Upanishada más viejo se escribió, y fu\'e conocido como ``Vedanta'', el fin o culminación de los Vedas. Upanishad literalmente significa ``sentarse cerca''; esto da una pista de cómo la sabiduría era transmitida de maestro a estudiante en mucha proximidad. El maestro o Guru puede practicar la t\'ecnica o recitar la información a un estudiante, y una vez terminado, tomar su cabeza, agitarla y pedir al estudiante que repita el ejercicio para asegurarse de que no lo olvide.

Dos creencias importantes que influenciaron el desarrollo del yoga creció durante este período de cambio y reflexión, notablemente Samsara (el eterno ciclo de vida, enfermedad, vej\'ez y muerte) y Karma (la creencia que todas las acciones tienen fruto). Si cada acción tiene fruto, y no puedes experimentar todos los frutos de las acciones en una vida, entonces renaces. Con este concepto evolucionado de la existencia como un ciclo de vida, muerte y renacimiento. La siguiente más grande pregunta realizada fue, ``hay algo más?'' Existe alguna forma de salir de este ciclo de renacimiento? Vida, en la noción de Samsara, es vista como una infinita y frecuentemente dolorosa experiencia, una vista del mundo fundamentalmente negativo de algo para trascender, para escapar.

En este momento en la historia, la cultura indua fue característicamente masculina. Aunque las referencias a personajes femeninos en los Upanishads no fueron raros, las tradiciones del yoga fueron dominadas en estos tiempos por austeridad, y un caracter marcial de dominio de mente y cuerpo. Parte de los grandes esfuerzos y sacrificios hechos por los yoguis en este período requerían alejarse del mundo, el mundo de distracción. Una de las preguntas persistentes eran si un yogui en una caverna hacía algo para beneficio del mundo. Para lo mismo los ascets preguntaban, ``por qu\'e vivir en un manicomio?'', por lo que podemos observar que la pregunta de si es posible integrar el yoga en nuestras vidas ha existido desde el inicio.

Cerca del quinto siglo AC, el período preclásico, tres ramas del yoga se desarrollaron: las tradiciones upanishadas, budismo y jainismo. El \textit{Bhagavad Gita} fue escrito poco despu\'es del quinto siglo, y fue probablemente completado al final del milenio. Con este texto sagrado, no hay nada corto en la revolución de la filosofía yoguica. Existe una ampliación en la práctica del yoga. Distintas de formas de práctica se describen: Karma Yoga, o el yoga de la ación; Bhakti Yoga, o el yoga de la devoción; y Jnana Yoga, o el estudio de la sabiduría. De esta forma, la práctica de yoga y los estados más alto de conciencia se hacen disponibles para todos y renuncuar al mundo e irse a una caverna ya no es necesario. Se menciona tambi\'en dentro del texto que las mujeres no pueden ser excluídas de esta práctica.

Al inicio de los primeros siglos de la Era Común, nace una síntesis de las filosofías indias. Este es el yoga clásico, o el yoga de Paranjali. Los \textit{Yoga Sutras} atribuídos a Patanjali son una organización de filosofía yógui en cortos aforismos o verbos. Patanjali es con frecuencia comparado con el sistema de Ashtanga Yoga, o las siete ramas del yoga, pero en lo que Patanjali se encuentra interesado principalmente no es una aproximación secuencial a la iluminación, ni un sistema de ramas asendiendo sutilmente. Patanjali está interesado en una cosa: Samadhi. El Samadhi es el estado meditativo más alto en el que una persona trasciende su ego individual y se une al universal. En los \textit{Yoga Sutras}, el da la definición de yoga en el segundo sutra, ``Yoga citta vrrti nirodhah'' o ``el yoga es cece de las fluctuaciones de la conciencia''. El describe las diversas formas de lograr este estado. En la perspectiva de Patanjali, hay solo dos cosas a considerar: el Ser, o el testigo consciente llamado Purusa, y todo lo demás que es persivido por ese testigo. Todo lo demás, pensamientos, emociones e incluso memoria, reside afuera del testigo consciente. Esto es llamado Prakriti, o naturaleza.

Patanjali nos dice que en algún punto, de alguna forma, nos olvidamos de nuestra naturaleza esencial. Nos identificamos con el mundo físico, que es llamado ``Pracriti''. Desarrollamos formas de pensamiento, apego a nuestras ideas, vemos si somos hombres o mujeres, grandes o pequeños, y de alguna forma esas cosas se vuelven nuestra identidad. La inhabilidad de ver la diferencia entre nuestra naturaleza esencial (Purusa) y todo lo demás (Prakriti) es llamado Avidya, o ignorancia. Cómo superamos esta ignorancia fundamental? Patanjali dice que la única forma de ver la diferencia entre nuestro testigo consciente y todo lo que la conciencia percive es crear tranquilidad. Como un lago tranquilo sin olas ni ondulaciones, en esa calma podemos observar nuestra naturaleza esencial, disimulada por los movimientos de la mente.

Desde los tiempos de los Yoga Sutras, hubo un período de gran interacción y creatividad de filosofía yóguica. Cerca del sexto siglo nació el Tantra yoga. En el octavo siglo un maestro llamado Sankara formuló una escuela no dual (Advaita) de la folosofía v\'edica. La visión del mundo del Sankara, igual seguía lejos de ser prometedor. Su creencia era que, aún cuando sólo hay una realidad, debido a nuestra ignorancia (Maya), nosotros nos imponemos limitaciones y separaciones en lo que vemos, y como un hombre caminando en la oscuridad y ve una cuerda enroscada piensa que es una seríente, nos encontramos engañados por nuestra inhabilidad para ver claramente. La única forma de ver claramente en la oscuridad es iluminandola, al igual que en la vista del Sankara una cosa puede ser curada con lo opuesto; oscuridad por luz, ignorancia por conocimiento, y no otra cosa. El mundo de la forma no está valuado por su filosofía, es visto como una ilusión.

La práctica del Tantra Yoga evolucionó a trav\'es de los siglos, y encontró una formación posterior en la escuela del Shivaismo de Cachemira alrededor del octavo siglo DC. El Tantra Yoga, evolucionando cuando lo hizo, tuvo el beneficio de siglos de desarrollo y por lo tanto fue capaz de voltear atrás y transmitir el conocimiento previo en una forma más sofisticada. El Shivaismo de Cachemira acepta que con la filosofía no dual del Sankara V\'edico pero plantea la cuestión, ``si sólo hay una realidad, entonces qu\'e es la cosa llamada ignorancia?'' La filosofía v\'edica no puede responder a esta pregunta debido a al ignorancia, a Sankara, no es algo en sí misma, sólo una ausencia de conocimiento. El Shivaismo de Cachemira responde que que si existe solo una realidad, tiene que respetar que cualquier cosa que suceda (o aparente suceder) a esa realidad tiene que ser una \textit{operación} de esa \textit{realidad misma}. La razón por la cuál vemos una diversidad de formas incluso cuando hay sólo una realidad es que esa realidad no ha creado una ilusión, sino un mundo físico vibrando en ser. Somos vistos como la condensación de ``la fuente'', conteniendo el poder completo de esta fuente. La práctica de yoga es entonces un recuerdo de este potencial. No tenemos que correr del mundo. El mundo es donde nuestro yoga se lleva a cabo.

La filosofía del Shivaismo de Cachemira dicta la necesidad de la gracia de un guru para otorgar el salto espiritual llamado ``Shaktipat''. Sin la transmisión de energía, el estudiante no puede alcanzar la iluminación. Este dilema es manejado por otra escuela de Tantra llamada ``Shri Vidya'' o sabiduría favorable. La forma más reciente (que conozco) de acercarse es actualmente siendo enseñada por Dr Douglas Brooks (2010). Douglas aprendió la forma de yoga llamada ``Rajanka'' de su maestro Gopala Aiyar Sundaramoorthy. Rajanaka puede ser traducido como ``Pequeño Príncipe'' o ``el que es soberano para sí mismo''. En este modelo horizontal de yoga, no hay búsqueda por un estado de iluminación sino una expansión continua de entendimiento y apreciación. Conforme los yoguis deciden involucrarse con los regalos y oportunidades que la vida presentan, nuestra sensibilidad e intimidad con nosotros mismos y los demás incrementa a trav\'es de compartir experiencias, regalos únicos e ideas.

\subsection{Los upanishads}
Los Upanishads son una colección de cerca de doscientos enseñanzas incluyendo historias, metáforas e instrucciones de meditación. Este arreglo no-homog\'eneo refleja la transmisión naturalmente oral de las enseñanzas. La gran variedad de información en ls Upanishads no refleja la filosofía yóguica unificada. Sin embargo, una consideración profunda referente a la divinidad innata de humanidad es claramente presente.

La raíz del mundo Upanishad ``Upa, Ni, Shad'' significa literalmente ``Cerca, bajo, sentado''. Esto implica tambi\'en que las enseñanzas pueden no ser inmediatamente obvias y requieren estudio y dedicación para ser asimiladas. Los Upanishads fueron enseñanzas impartidas por las clases más altas de la sociedad, la clase guerrera/governante y la clase eclesiasta. Esas enseñanzas no estaban disponibles para otros en la sociedad.

Aunque los Upanishads son parte del cuerpo V\'edico, contienen un mensaje distinto. Veda literalmente significa ``conocimiento'', así que los \textit{Vedas} eran libros de conocimiento. El enfoque de los primeros vedas es cómo vivir una buena vida y la explicación de la realización correcta de los rituales. Las primeras enseñanzas V\'edicas no muestran inter\'es claro en liberación espiritual o yoga. Los Upanishads, forman parte de las posteriores enseñanzas V\'edicas, hablan de los conceptos de transmigración y renacimiento.

El proceso de comprensión de las enseñanzas de los Upanishads es tambin dado por:
\begin{itemize}
	\item Escuchar
	\item Contemplar
	\item Meditar
\end{itemize}

Esta metodología de aprender no es casual. Para escuchar una enseñanza (realmente escuchar) necesitamos abrirnos a la enseñanza completamente, con una mente de principiante. Se dice en los Upanishads que debemos escuchar ``como un venado escucha la música''. Si puedes imaginar la sensivilidad del oído de un venado y la cualidad de alerta del animal, la naturaleza de este tipo de atención es similar.

Despu\'es de escuchar completamente, contemplamos la enseñanza. Debemos contemplar las enseñanzas ``como una vaca mastica el pasto''. Una vaca continuará mascando el mismo bocado de pasto, preparándolo para la digestión por un buen rato. Durante el proceso, comenzamos a ver las enseñanzas como algo importante para nosotros, la incorporarlos como sabiduría y no sólo información, verlos desde diferentes puntos de vista.

\subsection{El bagavad Gita}
\subsection{Los Yoga Sutras de Patanjali}
\subsection{Vedanta - La filosofía No-Dual del Sankara}
\subsection{Kashmir Saivism y los Tattvas}
\subsection{Yoga Moderno}
\subsection{Cronología del Yoga}
\section{Dieta y Estilo de Vida}
Es difícilmente necesario meditar en una caverna remota para crear un estilo de vida que ayude la práctica. Las prácticas de yoga presentadas en este manual son pensadas para ser integrales y adaptativas al mundo moderno. Entre el 1000 y 1400 DC, en Índia, la región de Kashmir vió una revolución en el penzamiento yóguico. Esos practicantes de yoga fueron, en realidad, amas de casa, esposos y esposas que tomaron la práctica de yoga y la convirtieron en la fábrica de sus vidas. Esta aproximación fue una no renuncia, entre otras prácticas, y fue después llamada tantra, que literalmente significa "tejido". Los yoguis tomaron el entendimiento moral más clásico de yoga y lo recrearon, redefiniéndolo en una práctica más significativa para ellos mismos.

"Una mente aguda, un corazón suave y un cuerpo vibrante" (John Friend) son entre otras cualidades de un maestro de yoga competente. Estas cualidades son apoyadas y fomentadas por:
\begin{itemize}
	\item descanso adecuado;
	\item relaciones alentadoras;
	\item dieta apropiada; y
	\item práctica personal y estudio.
\end{itemize}

Enseñanr una clase mientras te encuentras cansado, hambriendo, agitado o desconectado de tu práctica no es una experiencia de crecimiento. Si estás apropiadamente preparado, nutrido con comida, lleno de prana después de una buena práctica de yoga, bien descansado y sintiéndote apoyado, con seguridad darás una buena clase. El negocio del yoga y tu estilo de vida no pueden ser separados tal como en alguna otra ocupación. Como maestro, eres un ejemplo viviente de cómo la práctica de yoga funciona.

\subsection{Dieta}
La ecuanimidad de la mente ha sido siempre referida como una de las claves del yoga. "Yoga no es para quienes comen mucho, o no comen" (Upanishads). La dieta afecta directamente a nuestro ánimo y a nuestros cuerpos. Simplemente prestar atención a lo que comemos y a cuánto comemos es otra práctica de yoga, embebido en una práctica más larga. Extrema austeridad respecto a la dieta puede ser una forma que el ego toma para crear más separación. Como sea, comiendo una dieta balanceada con muchos vegetales y frutas es accesible para la casi todo. Disminuir los estimulantes como alcohol, azúcar y posiblemente cafeína creará un ánimo más estable y una atmósfera corporal donde la consciencia plena es más fácil de alcanzar. Comer una dieta vegetariana es considerado prificador por muchas ramas de la tradición del yoga. Comer la carne de un animal que ha sido mantenido en condiciones de vida no dignas será al finalde menor valor nutrimental que animales con libre albedrío. Nosotros, como seres sensibles, podemos elegir qué es útil para nosotros y cómo nuestras elecciones afectan a nuestros seres. Come con amigos o amados tanto como sea posible, dando bendiciones antes de comer.

\subsection{Sueño}
Hay una pulsasión universal llamada "Spanda". Desde esta pulsación viene la manifestación dual de la naturalesa; frío y calor, macho y hembra, luz y oscuridad. Una lugar fresco, oscuro y tranquilo y una cama cómoda para dormir recupera la energía vital del cuerpo y es el complemento natural a la actividad física. El ciclo del día es un modelo de vida, justo como la práctica del yoga y la postura de descanso Savasana es el modelo del día. Vive pleno y duerme profundo. Permite al día previo disolverse, brindando tu atención tan completamente como sea posible al momento presente.

\subsection{Práctica}
Enseñar yoga requiere la asimilación e incorporación de las enseñanzas, presentadas en tu propia voz y en tu propio modo. Conforme crecemos como maestros, agregamos nuestra propia voz al linaje de maestros. Es así como la tradición del yoga se mantiene viva y evoluciona. Al practicar por tu cuenta, tu sabiduría interna comienza a emerger. Al aceptar el conocimiento que ha sido revelado y al honrar a tus maestros, creces individualmente como maestro, y simultáneamente te conectas más con la energía universal. Elije un tiempo y lugar para la práctica diaria. El cuerpo cambia día a día. Algunos días nuestra energía está alta y el cuerpo se siente como una pluma. Otros días la energía corporal está menos disponible y se sentimos como concreto húmedo. Otros días de energía alta, tu práctica puede ser una celebración que es dinámica y retadora. En los días difíciles, desarrolla una secuencia de posturas más restaurativas para ayudar a aliviar la fatiga. Aún así no existe razón para dejar de practicar, y tus habilidades de enseñar a una variedad de estudiantes incrementará.

\subsection{Relaciones}
Todo en el universo es inatamente divino. La práctica de yoga no es algo que pase solamente en el tapete. Permanecer completamente consciente en relación con otros seres humanos es tal vez uno de los aspectos mas difíciles de la práctica. Es aquí el donde nuestra habilidad de recordar nuestra divinidad innata, y la divinidad de otros es revelada. Percibir patrones en tus relaciones. Percibir tus reacciones y deseos habituales. Abrazar las relaciones como parte de, no separarlas de tu práctica. De esta forma observarás la naturaleza sagrada de otros. Honra a aquellos que amas y aquellos que has amado.

\section{El principio de atracción}
\subsection{Yoga y el principio de atracción}
\subsection{Reciviendo}
\subsection{Práctica}
\section{Ética}
\subsection{Yamas y Niyamas}
\subsection{Mas de Ética de un maestro}
\section{Luz y Oscuridad}
\section{Mantras}
Un mantra es una invocación de sonidos sagrados, y como tal es otra forma de vibración en la forma de sonidos organizados. Repetir un mantra es una forma de yoga en sí mismo y es la práctica principal en Mantra Yoga. El efecto de la vibración importa. Incluso un instrumento musical, si está hecho de material orgánico como madera, absorverá las vibraciones resonando a travé de el. Si el instrumento se mantiene en sintonía y se toca regularmente, el tono del instrumento se profundiza, se vuelve más bello y permite acarrear la canción del música completa y acertadamente. El cuerpo humano en el caso de la repetición del mantra, es el instrumento, que toca la canción de la divinidad.

El mantra Gayatri es primero recordado en el \textit{Rig Veda} escrito en Sánscrito hace cerca de 2,500 a 3,500 años, y de acuerdo a algunas fuentas, puede haber sido cantado generaciones antes de eso.

\begin{multicols}{2}
	El Gayatri Mantra:\\*
	Om bhûr bhuva sva\\*
	tát savitúr várenyam\\*
	bhárgo devásya dhîmahi\\*
	dhíyo yó na prachodáyât
	\columnbreak
	\textit{Phonetic Pronunciation}
	Om burr buva-ha sva-ha\\*
	Tat sa-vi-tour vara-en ya-hum\\*
	Bar-go de vas-ya de my-hee\\*
	De yo-yo na pra-show-da-yat
\end{multicols}

\section{Visión general de los Estilos de Yoga}
Una variedad de caminos del yoga continúan entrelazándose e informándose entre sí. Algunos ejemplos son:
\begin{itemize}
	\item \textbf{Anusara Yoga:} Desarrollado por John Friend en 1997; unirica la filosofía enfocada al corazón Tántrica con principios de alineación bio-mecánicos.
	\item \textbf{Ashtanga Vinyasa Yoga:} Desarrollado por T.Krishnamacharya y su estudiante Pattabhi Jois; acercamiento sistemático y secuencial a la práctica de los asanas donde las posturas son separadas en series. Vinyasa, una conexión energetica de un asana a otro, es usado para crear y mantener calor y un movimiento de estado meditativo.
	\item \textbf{Ashtanga Yoga:} Llamado así por Baba Hari Dass, despus del camino de ocho pasos de Patanjali, no confundirlo con Ashtanga Vinyasa Yoga.
	\item \textbf{Bikram Yoga:} Una secuencia de veintiseis posturas en un cuarto calentado a 100 grados Fahrenheit.
	\item \textbf{Iyengar Yoga:} B.K.S. Iyengar, otro de los estudiantes de Krishnamacharya, refinó el aprendizaje de su gurú despus de mudarse a Pune. Avandonó el estilo Vinyasa y se enfocó en las enseñanzas de la salud, alineación estructural y beneficios terapeuticos de las posturas.
	\item \textbf{Kundalini Yoga:} Despertar energía, Kundalini yoga llegó al oeste en 1969, cuando Sikh Yogi Bhajan desafió la tradición y comenzó a enzeñarlo públicamento. Esta práctica es designada para despertar la energía Kundalini, la cuál es almacenada en la base de la espina y en ocasiones representada como una serpiente enroscada. Kundalini mezcla cantos, prácticas de respiración y ejercicios de yoga. El enfasis no es en los asanas, sino en los cantos y respiraciones.
	\item \textbf{Mysore Style:} Nombrado así en honor a la ciudad en india donde Pattabhi Jois enseña el metodo Ashtanga Vinyasa; una práctica autoencaminada con supervición y ajustes físicos de un instructor.
	\item \textbf{Vijnana Yoga:} Una práctica de Hatha Yoga desarrollada por Donna Holleman y Orit Sen-Gupta, basada en siete \"principios vitales\" diseñados para usar el cuerpo para explorar los más profundos niveles de nuestro ser.
	\item \textbf{Viniyoga:} Esta forma gentil de flow yoga pone gran \'enfasis en la respicación y cordina respiración con movimiento. El movimiento fluido de Viniyoga o Vinyasa es similar a la dinámica de la serie de poses de Ashtanga, pero ejecutado con una gran reducción de paz y nivel de estr\'es. Las posturas y secuencias son elejidas para encajar en las habilidades del estudiante. Se enseña al estudiante cómo aplicar las herramientas de yoga: asana, cántos, pranayama (control de la respiración), y meditación, en una práctica individual. Desarrollado por T.K.V. Desikachar, el hijo de Krisnamacharya (maestro de algunos grandes instructores de yoga incluyendo Iyengar y Pattabhi Jois), Viniyoga pone menos estres en uniones y rodillas manteniendo las posturas con una ligera flexión en las rodillas. Viniyoga es considerado excelente para principiantes, y es incrementalmente usado en ambientes rerap\'euticos.
	\item \textbf{Yin Yoga:} Un t\'ermino creado por Paul Grilley para describir una forma de práctica con un \'enfasis en posturas mantenidas por mucho tiempo, usualmente sentados, boca abajo o boca arriba. Yin Yoga se enfoca en fortalecer y alargar el tejido conectivo, que en turno, a partir de líneas meridianas, tiene un efecto óptimo en el funcionamiento de los órganos.
\end{itemize}

\subsection{Viendo la Imagen Completa}

\section{El negocio del Yoga}
\subsection{Marketing}
\subsection{Yoga en casa}
\section{Sencillez Voluntaria}
\subsection{Principios de Sencillez}
\subsection{Acercamiento a Sencillez}
\subsection{Un Acercamiento al Lado Financiero}

\chapter{Técnicas de Entrenamiento y Práctica}
\section{Técnicas de Asana}
\subsection{Alineamientos Fundamentales}
El cuerpo humano tiene una alineación óptima. Cuando el cuerpo se mueve hacia o en un alineamiento óptimo, hay un incremento en la libertad de movimiento de las articulaciones y más energía disponible en el cuerpo, dado que el cuerpo no está peleando consigo mismo y le es pocible moverse libremente. El dolor es tambi\'en reducido o eliminado cuando los huesos y tejidos del cuerpo se encuentran en una relación cooperativa.

La habilidad para realizar una asana puede variar enormemente, dependiendo de facotes como la estructura osea, lesiones previas, edad, nivel de enería y empeño del estudiante. Comenzar una práctica de asana como disciplina de consciencia y no violencia (especialmente a los practicantes) es un comienzo fundamental para una práctica libre de dolor.


\subsection{Biomecánicas Holísticas}
\begin{itemize}
	\item Holistico - ``visto como un todo, integrar.''
	\item Biomecánico - ``el estudio y la aplicación de fuerzas físicas benficas en seres sensibles.''
	\item O: ``No uses el cuerpo en esa postura\ldots usa la postura para entrar en el cuerpo''
\end{itemize}

\textbf{Biomecánizas holísticas en pocas palabras:}
\begin{itemize}
	\item Reconoce que cada parte del cuerpo humano (cuerpo, mente y emociones) es involucrada en cualquier actividad, sea una postura de yoga, una discusión o comer una comida.
	\item Reconoce el potencial para la salud a trav\'ez de la aplicación apropiada de conciencia y forza la reación de relaciones armoniosas entre articulaciones, músculos, huesos, mente y emociones.
	\item Reconoce las limitaciones estructurales del cuerpo y trabaja dentro de esos límites.
	\item Reconoce que cada persona es diferente (músculos, huesos, mente y corazón). La aplicación de cualquier t\'ecnica debe ser adaptativa a las necesidades individuales de esa persona.
\end{itemize}

Una obstrucción primaria para los estudiantes de Hatha Yoga (yoga físico) es intentar interpretar el lenguaje de un maestro conforme dirige el movimiento de ciertas partes de tu cuerpo en formas específicas. El distinto acercamiento a las posturas de yoga puede dar algunas veces conflicto verbal de instrucciones a elos estudiantes. El siguiente trabajo intenta ayudarte como estudiante a entender el trabajo de tu cuerpo en acción.

Desde el trabajo del maestro T. Krishamacharya y otrs de 1930 en adelante, el potencial de curar al cuerpo y obtener un completo rango de mobilidad y vibración física a partir de la práctica de Hatha Yoga ha sido reconocida. El buen alineamiento y la aplicación de fuerza apropiada en el cuerpo físico es la llave. Sincronizando la mente con el cuerpo e incluso emociones es lo que yo llamo ``Biomecánicas holísticas''.

Las biomecánicas holísticas toman la perspectiva del yoga llamada ``Tantra'', y acepta que el cuerpo no es simplemente un vehículo inherte para un espíritu separado, sino un universo inteligente en miniatura. Fuerzas en nuestro universo interactúan en ciertas formas predecibles que son reflejadas en nuestro cuerpo, mente y emociones. La aparente separación de cuerpo, mente y el mundo exterior es una ilusión.

Los movimientos y direcciones de energía o prana en el cuerpo han sido descritos anteriormente en los Upanishads, una antigua compilación de sabiduría yoguica obtenida por varios autores. Muchas escuelas modernas de yoga utilizan antiguos entendimientos de las energías sutiles en el cuerpo, y las describen en diferentes formas.

La perspectiva oriental de las energías sutiles y de cómo el prana se mueve benficamente dentro de el cuerpo junto con algunos modernas (generalmente occidentales) entendimientos de las relaciones de músculos y huesos, conforman la biomecánica holística.

\textbf{Methodology}
Cuando se realizan los estramientos terrenales que llamamos ``asana'', cualquier forma de nuestro cuerpo tiene el potencial de beneficiar la elasticidad y fuerza, o causar lesiones. Si no hacemos nada, sólo nos sentamos en el sofá, por ejemplo, hay poca probabilidad de algún daño inmediato en el cuerpo.

Existe tambin una posibilidad muy alta de eventualmente perder salud debido al atrofiamiento de tejidos del cuerpo, y correspondiendo contraindicaciones de la mente y el cuerpo emocional. La salud se encuentra entre hacer nada y hacer mucho. Cuando el cuerpo se mueve con habilidad y armonía, la intensidad del ejercicio puede ser incrementado con seguridad y los beneficios incrementan tambin. Un ejercicio hecho sin cuidado tiene más alto potencial de lesiones y menos potencial de beneficio, poque mejoramosen las cosas que practicamos.

Las siguientes decuencias de acciones son una síntesis, tomada del trabajo de maestros de yoga, fisicoterapeutas occidentales, observación de animales y práctica personal. Estas tcnicas no pertenecen a nadie, son parte de nuestra cultura e intelectual común. Cómo puedes saber si la t\'ecnica de un fisioterapeuta está funcionando? Se siente bien. Ese reconocimiento es innato, como el deseo de estirarse y descansar, ese conocimiento es intrínseco en todos los seres. (Opuesto de sedentario y sobre-activo)

\textbf{Samasasthiti (pronunciado, sama stee tee hee)}
Samasthiti es la palabra en sánscrito que significa ``esparcir la luz de la consciencia a travs del cuerpo.'' Samasthiti es el estado de consciencia dentro de la postura inicial de levantarse listo frente al tapete, y en cualqueier postura, demostrar un estado que permite la concentración luminosa y compasiva.

El resultado emocional e intelectual de esta ación es una receptividad, físicamente una falta de armonía exterior y sin embargo una fuerte base de las partes de tu cuerpo que tocan la tierra.

\textbf{Integración}
Como "condensaci\'on de consciencia", el practicante dirije los tejidos del cuerpo como uno hacia el centro, localizado a lo largo del eje de la columna vertebral, entre los hombros y las caderas.

Los hombros y caderas se mueven hacia atras en su "hogar" estructural:
\begin{itemize}
	\item los muslos se mueven hacia los izquiotiviales, suavizando el frente de las ingles; y
	\item los hombros descanzan c\'omoda y fuertemente hacia la parte trasera del cuerpo.
\end{itemize}
(Tadasana de nuevo, energ\'ia moviendose hacia adentro)

\textbf{Expansi\'on}
Una vez que la integración se ha llevado a cabo, la energia sublime del cuerpo que se encuentra en distintas formas, se basa en previa integración muscular, fluye del centro del cuerpo a los ejes de la espina, desde abajo de las piernas como las raíces de un árbol, hasta la cima de la espina, brazos y cabeza, como ramas de un árbol.

(energía movi\'endose hacia afuera)

En pocas palabras, la práctica de esas tres acciones: Suavizar, Flexionar y Estirar. Estas acciones pueden ser realizadas en cualquier postura, posición o actividad dentro de una práctica formal de yoga o en actividades diarias incluso, como lavar los platos.

Los beneficios de una fuerza incrementada, un mayor rango de movilidad y una sensación de serenidad aparecen porque el cuerpo, la mente y las emociones son conducidad en la fábrica de nuestro mundo, de nuestro universo. No estamos separados de todo lo que vemos, así que cuando reconozcamos nuetra conección fundamental, y nos comportemos como todas las cosas en el universo lo hacen (pulsando con expansión y contracción) entonces encontraremos esta relación armoniosa. No una relación de inactividad o separación, sino una de participación con la vida.

\textbf{Dolor contra intensidad:}
Para un practicante principiante, la sensación experimentada durante una práctica de asana es con frecuencia desconocida. Con más experiencia, una diferenciación puede ser encontrada entre dolor e intensidad. Dolor repentino, sensaciones desagradables, especialmente alrededor de las articulaciones no debe ser ignorado. El cuerpo envía una señal de desalineación o desconexión que podría ser dañina.

Sensaciones de estiramientos intensos a un músculo puede ser interpretado como dolor, pero la sensación es muy distinta. Con frecuencia un practicante se encuentra en control de la cantidad de sensaciones experimentadas, como en una flexión sentado hacia el frente. Estas sensaciones de intensidad son intrínsecamente parte de la práctica. Generalmente respirar en las posturas intensas recoge la mente del cuerpo y la resistencia decrementa.

\subsection{Terapia Estructural}

La terapia estructural es la aplicación consciente de las holisticas biomecanicas de un practicante a su cliente, con una completa participación del cliente. Tocar con vigor establece resonancia entre el practicante y el cliente, antes y durante los ajustes realizados. Con práctica, el practicante aprende a diferenciar entre tipos de resistencia: compresión, tensión, músculo y tejido conectivo. Junto con la resistencia física, pueden ser encontrados patrones de resistencia en los cuerpos emocionales y mentales.

La realización de ajustes será de poco valor a largo plazo a menos que se ilustre al cliente ómo realizar una buena alineación, con apropiada integración y expansión. La idea es que el cliente gane consciencia kinestésica de cómo crear alineación curativa para ellos mismos.

\textbf{Qué tanta presión?}
Cuando se aplica un ajuste, primero debes conectar con la parte del cuerpo que vas a ajustar con seguridad, incluso presión. Moverás piel, músculo y hueso como una unidad. Poca presión será inefectiva. El ajuste debe comenzar desde la más baja intensidad e ir incrementando hasta el máximo en un período de 3 o 4 segundos, dl ajuste mismo puede durar desde 5 segundos hasta 1 minuto.

Mantén tu atención en la cara del cliente buscando signos de inconformidad, dolor o alivio. Cuando trabajes con piernas o cadera, tus manos y brazos son usualmente menos fuertes que con lo que estás trabajando. Cada cuerpo es distinto y la sensitividad o sensación es distinta de persona a persona. Mantente comunicado con tu cliente, revisando si está experimentando dolor o alivio.

\textbf{Areas clave de imbalance}

Desalineación ocurre donde los huesos se encuentran. La desalineación habitual ocurre en el estilo de vida: sentado por largos períodos, movimientos repetitivos que crean imbalance muscular, y posiblemente nuestra forma de estar erguidos y caminar como tales.

Es posible que desde una perspectiva evolutiva, aún nos encontremos en evolución física y que no estemos por completo adaptados a una posición erguida. La parte superior de nuestro fémur encaja mejor en el hueco de la cadera en una posición inclinada en lugar de en una orientación vertical, la parte exterior de nuestras piernas se encuentran mas rígidas cuando nos encontramos de pie igualmente. Nuestros muslos interiores son difíciles de involucrar muscularmente para rebalancear la rotación común del femur debido a la rigidez de las piernas exteriores
éáíñúó
\subsection{Técnicas de Asana: Categorías de Posturas}
\subsection{Forma y Acción}
\subsection{Navegando el Tapete}
\subsection{Sacro}
\subsection{La Práctica de Asana}
\subsection{Yin y Yang}
\subsection{Polaridades de la Energía Física}
\subsection{Fuerzas Opuestas}
\section{Téecnicas de Purificación}
\section{Meditación}
\section{Pranayama}
\subsection{Nadi Shodhana}

\chapter{Anatomía y Psicología}
\section{Chakras}
Un chakra es el centro de actividad que recive, procesa y expresa energía de fuerzas vitales o prana. La palabra sánscrita chakra se traduce como ``rueda`` o ''disco`` y se refiere a una esfeza giratoria de bio energía. Hay, en este modelo particular, siete chakras posisionados en una columna de energía de la base de la espina a la coronilla de la cabeza. Los siete chakras mayores que se correlacionan con los estados básicos de consciencia. Como transformadores de energía, ellos bajan de la energía universal de consciencia al plano físico. De esta forma nosotros estamos conectados a la fuente de energía, se encuentra disponible para nosotros en diferentes formas de energía. Similar a los conectores elctricos, diferentes formas de energía son apropiadas para diferentes usuarios.\\
Los colores asociados con los chakras son tambi\'en una forma de energía. Color es energía expresada como una onda de luz que podemos ver (existen ondas de luz que no podemos ver con nuestros ojos). Existen tambi\'en sonidos correspondientes, asociados a cada chakra. De nuevo, el sonido es solo otra forma de energía vibrante a distintas frecuencias.\\

\subsection{Primer Chakra: Muladhara}
Tierra, identidad física, orientado a autopreservación. Color rojo. Localizado en la base p\'elvica. Este Chakra forma nuestra base. Está relacionado a nuestos instintos de supervivencia y a nuestro sentido de apego y conexión a nuestros cuerpos y al plano físico. Idealmente este chakra nos otorga salud, prosperidad, seguridad y presencia dinámica.\\
\subsection{Segundo Chakra: Svadhisthana}
Agua, identidad emocional, orientado a autogratificación. Color naranja. Colocado en el área del sacro. Este chakra está relacionado al elemento agua, y a las emociones y sexualidad. Se conecta a nosotros a trav\'es de los sentimientos, deseos, sensaciones y movimiento. Idealmente este chakra nos brinda fluidez y gracia, profundidad en los sentimientos, realización sexual y la habilidad de aceptar el cambio.
\subsection{Tercer Chakra: Manipura}
Fuego, identidad individual, orientacion a autodefinición. Color amarillo. Localizado en el plexo solar. Comanda nuestro poder personal, deseo, autonomía y metabolismo. Cuando se encuentra saludable, este chakra otorga energía, eficiencia, esponaneidad, y poder no-dominante.
\subsection{Cuarto Chakra: Anahata}
Aire, identidad social, orientado a autoaceptación. Color verde. Localizado en el corazón. Está relacionado a la verdadera compasiń y es el integrador de opuestos: izquierdo y derecho, arriba y abajo, hombre y mujer, expansión y contracción. Un cuarto chakra saludable nos permite amar profundamente, sentir empatía y tener una profunda sensación de paz y concentración.
\subsection{Quitno Chakra: Vishudha}
Sonido, identidad creativa, orientado a la autoexpresión. Color azul. Este chakra se encuentra localizado en la garganta y por lo tanto está relacionado con la comunicación y creatividad. Aquí nosotros experimentamos el mundo simbólicamente a trav\'es de vibraciones, tales como la vibración del sonido representando lenguaje.
\subsection{Sexto Chakra: Ajna}
Luz, identidad de estereotipo, orientado a autoreflexión. Localizado en el entrecejo (tercer ojo). Color índigo. Está relacionado al acto de ver, tanto física como intuitivamente. Como tal abre nuestras facultades psíquicas. Cuando se encuentra saludable nos permite ver claramente, completamente, permitiendonos \"ver la imagen completa\".
\subsection{S\'eptimo Chakra: Sahasrara}
Pensamiento, identidad universal, orientado al autoconocimiento. Color violeta. Localizado en la coronilla de la cabeza. Este chakra se relaciona con la consciencia como consciencia pura. Es nuestra conexión con la Consciencia pura al nivel universal. Cuando se desarrolla, este chakra nos otorga conocimiento, sabiduría, entendimiento, conexión espiritual y dicha.
\\
Una práctica apropiada de asanas puede ayudarnos a balancear estas energías sutiles del cuerpo. Los chakras se balancean por medio de unir las energías de Siva (consciencia) y Shakri (creción). Cuando se encuentran balanceados, cada chakra funciona óptimamente, dándonos acceso espontáneo a todas las formas de energía corporales. La meditación de chakra es una excelente forma de mejorar tu entendimiento de esos centros, al igual que una dieta propia y elecciones de estilo de vida.

\section{Los Vayus}
El entendimiento yoguico del cuerpo es experienciado en lugar de teótico. El entendimiento fundamental es que el cuerpo es una expresión de la fuenta universal, como una onda es una expresión del oc\'eano.  Dentro de esta expresión universal hay formas en que la energía vital fluye, como corrientes en un cuerpo de agua. Los yoguis dieron nombres a estas corrientes, y varias escuelas de yoga suelen nombrar a esas expresiones de energía de diferentes formas. La fuerza esencial del cuerpo es conocida como prana, la menor unidad de fuerza vital.\\
Prana y la respiración se encuentran intimamente unidos. Prana mueve la respiración. Sin fuerza vital, no hay respiración, no hay otra forma. Podemos interactuar de cierta forma con esta fuerza vital conocida como prana al sentir y manipular la respiración (incluso deteniendo la respiración por un período de tiempo). Dentro de un ciclo respiratorio, prana se vuelve perceptible.\\
Los cinco Vayus principales:
\begin{itemize}
	\item Prana - la asendencia del flujo de energía, que puede ser sentido en la inalación.
	\item Apana - la desendencia del flujo de energía, percibido en la exalación.
	\item Samana - la corriente de energía que se digire, se representa hacia nuestro centro.
	\item Udana - la corriente de energía que se consume conforme se expande a las extremidades desde nuestro centro.
	\item Yyana - la corriente integradora de energía que mantiene el equilibrio.
\end {itemize}
Conforme un principiante mueve el cuerpo en una práctica de asanas, es en un inicio usualmente una experiencia ``corporal exterior''. Posturas de formas básicas, sentimientos de rigidez o fatiga en partes del cuerpo son notables. Conforme la práctica continúa, se hacen conscientes  sensaciones más sublimes. Es aquí cuando el trabajo con los vayus puede comenzar.\\
Prana puede ser sentido como una fuerza que sube en la inalación cuando los brazos son levantados sobre la cabeza.\\
Apana puede ser sentido hacia abajo en la exalación cuando los brazos se colocan hacia los lados del cuerpo.\\
Samana puede sentirse como una fuerza integradora, dibujando flechas en el cuerpo indicando al centro.\\
Udana puede sentirse como una expresión sublime de expansión, o hacia afuera muviendo la energía desde el centro del cuerpo.\\
Vyana puede ser experimentado al esparcir consciencia atrav\'es del cuerpo, notando cómo diferentes partes pueden comunicarse y son mantenidas juntas.
\section{Compresión y Tensión}
\section{Anatomía Funcional}
\subsection{Huesos y Articulaciones}
\subsection{La Espina}
\subsection{Muscúlos}
\subsection{Muscúlos y Posturas}
\subsection{Yoga y Posturas}
\newpage
\section{Los Bandhas}
Bandha significa \lq\lq candado\rq\rq. Este tipo de candado, en lugar de cerrar, como el tipo de candado donde se requiere una llave para abrir, fue de hecho un termino figurado. Estos candados son como una zanja usada para dirigir el agua a diferentes partes de un campo. Bandhas en el cuerpo son usados para dirigir la energía tanto física como energeticamente. Físicamente, los bandhas funcionan para mantener estimulados y armonizados nuestros órganos internos. Energeticamente, asisten al movimiento del prana, o energía en el cuerpo. Hay tres tipos de bandhas usados en la práctica de las asanas:

\subsection{Mulabandha}
Localizado entre el ano y los genitales, es el músculo perineo para los hombres. Para las mujeres está ubicado cerca del límite superior del cuello del útero. La activación del Mulabandha no es una fuerte contracción forzando los músculos que lo rodean, es mas sutil que eso. Mulabandha puede ser experimentado activando los músculos hacia atras, incrementando la curvatura lumbar en la espina, despus permitiendo al coxis alargarse, estimulando al abdomen a activarse y la base de la pelvis a levantarse.

Activar los muslos hacia atras acomoda las cabezas del femur hacia atras y crea una expansión en el área pélvica.
Bajar el sacro reafirma la carne de los glúteos. El abdomen bajo se mueve de la pubis al ombligo.

La sinergía creada por esos dos complementarias, y a la vez opuestas fuerzas, crean Mulabandha. En lugar de endurecer o relajar en el área pélvica, un levantamiento es creado similar a lo que sería el último medio centímetro de una malteada por un popote.

\subsection{Uddiyana Bandha}
Localizado ligeramente abajo del ombligo, Uddiyana Bandha significa \lq\lq Volar hacia arriba\rq\rq refiriendose a su efecto en el prana. Este segundo banda es activado en una forma parecida a Mulbandha, con un mínimo de endurecimiento externo o contracción. En el proceso de crear este candado, el centro del plexo solar es llevado adentro y hacia arriba en un levantamiento abdominal y es cuando se encuentra la activación. En expresión completa es llevado acabo exalando completamente y luego llevando el abdomen bajo hacia adentro y arriba, mientras se eleva el diafragma. Este nivel de Uddiyana Bandha será usdo en la práctica de la retención de exalación en Prnayama, pero debido a la incapacidad de inalar mientras se ejecuta este nivel, simplemente mantener tranquilidad cerca de tres dedos debajo del ombligo otorga espacio para que el diafragma baje durante cada exalación. Conforme el diafragma baja, la respiración es impulsada a moverse hacia las costillas, espalda y pecho. En cada exalación los músculos abdominales promueven a completar el vaciado de los plmones. El proceso toma práctica, y las sutilezas de la relación entre respiración y bandhas debe ser explorada experimentalmente.


\subsection{Jalandhara Bandha}
Este candado es creado levantando y girando los hombros  hacia atras para primero ampliar y luego levantar el pecho. Despues la parte trasera de la cabeza se extiende hacia el cielo y la barbilla se mueve en una contracción, la cuál es formada donde dos huesos de la clavícula se encuentran. El candado ocurre espontáneamente en algunas posturas como el parado de hombros, pero no es usado tan ampliamente como los otros dos candados.

\section{La Respiración}
\subsection{Respiración Ujjayi}
\section{Elementos de la Naturaleza}
\subsection{Cualidades Caracteristicas de los Cinco Elementos}
\subsection{Ayurveda}
\section{Los Cinco Koshas}



\chapter{Metodología de Enseñanza}
\section{Enseñando a dirigir}
Antes de hablar acerca de la filosofía de yoga dentro de una clase o incluso enseñar una postura, debes ser capaz de dirigir el movimiento de un estudiante con claridad y unas cuantas palabras.

\textbf{Ejercicio:}

Elige diariamente una actividad como abrir una puerta, quitarte un zapato, o rascar tu pierna. Escribe un guión para esa acción de tal manera que pueda ser llevada a cabo sin interpretación. Un ejemplo de una instrucción con escasez de claridad sería:

``Camina y toma la perilla de la puerta y abre la puerta.''

Esta instrucción funcionaría, pero sólo porque la persona a qui\'en instruyes ha abierto muchas puertas antes y sabe como, una mejor instrucción sería:

``Colocado a un brazo de distancia de la puerta, coloca tu pie izquierdo al frente y toma la perilla de la puerta con tu mano derecha. Rota la perilla en el sentido del reloj hasta que se detenga, luego colocando un poco más de peso en tu pi\'e derecho, suavemente jala la perilla hacia ti, abriendo la puerta.''

Enseña a un amigo a usar distintos objetos y movimientos diariamente, asegurándote de que tu amigo no interprete las instrucciones a su manera, sino que haga exactamente lo que instruíste. La experiencia puede ilustrar qu\'e tan difícil puede ser mantener tu voz con claridad y fuerza.

\section{Agregando Contenido}
Comienza con sencillez. Simple es claro. Claro es bueno, Practica tus enseñanzas simplemente instruyendo la respirazión de tu propia práctica del Saludo al Sol, una respiración por movimiento. Desde ahí, utiliza instrucciones fundamentales que est\'en tan bien establecidas que te permitan tener libertad creativa.

\subsection{Nivel 1 - Respiración}
Conforme instruyes verbalmente la respiración en tu propia práctica, escucho el tono de tu voz. Observa la sincronización y el paso de la instrucción simple de inhalar, exhalar. Crea un flujo tranquilo en tu cuerpo y tus palabras. Di la palabra ``inhala'' justo un poco antes de iniciar el movimiento en Urdhva Hastasana, y ``exhala'' justo antes del segundo movimiento hacia Uttasana. Observa tus propias tendencias a acortar la respiración.

Cuando te sientas cómodo dirigiendo tu propia respiración, visualiza  la secuencia de movimientos de Surya Namaskara en lugar de realizarlos, y dirige la respiración audiblemente. Cuando te encuentres cómodo con ese nivel de instrucción, camina alrededor del cuarto y dirige la respiración, manteniendo un paso tranquilo de movimiento sincronizando con tu propia inhalación y exhalación. Incorpora las enseñanzas de la respiración tan completamente de tal forma que si tu congelaras un instante en cualquier parte del saludo al sol, sabrías que parte de la respiración (inhalación o exhalación) se conecta.

\subsection{Nivel 2 - Movimiento del Cuerpo Exterior}
Lo siguiente es enseñar el movimiento del cuerpo exterior. La duración dentro de una postura es dependiente del estilo de clase que est\'es enseñando (como alentadora, restaurativa, meditativa, etc). Como punto inicial, cuenta tu número de respiraciones en un minuto, utilizando tus respiraciones para marcar la duración. Enseña posturas por 45 segundos por lado para una clase de paso medio, un minuto para un paso más lento.

Necesitarás escribir un guión para instruír la entrada a una postura. Comenzando tu entrada a la mayoría de las posturas de pie desde un desplante te proporciona un punto de referencia ``base''. Aquí está un ejemplo de instrucción meramente física a Parsvakonasana (postura de lado) entrando desde un desplante:

``Desde un desplante (realizado con el pi\'e derecho al frente, pi\'e izquierdo atrás) rotando tu pi\'e trasero 90 grados y presionando por completo las cuatro esquinas del pi\'e a la tierra. Coloca tu antebrazo derecho en tu muslo derecho y tu mano izquierda en la cadera. Rota tu torso hacia la izquierda.''

Esta es una instrucción muy básica, libre de palabras innecesarias. Esta instrucción tarda alrededor de quince segundos, permitiendo a los estudiantes realizar cada parte de la instrucción. Una vez que ellos han tomado la forma básica, la duración es cinco respiraciones o aproximadamente un minuto. La postura es repetira de nuevo hacia el otro lado con la misma duración. Escribe un guión básico para el cuerpo exterior para todas las posturas de pie que enseñarás y practica decirlas conforme realizas el asana, utilizando los movimientos de tu propio cuerpo como guía para la sincronización. Habla primero, despu\'es mu\'evete.

Cuando te sientes cómodo a este nivel, instruye tu propio cuerpo, intenta levantarte como si estuvieras dirigiendo una clase al frente del cuarto y dirigiendo oralmente una postura de pi\'e a la vez. Luego sincronízate, eliminando cualquier cosa poco clara o innecesaria.

\subsection{Nivel 3 - Alineamiento Fisico/Movimiento Energético}
Construyendo sobre la respiración y la forma básica de la postura, alineando el cuerpo óptimamente es lo siguiente. La alineación básica del cuerpo exterior (longitud de la postura, posición del cuerpo) debe ser tomada en cuenta en tu instrucción de movimiento del cuerpo exterior. Ahora puedes comenzar a describir el movimiento del prana, conectado a la inhalación y exhalación para alinear al estudiante con el pulso de la naturaleza. Conectando la inhalación a energía condensada, la exhalación a energía expansiva. Aquí está un ejemplo, basado en las previas instrucciones a Parsvakonasana:

``Desde un desplante (realizado con el pi\'e derecho al frente, pi\'e izquierdo atrás) rotando tu pi\'e trasero 90 grados y presionando por completo las cuatro esquinas del pi\'e a la tierra. Inhala conforme recoges tu energía de la tierra hacia tu centro. Coloca tu antebrazo derecho en tu muslo derecho y tu mano izquierda en la cadera. Rota tu torso hacia la izquierda. En tu siguiente exhalación, envía la energía hacia atrás a trav\'es de las piernas hacia la tierra''

Esta instrucción mejorada ahora toma alrededor de 25 segundos en ser expresada. A este punto, estás comenzando a introducir la intención filosófica. Al simplemente describir el flujo de la energía la atención del estudiante es recogida a esta pulsación universal de opuestos. Dependiendo del estudiante, esta incorporación física puede ser un momento ``Ah,Ha!'', o puede no resonar del todo. Continúa enseñando, Practica esto con todas las posturas de pie de nuevo, conforme las realizas, y luego mantente quieto.

\subsection{Nivel 4 - Incorporando Intención}
\subsection{Enseñando lo que Observas}
\subsection{Temporizando una Clase}
\subsection{Saludando/Centrando}
\subsection{Secuencia}
\subsection{Trabajo en Equipo}
\section{Tematizando}
\subsection{Tematizando a una Postura Especifíca}
\section{Organización del Salón de Clase}
\subsection{Líneas Visuales}
\subsection{Diseño de Clase}
\subsection{Ofreciendo Props}
\section{Demostración}
\subsection{Demostración Silenciosa}
\section{Cuestiones de Salud}
\subsection{Estudiantes Lesionados}
\subsection{Usando Props}
\subsection{Cuidados Específicos de Salud}
\subsection{Yoga para Artritis}
\subsection{Fibromalgia}
\section{Lenguaje}
\subsection{Tono de Voz}
\subsection{Que Tanto Decir?}
\subsection{Se Conciso}
\subsection{Volumen/Contenido}
\section{Modificación de Postura}
\section{Observación: Estudiante Individual}
\subsection{Fundación}
\subsection{Estado de Animo General}
\section{El Rol del Maestro}
\section{Fundamentos de Secuenciado}
\subsection{Secuenciado Variable y Establecido}
\subsection{Principios de Secuenciado}
\subsection{Entretenimiento - Centrando la Clase}
\subsection{Secuenciando la Clase a Nivel Mixto}
\section{Creando Intención}
\subsection{La Intención de "Liberar Tensión"}
\section{La Práctica y el Servicio de Enseñar Yoga}



%\chapter{Práctico}
%\section{Tarea}
%\subsection{Clases en Secuencia}
%\subsection{Desarrollar Intención para Clases}
%\subsection{Autoevaluación}

\chapter{Términos de Sanscrito}
\section{Glosario y Términos en Sanscrito}



\chapter{Un Entrenamiento Ejemplo}
\section{Un Entrenamiento Ejemplo - Yoga en Sillas}



\chapter{Ilustraciones}
\section{Flujo de Posturas}
\subsection{Surya Namaskara}
\subsection{Todos los Niveles de Practica de Asana}
\section{Posturas Syllabus}
\section{Ajustes Hands-on}



\begin{table}[h!]
	\centering
	\begin{tabular}{l|c||r}
		1 & 2 & 3\\
		\hline
		a & b & c\\
	\end{tabular}
	\caption{Caption for the table}
	\label{tab:table}
\end{table}


\begin{table}[h!]
	\centering
	\begin{tabular}{ccc}
		\toprule
		Some & actual & content\\
		\midrule
		prettifies & the & content\\
		as & well & as \\
		using & the & booktabs package\\
		\bottomrule
	\end{tabular}
	\caption{The other table}
	\label{tab:table2}
\end{table}


\newpage
áéíóú
\'a\'e\'i\'o\'u
ñ



\end{document}
