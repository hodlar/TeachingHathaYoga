\chapter{Anatomía y Psicología}
\section{Los Vayus}
\section{Comprensión y Tensión}
\section{Anatomía Funcional}
\subsection{Huesos y Articulaciones}
\subsection{La Espina}
\subsection{Muscúlos}
\subsection{Muscúlos y Posturas}
\subsection{Yoga y Posturas}
\newpage
\section{Los Bandhas}
Bandha significa \lq\lq candado\rq\rq. Este tipo de candado, en lugar de cerrar, como el tipo de candado donde se requiere una llave para abrir, fue de hecho un termino figurado. Estos candados son como una zanja usada para dirigir el agua a diferentes partes de un campo. Bandhas en el cuerpo son usados para dirigir la energía tanto física como energeticamente. Físicamente, los bandhas funcionan para mantener estimulados y armonizados nuestros órganos internos. Energeticamente, asisten al movimiento del prana, o energía en el cuerpo. Hay tres tipos de bandhas usados en la práctica de las asanas:

\subsection{Mulabandha}
Localizado entre el ano y los genitales, es el músculo perineo para los hombres. Para las mujeres está ubicado cerca del límite superior del cuello del útero. La activación del Mulabandha no es una fuerte contracción forzando los músculos que lo rodean, es mas sutil que eso. Mulabandha puede ser experimentado activando los músculos hacia atras, incrementando la curvatura lumbar en la espina, despus permitiendo al coxis alargarse, estimulando al abdomen a activarse y la base de la pelvis a levantarse.

Activar los muslos hacia atras acomoda las cabezas del femur hacia atras y crea una expansión en el área pélvica.
Bajar el sacro reafirma la carne de los glúteos. El abdomen bajo se mueve de la pubis al ombligo.

La sinergía creada por esos dos complementarias, y a la vez opuestas fuerzas, crean Mulabandha. En lugar de endurecer o relajar en el área pélvica, un levantamiento es creado similar a lo que sería el último medio centímetro de una malteada por un popote.

\subsection{Uddiyana Bandha}
Localizado ligeramente abajo del ombligo, Uddiyana Bandha significa \lq\lq Volar hacia arriba\rq\rq refiriendose a su efecto en el prana. Este segundo banda es activado en una forma parecida a Mulbandha, con un mínimo de endurecimiento externo o contracción. En el proceso de crear este candado, el centro del plexo solar es llevado adentro y hacia arriba en un levantamiento abdominal y es cuando se encuentra la activación. En expresión completa es llevado acabo exalando completamente y luego llevando el abdomen bajo hacia adentro y arriba, mientras se eleva el diafragma. Este nivel de Uddiyana Bandha será usdo en la práctica de la retención de exalación en Prnayama, pero debido a la incapacidad de inalar mientras se ejecuta este nivel, simplemente mantener tranquilidad cerca de tres dedos debajo del ombligo otorga espacio para que el diafragma baje durante cada exalación. Conforme el diafragma baja, la respiración es impulsada a moverse hacia las costillas, espalda y pecho. En cada exalación los músculos abdominales promueven a completar el vaciado de los plmones. El proceso toma práctica, y las sutilezas de la relación entre respiración y bandhas debe ser explorada experimentalmente.


\subsection{Jalandhara Bandha}
Este candado es creado levantando y girando los hombros  hacia atras para primero ampliar y luego levantar el pecho. Despues la parte trasera de la cabeza se extiende hacia el cielo y la barbilla se mueve en una contracción, la cuál es formada donde dos huesos de la clavícula se encuentran. El candado ocurre espontáneamente en algunas posturas como el parado de hombros, pero no es usado tan ampliamente como los otros dos candados.

\section{La Respiración}
\subsection{Respiración Ujjayi}
\section{Elementos de la Naturaleza}
\subsection{Cualidades Caracteristicas de los Cinco Elementos}
\subsection{Ayurveda}
\section{Los Cinco Koshas}


