\chapter{FILOSOFÍA, ESTILO DE VIDA Y ÉTICA}
texto
\section{Por qué el yoga pudo haber pasado}
\textbf{\textit{Imaginate a ti mismo hace 10,000 años\ldots}}
Te despiertas con el olor a tierra conforme el sol calienta y comienza a evaporar la humedad. Tu vivienda simple provee algo de refugio de los elementos. Tu tribu se levanta al amanecer; los niños juegan alrededor. Conoces a todos en tu tribu, y ellos te conocen. Más tarde llega la el resultado de la caza para ser cocinada y compartida por todos. Cuando el sol se pone, se hace un fuego y observas la madera convertirse en caloe, luz y ceniza. El humo se eleva en el cielo donde las estrellas son tan claras debido a la ausencia de luz ambiental que puedes reconocer las constelaciones como viejos amigos. Cuando es tiempo de dormir, el sueño llega fácil al cuerpo en armonía con el resto de la naturaleza. Cuando la tribu se mueve, volteas atrás y todo lo que queda detrás de ti es el cesped doblado de donde los refugios se encontraban, y un pozo de fuego. En unas cuantas semanas incluso esto se volverá invisible.

La idea de ``tener relaciones'' con otros en tu tribu no es un concepto que nadie comprenda. Conoces a estas personas de toda la vida. Algunos te agradan más que otros, pero no se comienzan y terminan relaciones con ellos. Ellos son literalmente tus relaciones, tanto como los animales y plantas. Los niños alrededor son responsabilidad de todos, y obtienen diferentes habilidades de otros en tu tribo conforme se mueven libremente alrededor.

Este es un regalo económico. La moneda no existe, y la respuesta natural a la abundancia de la vida en la tierra es la gratitud de uno. Nadie tiene la idea de que la vida humana deba ser prolongada, o que la juventud es mejor que la madurez. Los espíritus humanos y animales habitan la tierra.

El escenario anterior puede sonar utópico. Eso es mayoritariamente porque a trav\'es de proyecciones y algunos estudios poco cienfíficos de el siglo 19 en europa, se nos ha hecho creer que la vida de nuestros ancestros es ``brutal y corta''. Este no era el caso. Ni lo era acerca de que nuestros ancestros eran menos saludables que nosotros. Comiendo una dieta ``orgánica'', azúcar en cualquier forma dado que era difícil encontrar, y caminando kilómetros diariamente, nuestros predecesores eran generalmente atl\'eticos, saludables y competentes en diversas áreas.

La conexión profunda de nuestros ancestros nómadas con el mundo natural hace viible la continuidad de todas las cosas en su primer naturaleza. Justo cuando comenzamos a forzar la naturaleza para proveernos con más de lo que podemos consumir al instante, a trav\'es de la agricultura, removimos la idea del despíritu de la naturaleza, por un regalo nunca pedido, y una vez pedido, no es más un regalo.

Dejando atrás el estado de reconocimiento de la naturaleza sagrada de las cosas, la jerarquía espiritual evoluciona. Una vez removido de la tierra, el espíritu se movió a las montalas (dominio de los dioses antiguos) y entonces a los cielos. Abajo es mejor que aquí, abajo es peor. Subir es bueno, mientras que decender es malo. Este ``verticalismo'' tambi\'en disminuye nuestra conexión horizontal del uno con el otro. Cuando observa la historia de las religiones organizadas, la representación de dios es ``en lo más alto'' (en una plataforma, un trono o si está caminando, es muy alto).

Lo pantanoso, la tierra f\'ertil de la vida diaria no era sagrada, y junto con el modelo vertical del espíritu, llegó otro modelo, pureza: blanca, brillante, inmaculada. Nuestro instinto, la naturaleza crnal humana se volvió menos espiritual y finalmente pecado, y a su vez, la idea abstractca del cielo deseada. El propósito del hombre en la vida estaba ahora no solo restrinjida de la naturaleza a trav\'ez de cultivos de la tierra y domesticación de animales, sino tambi\'en restringido de su propia naturaleza. Esto es, convertirse en un cultivo.

La creación del mito de Adan y Eva arrojados del jardín del Eden había sido invertida. El jardín natural del mundo, cultivado por el hombre. Adan y eva fueron arrojados \textit{en} un jardín, donde tenían aue labrar la tierra con el sudor de su frente.

Los orígenes del yoga son algo misteriosos, con la tradición siendo largamente oral en naturaleza. Esculturas encontradas en las civilizaciones de valles de ríos Indús de Harrapa y Mohenjo-daro representando una figura sentada en lo que parece ser una postura de yoga data del 2500AC.

Siendo los orgígenes de la agricultura aproximadamente 5,000 años antes de esto. La práctica del yoga comienza \textit{despus\'es} de la revolución agrícola. La mayoría, mas no todos los cazadores se convirtieron en granjeros cultivando la tierra. Los granjeros deben protejer sus cultivos, construir cercas, guardar la comida restante, y ser capáz de intercambiar ese restante. Por lo tanto la agricultura arroja ideas como posesión, control, cambio, reglas y ley. Los granjeros comen una dieta más monótona de cultivos plantados y animales domsticos, opuestamente a la dieta variada que se encuentra en los cazadores y recolectores de temporada. Los efectos negativos en la salud y esperanza de vida en muchas culturas fue enorme. La práctica de yoga puede haberse acentuado como cura para este nuevo estulo de vida: una forma de reconectarse con las fuerzas naturales y ritmos que fueron olvidándose.

Las herramientas del yoga son las que ya conocemos: cuerpo, respiración y mente. Algunas de las perspectivas posteriores del yoga del cuerpo son muy distintas a la idea de la autodivisión del espíritu que se ha inculcado profundamente en nuestra cultura, actitud y comportamiento.

La palabra yoga puede significarse ``unión'' o una aplicación de significados; en este caso, significa la reconexión con algo. Recordemos que \textit{algo} ya está. No hay separación del espíritu de la naturaleza, excepto en nuestra mente. Cualquier búsqueda interior nos llevará de nuevo al sentimiento de ser. Las ideas que la mente genera y que pone en el cuerpo, al igual que cualquier cosa que se encuentre en la tierra, se encuentran conectadas a qu\'e tanto se toma de nuestra conexión con los elementos primarios de los que nos generamos.

\section{Un resumen histórico de la filosofía del yoga}
El origen específico del yoga sigue siendo un misterio. Existe evidencia que indica que el yoga pudo existir desde 2500-1500AC, en la región de la India ``Indus Valley''. Esculturas de figuras sentadas en lo que parece postura de loto de esta era han sido encontradas, pero debido a que la leyenda que acompaña estas figuras es desconocido, no es posible determinar con ninguna certeza si las esculturas son representaciones de una postura de yoga, o simplemente una forma de sentarse en el suelo. Lo que está claro es que desde tiempos antiguis ha existido el conocimiento de que la conciencia humana es vasta, puede ser explorada, y desde la exploración interior ha sido revelada sabiduría sobre la condición humana, el universo, y nuestro lugar en el.

En los primeros siglos de los primeros milenios AC, dos ramas de cultura existieron en India: v\'edicos y no v\'edicos. Los \textit{Vedas} contenían textos sagrados de sabiduría revelada, o Sruti, que significa ``lo que es escuchado de una fuente superior''. Los cuatro \textit{Vedas} componen las escrituras más antigual de la f\'e Hindu. La cultura india no v\'edica incluía Jainismo y Budismo, ninguno de los cuales aceptaban la autoridad de los \textit{Vedas}, y consecuentemente evolucionaron en creencias separadas. Es importante recordar que, dentro de la cultura india, el conocimiento era transmitido de Guru a estudiante: el Guru transformaba su sabiduría en algo consistente y apropiado para ese estudiante. Dado este medo de transmisión de conocimiento, diferentes escuelas de filosofía se relacionaban e influenciaban la una a la otra en una forma mucho menos rígida de lo que imaginamos.

Es poco claro si el yoga evolucionó de la cultura v\'edica o no v\'edica. Los acadm\'emicos han observado que durante el período Sramanas (literalmente, ``aquellos que se esfuerzan'') estuvieron involucrados en austeridades (actividades practicadas por individuos que eran renunciantes y ascetas de la cultura no v\'edica).

El primer milenio AC fue un período de cambio dramático social y cultural en India. Cerca del s\'eptimo siglo AC, largos centros urbanos fueron tomando forma al norte de la India. Los centros urbandos crecieron donde había abundancia de comida y forma de guardarla. No enteramente dependiente de agricultura, y otros bienes comnzaron a ser producidos, el comercio evolucionó junto con las rutas de comercio, y las idea tambi\'en eran bienes intercambiables. Durante este período de rápido cambio, la filosofía evolucionó tambi\'en. Posiblemente como resultado de epidemias esparci\'endose de pueblos aislados a centros urbanos mayores, que resultaron en muerte esparcida, los filósofos comenzaron a cuestionarse el mero significado de la vida y la naturaleza de la existencia. Cerca de el s\'eptimo siglo AC el Upanishada más viejo se escribió, y fu\'e conocido como ``Vedanta'', el fin o culminación de los Vedas. Upanishad literalmente significa ``sentarse cerca''; esto da una pista de cómo la sabiduría era transmitida de maestro a estudiante en mucha proximidad. El maestro o Guru puede practicar la t\'ecnica o recitar la información a un estudiante, y una vez terminado, tomar su cabeza, agitarla y pedir al estudiante que repita el ejercicio para asegurarse de que no lo olvide.

Dos creencias importantes que influenciaron el desarrollo del yoga creció durante este período de cambio y reflexión, notablemente Samsara (el eterno ciclo de vida, enfermedad, vej\'ez y muerte) y Karma (la creencia que todas las acciones tienen fruto). Si cada acción tiene fruto, y no puedes experimentar todos los frutos de las acciones en una vida, entonces renaces. Con este concepto evolucionado de la existencia como un ciclo de vida, muerte y renacimiento. La siguiente más grande pregunta realizada fue, ``hay algo más?'' Existe alguna forma de salir de este ciclo de renacimiento? Vida, en la noción de Samsara, es vista como una infinita y frecuentemente dolorosa experiencia, una vista del mundo fundamentalmente negativo de algo para trascender, para escapar.

En este momento en la historia, la cultura indua fue característicamente masculina. Aunque las referencias a personajes femeninos en los Upanishads no fueron raros, las tradiciones del yoga fueron dominadas en estos tiempos por austeridad, y un caracter marcial de dominio de mente y cuerpo. Parte de los grandes esfuerzos y sacrificios hechos por los yoguis en este período requerían alejarse del mundo, el mundo de distracción. Una de las preguntas persistentes eran si un yogui en una caverna hacía algo para beneficio del mundo. Para lo mismo los ascets preguntaban, ``por qu\'e vivir en un manicomio?'', por lo que podemos observar que la pregunta de si es posible integrar el yoga en nuestras vidas ha existido desde el inicio.

Cerca del quinto siglo AC, el período preclásico, tres ramas del yoga se desarrollaron: las tradiciones upanishadas, budismo y jainismo. El \textit{Bhagavad Gita} fue escrito poco despu\'es del quinto siglo, y fue probablemente completado al final del milenio. Con este texto sagrado, no hay nada corto en la revolución de la filosofía yoguica. Existe una ampliación en la práctica del yoga. Distintas de formas de práctica se describen: Karma Yoga, o el yoga de la ación; Bhakti Yoga, o el yoga de la devoción; y Jnana Yoga, o el estudio de la sabiduría. De esta forma, la práctica de yoga y los estados más alto de conciencia se hacen disponibles para todos y renuncuar al mundo e irse a una caverna ya no es necesario. Se menciona tambi\'en dentro del texto que las mujeres no pueden ser excluídas de esta práctica.

Al inicio de los primeros siglos de la Era Común, nace una síntesis de las filosofías indias. Este es el yoga clásico, o el yoga de Paranjali. Los \textit{Yoga Sutras} atribuídos a Patanjali son una organización de filosofía yógui en cortos aforismos o verbos. Patanjali es con frecuencia comparado con el sistema de Ashtanga Yoga, o las siete ramas del yoga, pero en lo que Patanjali se encuentra interesado principalmente no es una aproximación secuencial a la iluminación, ni un sistema de ramas asendiendo sutilmente. Patanjali está interesado en una cosa: Samadhi. El Samadhi es el estado meditativo más alto en el que una persona trasciende su ego individual y se une al universal. En los \textit{Yoga Sutras}, el da la definición de yoga en el segundo sutra, ``Yoga citta vrrti nirodhah'' o ``el yoga es cece de las fluctuaciones de la conciencia''. El describe las diversas formas de lograr este estado. En la perspectiva de Patanjali, hay solo dos cosas a considerar: el Ser, o el testigo consciente llamado Purusa, y todo lo demás que es persivido por ese testigo. Todo lo demás, pensamientos, emociones e incluso memoria, reside afuera del testigo consciente. Esto es llamado Prakriti, o naturaleza.

Patanjali nos dice que en algún punto, de alguna forma, nos olvidamos de nuestra naturaleza esencial. Nos identificamos con el mundo físico, que es llamado ``Pracriti''. Desarrollamos formas de pensamiento, apego a nuestras ideas, vemos si somos hombres o mujeres, grandes o pequeños, y de alguna forma esas cosas se vuelven nuestra identidad. La inhabilidad de ver la diferencia entre nuestra naturaleza esencial (Purusa) y todo lo demás (Prakriti) es llamado Avidya, o ignorancia. Cómo superamos esta ignorancia fundamental? Patanjali dice que la única forma de ver la diferencia entre nuestro testigo consciente y todo lo que la conciencia percive es crear tranquilidad. Como un lago tranquilo sin olas ni ondulaciones, en esa calma podemos observar nuestra naturaleza esencial, disimulada por los movimientos de la mente.

Desde los tiempos de los Yoga Sutras, hubo un período de gran interacción y creatividad de filosofía yóguica. Cerca del sexto siglo nació el Tantra yoga. En el octavo siglo un maestro llamado Sankara formuló una escuela no dual (Advaita) de la folosofía v\'edica. La visión del mundo del Sankara, igual seguía lejos de ser prometedor. Su creencia era que, aún cuando sólo hay una realidad, debido a nuestra ignorancia (Maya), nosotros nos imponemos limitaciones y separaciones en lo que vemos, y como un hombre caminando en la oscuridad y ve una cuerda enroscada piensa que es una seríente, nos encontramos engañados por nuestra inhabilidad para ver claramente. La única forma de ver claramente en la oscuridad es iluminandola, al igual que en la vista del Sankara una cosa puede ser curada con lo opuesto; oscuridad por luz, ignorancia por conocimiento, y no otra cosa. El mundo de la forma no está valuado por su filosofía, es visto como una ilusión.

La práctica del Tantra Yoga evolucionó a trav\'es de los siglos, y encontró una formación posterior en la escuela del Shivaismo de Cachemira alrededor del octavo siglo DC. El Tantra Yoga, evolucionando cuando lo hizo, tuvo el beneficio de siglos de desarrollo y por lo tanto fue capaz de voltear atrás y transmitir el conocimiento previo en una forma más sofisticada. El Shivaismo de Cachemira acepta que con la filosofía no dual del Sankara V\'edico pero plantea la cuestión, ``si sólo hay una realidad, entonces qu\'e es la cosa llamada ignorancia?'' La filosofía v\'edica no puede responder a esta pregunta debido a al ignorancia, a Sankara, no es algo en sí misma, sólo una ausencia de conocimiento. El Shivaismo de Cachemira responde que que si existe solo una realidad, tiene que respetar que cualquier cosa que suceda (o aparente suceder) a esa realidad tiene que ser una \textit{operación} de esa \textit{realidad misma}. La razón por la cuál vemos una diversidad de formas incluso cuando hay sólo una realidad es que esa realidad no ha creado una ilusión, sino un mundo físico vibrando en ser. Somos vistos como la condensación de ``la fuente'', conteniendo el poder completo de esta fuente. La práctica de yoga es entonces un recuerdo de este potencial. No tenemos que correr del mundo. El mundo es donde nuestro yoga se lleva a cabo.

La filosofía del Shivaismo de Cachemira dicta la necesidad de la gracia de un guru para otorgar el salto espiritual llamado ``Shaktipat''. Sin la transmisión de energía, el estudiante no puede alcanzar la iluminación. Este dilema es manejado por otra escuela de Tantra llamada ``Shri Vidya'' o sabiduría favorable. La forma más reciente (que conozco) de acercarse es actualmente siendo enseñada por Dr Douglas Brooks (2010). Douglas aprendió la forma de yoga llamada ``Rajanka'' de su maestro Gopala Aiyar Sundaramoorthy. Rajanaka puede ser traducido como ``Pequeño Príncipe'' o ``el que es soberano para sí mismo''. En este modelo horizontal de yoga, no hay búsqueda por un estado de iluminación sino una expansión continua de entendimiento y apreciación. Conforme los yoguis deciden involucrarse con los regalos y oportunidades que la vida presentan, nuestra sensibilidad e intimidad con nosotros mismos y los demás incrementa a trav\'es de compartir experiencias, regalos únicos e ideas.

\subsection{Los upanishads}
Los Upanishads son una colección de cerca de doscientos enseñanzas incluyendo historias, metáforas e instrucciones de meditación. Este arreglo no-homog\'eneo refleja la transmisión naturalmente oral de las enseñanzas. La gran variedad de información en ls Upanishads no refleja la filosofía yóguica unificada. Sin embargo, una consideración profunda referente a la divinidad innata de humanidad es claramente presente.

La raíz del mundo Upanishad ``Upa, Ni, Shad'' significa literalmente ``Cerca, bajo, sentado''. Esto implica tambi\'en que las enseñanzas pueden no ser inmediatamente obvias y requieren estudio y dedicación para ser asimiladas. Los Upanishads fueron enseñanzas impartidas por las clases más altas de la sociedad, la clase guerrera/governante y la clase eclesiasta. Esas enseñanzas no estaban disponibles para otros en la sociedad.

Aunque los Upanishads son parte del cuerpo V\'edico, contienen un mensaje distinto. Veda literalmente significa ``conocimiento'', así que los \textit{Vedas} eran libros de conocimiento. El enfoque de los primeros vedas es cómo vivir una buena vida y la explicación de la realización correcta de los rituales. Las primeras enseñanzas V\'edicas no muestran inter\'es claro en liberación espiritual o yoga. Los Upanishads, forman parte de las posteriores enseñanzas V\'edicas, hablan de los conceptos de transmigración y renacimiento.

El proceso de comprensión de las enseñanzas de los Upanishads es tambin dado por:
\begin{itemize}
	\item Escuchar
	\item Contemplar
	\item Meditar
\end{itemize}

Esta metodología de aprender no es casual. Para escuchar una enseñanza (realmente escuchar) necesitamos abrirnos a la enseñanza completamente, con una mente de principiante. Se dice en los Upanishads que debemos escuchar ``como un venado escucha la música''. Si puedes imaginar la sensivilidad del oído de un venado y la cualidad de alerta del animal, la naturaleza de este tipo de atención es similar.

Despu\'es de escuchar completamente, contemplamos la enseñanza. Debemos contemplar las enseñanzas ``como una vaca mastica el pasto''. Una vaca continuará mascando el mismo bocado de pasto, preparándolo para la digestión por un buen rato. Durante el proceso, comenzamos a ver las enseñanzas como algo importante para nosotros, la incorporarlos como sabiduría y no sólo información, verlos desde diferentes puntos de vista.

Finalmente, asimilamos las enseñanzas ``como un cisne puede separar la leche del agua en un lago''. Si se vierte leche en un lago, un cisne puede separar la leche para beberla, y dejar el agua. De esta forma tomamos la esencia de las cosas que son relevantes para nosotros.

La clave de las enseñanzas de los Upanishads es simple:
\begin{center}
	Atman = Brahman
	Or
	Ser individual = Ser Universal
\end{center}

En Sánscrito, la frase ``Tat tvam asi'' o ``Tu eres Eso'', lo dice todo. Tu \textit{eres} lo que buscas. La conciencia es siempre presente, no un estado mental o algo a ser alcanzado. Más bien, nuestro Ser simplemente necesita ser revelado por identificación, y luego hacer a un lado todo lo que no es la verdadera naturaleza. Como el artista Miguel Angel dijo, para hacer una escultura el solo tenía que ``remover la piedra que no era parte de la estatua''.

Los Upanishads tambi\'en transmiten las enseñanzas de este Ser Universal que puede de hecho tomar forma humana. Esta enseñanza permite respetar cualquier otra religión, dado que la Conciencia Suprema puede tomar cualquier forma, o ninguna forma. Esta es una forma inclusiva de ver las distintas prácticas espirituales que buscamos.

\subsection{El bagavad Gita}
Escrito aproximadamente 400AC, el \textit{Bhagavad Gita} cuenta la historia de Arjuna, un gran guerrero listo para la batalla entre dos ej\'ercitos guiados por primos en guerra, los Pandavas y los Kauravas. Arjuna pide a su conductor de carruaje, quien resulta ser Krishna (una manifestación de Dios) llevar su carruaje entre los dos ej\'ercitos para ver contra qui\'en va a pelear. El observa el ej\'ercito opuesto, los Kauravas, y ve en las filas a sus primos, amigos y maestros. Desanimado por la idea de matar a sus maestros y amigos, baja su arco y cae en el suelo de su carruaje desmayado, no puede encaminarse a la batalla.

En esas circunstancias que Krishna, quien hasta ahora ha sido amigo de Arjuna, se convierte en su maestro y explica por qu\'e Arjuna debe pelear. Arjuna es un guerrero hábil en batalla quien solo mata de una forma específica, dado que todos los seres son inmortales y tienen muchas encarnaciones antes y las tendrán despu\'es.

Krishna explica cómo la práctica de yoga puede librerar a Arjuna del karma de su acción. Menciona que los frutos de las acciones, cualquier acción, no está bajo nuestro control. Es sólo la acción misma que tiene la autoridad sobre uno mismo.

Krishna tambi\'en señala las diferentes formas de yoga: Karma Yoga, o el yoga de las acciones; Bhakti Yoga, o el yoga de la devoción; y Jnana Yoga, el yoga de la sabiduría. Krishna menciona que cualquiera de esas prácticas pueden llevanos a un completo entendimiento de quienes somos.

El \textit{Bhagavad Gita} es un texto revolucionario que abre las puertas a la práctica de yoga para todos, en cualquier clase de sociedad. Hasta donde sabemos, esto no ha pasado antes en la tradición del yoga. El yoga ha sido hasta ahora reservado para las clases sociales altas y excluído a las mujeres. Las tradiciones monásticas eran exclisuvas a su manera. No solo el yoga se hizo ahora inclusivo, sino que tambi\'en la necesidad de un intermediario entre dios y el hombre, como un sacerdote, fue removido. Krishna explica que incluso una humilde oferta de corazón como una hoja o un poco de agua puede ser ofrenda devocional.

El significado alegórico del \textit{Bhagavad Gita} es ilustrado al final del cuarto capítulo, cuando Krishna inspira a Arjuna a tomar acción, ``Mata con la espada de la sabiduría la duda nacida de la ignorancia que se esconde en el corazón. Se uno en armonía propia, en yoga, levántate, gran guerrero, levántate''.

\subsection{Los Yoga Sutras de Patanjali}
Cerca del segundo siglo DC, fueron escritos los \textit{Yoga Sutras} por Paranjali. Patanjali significa ``angel caído'', la idea de que el vino a ayudar a la humanidad. Los \textit{Yoga Sutras} son una colección de 196 versos cortos, o sutras, organizados en cuatro capítulos. En el primer capítulo, Patanjali da su definición de yoga casi de inmediato. Los siguientes tres capítulos enumeran prácticas para ayudar a los estudiantes de yoga que tienen dificultades siguiendo sus enseñanzas iniciales sin un guía, tambin describe algunas de las experiencias que el Yogui puede exrimentar.

Todos los maestros iluminados transmitiendo enseñanzas a trav\'es de libros o por otros medios, comienzan con la más alta enseñanza primero. Los sutras son paquetes increiblemente condensados de sabiduría, creados para ser desempacados y considerados, si es posible, con la ayuda de un maestro. Con frecuencia incluso el verbo se retira de el sutra para hacerlo más corto y fácil de recordar. Cuando Patanjali compuso los sutras, ellos no escribieron, ellos lo transmitieron oralmente. Cuando considieraron hacer esto, la organizacion de los sutras tuvo más sentido. Una repetición no lineal y la repetición de conceptos es parte de la forma en que hablamos, no necesariamente en la que escribimos, y esto es reflejado en los sutras. Ricos con sabiduría, pueden ser interpretados en la luz de nuestra cultura presente y circunstancias. Revisemos el primer sutra:

\textbf{\textit{``Atha yoganusaranam''} - Ahora comienza la práctica de yoga}
La palabra ``Ahora'' suele ser vista en escritugas yóguicas como una invocación, un comienzo. La palabra ``Ahora'', es utilizada al inicio de esos sutras implicando que cualquier cosa que hayamos hecho antes a este estudio, ahora comenzamos  con seriedad el estudio de yoga. El estudio no debe posponerse mas, por ahora el tiempo comienza. El tiempo de entender nuestra condición humana es ahora.

``Nosotros'', Patanjali nos informa que este estudio no está pensado para realizarse en aislamiento. El apoyo de una cominidad de personas con un objetivo común es indispensable en la creación de un momentum para estudio e intercambio de ideas. Cualquier comunidad con la capacidad de intercambiar ideas libremente evoluciona esas ideas mas rápidamente que cuando hay impedimento de distancia, o alguna egoc\'entrica horda de información.

``Yoga'', El t\'ermino \textit{yoga} ha significado diferentes cosas en diferentes tiempos. Para Patanjali, Yoga significa \textit{Samadhi}. Samadhi es el estado en que la conciencia auprema fluye libremente a trav\'es de nosotros. El ego individual se disuelve en este río de claridad.

\textbf{\textit{``Yoga cittavritti nirodhah''} - Yoga es el cese de las fluctuaciones de la conciencia.}
En este sutra PAtanjali nos da su definición de \textit{yoga}, ``Yoga es el cese de las fluctuacions de la conciencia''. Este es el más famoso de todos los sutras. Patanjali nos da su definición justo aquí, inmediatamente despus de su bienvenida. Yoga ocurre cuando los movimientos (vrttis) de la mente (citta) están tranquilos (nirodhah). En esta calma, en la ausencia de distración y preocupación mental, nuestra verdadera naturaleza puede ser alcanzada. Un ``vrtti'' es cualquier cosa que que gire o se mueva.

El movimiento de la mente incluye pensamiento, emoción y memoria. Cualquier cosa que es un disturbio a la quietud de la mente requiere restricción.

Se puede ver por qu\'e Patanjali tiene otros 194 sutras que ayudan a explicar cómo lograr esto. Calmar la mente no es una cosa fácil de hacer. En el siguiente sutra Patanjali explica por qu\'e podríamos querer emprender esta difícil tarea.

\textbf{\textit{``Tada drasruh svarupe vasthanam''} - Entonces el observador habita en su  propio esplendor.}
Entonces, cuando los movimientos de la mente están en calma, vivimos en ese lugar de Suprema Conciencia. Esa es la esencia de las enseñanzas de Patanjali. El resto de los sutras nos ayudan a entender cómo este estado del ser puede ser alcanzado.

\subsection{Vedanta - La filosofía No-Dual del Sankara}
\textbf{\textit{Historia}}
Entre el segundo y sexto siglo DC, los \textit{Yoga Sutras} de Patanjali al igual que otras filosofías incluyendo el Jainismo u Budismo, evolucionaron en un período intelectual muy f\'ertil, potenciado por el patrocinio de escuelas de aldo aprendizaje. Tan lejos como el segundo siglo AC, un texto fue escrito por el maestro Badarayana llamado el ``Brahma Sutra''. Dentro de este texto Badarayana intenta sistematizar los Upanishads. La sistematización de un texto tan variado como los Upanishads es una tarea muy complicada, especialmente desde que algunas enseñanzas de los Upanishads sugirieren una naturaleza dual del universo, y algunas sugieren unión total, o no dualidad. Como sea, distintas interpretaciones de los sutras de Brahma Badarayana abrieron incluso a discusiones más creativas entre los acad\'emicos respecto a la naturaleza del universo.

\textbf{\textit{La vida de los Sankara}}
Los Sankara vivieron solo por 32 años, entre 788 y 820 DC. En su corta vida, los Sankara sistematizaron las enseñanzas previas de los Upanishads y los \textit{Vedas}, crearon cuatro Mathas, centrando el conocimiento, en las cuatro esquinas de la India, haciendo cada uno un custodio de uno de los cuatro \textit{Vedas}, y asignaron a cada Matha uno de sus principales cuatro discípulos, asegurando la continuidad de sus enseñanzas. Esos Mathas continúan actualmente. Sankara era un escritor extremadamente prolífico, ofreciendo comentarios en los principales textos yóguicos al igual que su propia articulación de la naturaleza del universo.

Sankara es visto como una manifestación de Shiva, un aspecto de lo Supremos. Como dice la leyenda, los padres de Sankara no tenían niños y rezaron por descendencia. Se les dio una opción por los dioses: podrían tener tantos hijos como quisieran y serían irelevantes y tontos, o podrían tener un hijo que sería brillante, pero viviría una corta vida. Ellos opraton por la segunda opción, y Sankara nació. El padre de Sankara murió joven, por lo que Sankara fue criado por su madre. Pronto, el expresó su deseo de convertirse en un renunciante. Siendo el único hijo, su madre no lo permitiría, aqu\'el que renuncia debe dejar su familia por siempre.

Un día mientras rezaba cerca del río, un gran cocodrilo emergió del agua y tomó la pierna de Sankara. Sankara gritó a su madre ``D\'ejame renunciar AHORA!'', se entiende que si uno renuncia al momento de la muerte, alcanzará la liberación. Su atemorizada madre aceptó. Tan pronto como ella habló, el cocodrilo liberó la pierna de Sankara y regresó al agua. Sankara pudo tomar su camino, con algo de ayuda de los dioses.

\textbf{\textit{Advaita Vedanta}}
La filosofía de Sankara de la naturaleza del universo fue fundamentada por los textos base de la tradición del yoga, el \textit{Bhagavad Gita}, los Upanishads, e incluso por los Brahma Sutras. Pero refinó su filosofía tomando una idea común del Budismo, maya. Maya significa ilusión. En la filosofía de Sankara, existe solo una realidad. Pero debido a maya, nuestra percepción es alterada. Experimentamos el mundo desde una perspectiva alterada, y por lo tanto no vemos la realidad con claridad. Nosotros proyectamos ilusiones, como un hombre viendo un trapo en la osucridad y pensando que es una serpiente, y somos engañados por un velo de ignorancia. Si pudieramos ver con claridad, veríamos que solo hay una realidad estática que reside sobre este mundo de fenómenos.

En esta articulación de realidad, Sankara se puede referir a las enseñanzas previas de los Upanishads para apoyar su filosofía. Los Upanishads, una coleción de sabiduría yóguica recopilada por una variedad de maestros, algunas veces refiriendose a una naturaleza dual y otras veces no dual de las cosas. En la perspectiva de Sankara, las enseñanzas no duales hablaban sobre la perspectiva de lo Supremo y aparentemente las enseñanzas dualistas se dieron desde una perspectiva más baja de la vida diaria, por lo que hay solo una realidad, y dos formas de persivirla, una forma muy inteligente de armonizar las enseñanzas de los Upanishads. Existe, sin embargo, un problema con la filosofía de Sankara: maya. En la perspectiva de Advaita Vedanta, maya no es real, ni irreal, ya que no existe en sí misma, pero los efectos pueden sentirse. No es \textit{algo}, sino la \tetit{ausencia} de algo, es la ausencia de conocimiento. Sankara menciona que existe una realidad ultimadamente, entonces, donde deja eso a maya? Es parte de la realidad o una ilusión? Sankara dice que simplemente no se puede explicar, una respuesta algo insatisfactoria. La idea de maya y sus retos inherentes a una filosofía unificada serían explorados unos cuandos siglos despu\'es en las enseñanzas de Tantra Yoga, en la región de Kashmir.

\textbf{\textit{Tantra Yoga}}
Las raíces de la práctica de Tantra pueden ser rastreadas hasta el período prehistórico, donde las representaciones de adoración a la mujer, o el aspecto femenino de lo supremo, puede ser encontrado. En algún lugar cerca del quinto siglo AC, la práctica de adorar a esta energía femenina se volvió más sistematizada. Esos rituales y prácticas, y una filosofía articulada creada para traer transformación a la conciencia, es lo conocido como Tantra Yoga. Los primeros practicantes Tántricos fueron muy distintos en su aproximación a la espiritualidad, usaban esqueletos humanos como cuenco de plegarias o meditaban en suelo ceremonial fueron como prácticas comunes del Tantra. Comer carne o pescado, realizar rituales sexuales, usar vino o grano seco eran otras formas de intentar transformar la conciencia al ilustrar que el Supremo puede ser encontrado en todas partes, incluso en los lugares donde otros ven impureza. Despu\'es, muchas de esas prácticas se volvieron transformadas en ritual, como comer una oblea y tomar un sorbo de vino en un cáliz católico simbolizando comer el cuerpo y beber la sangre de Cristo. El objetivo del tantra era usar \textit{todo} alrededor nuestro para incrementar la conciencia, en lugar de negar la forma del mundo.

Tantra Yoga, como todas las tradiciones, evolucionó con el tiempo. El tantra incorpora muchas de las enseñanzas de los Upanishads, los Vedas, e incluso el yoga clásico de Patanjali. Una diferencia de las enseñanzas tántricas y algunas otras escuelas de yoga es que el Tantra siempre afirma nuestra realidad. el Tantra dice que este mundo es real. Todo en el universo es una manifestación de lo divino, y todo tiene una cualidad sagrada innata. El Tantra no niega un valor relativo, bien, mal, apropiado o inapropiado, pero siempre busca la divinidad última en todas las cosas. De esta forma, la filosofía es radicalmente distinta de las enseñansas de Advaita Vedanta de Sankara.

\textbf{\textit{Shivaísmo de Cachemira}}
El Tantra Yoga encontró una articulación más refinada en la región de Cachemira, India, entre el noveno y doceavo siglo AC. Podría no ser una conincidencia que esta sublime filosofía de vida se levantara en uno de los más lindos y encantadores lugares del mundo. Aunque había distintos maestros y escuelas de pensamiento informando una a otra relevantes principios y prácticas de Tantra (y es importante notar que el t\'ermino ``Tantra'' fue aplicado hasta despu\'es por acad\'emicos occidentales para su desarrollo en la folosofía yóguica), el mayor de estos maestros fue Abhinavagupta, quien vivió entre 975-1025 AC. Extremadamente prolífico en sus escrituras, su más importante obra fue el \textit{Tantraloka}, literalmente significa ``Luz en Tantra''. Este tratado sistematiza y resume la base de los textos del Tantra.

\subsection{Kashmir Saivism y los Tattvas}
\subsection{Yoga Moderno}
\subsection{Cronología del Yoga}
\section{Dieta y Estilo de Vida}
Es difícilmente necesario meditar en una caverna remota para crear un estilo de vida que ayude la práctica. Las prácticas de yoga presentadas en este manual son pensadas para ser integrales y adaptativas al mundo moderno. Entre el 1000 y 1400 DC, en Índia, la región de Kashmir vió una revolución en el penzamiento yóguico. Esos practicantes de yoga fueron, en realidad, amas de casa, esposos y esposas que tomaron la práctica de yoga y la convirtieron en la fábrica de sus vidas. Esta aproximación fue una no renuncia, entre otras prácticas, y fue después llamada tantra, que literalmente significa "tejido". Los yoguis tomaron el entendimiento moral más clásico de yoga y lo recrearon, redefiniéndolo en una práctica más significativa para ellos mismos.

"Una mente aguda, un corazón suave y un cuerpo vibrante" (John Friend) son entre otras cualidades de un maestro de yoga competente. Estas cualidades son apoyadas y fomentadas por:
\begin{itemize}
	\item descanso adecuado;
	\item relaciones alentadoras;
	\item dieta apropiada; y
	\item práctica personal y estudio.
\end{itemize}

Enseñanr una clase mientras te encuentras cansado, hambriendo, agitado o desconectado de tu práctica no es una experiencia de crecimiento. Si estás apropiadamente preparado, nutrido con comida, lleno de prana después de una buena práctica de yoga, bien descansado y sintiéndote apoyado, con seguridad darás una buena clase. El negocio del yoga y tu estilo de vida no pueden ser separados tal como en alguna otra ocupación. Como maestro, eres un ejemplo viviente de cómo la práctica de yoga funciona.

\subsection{Dieta}
La ecuanimidad de la mente ha sido siempre referida como una de las claves del yoga. "Yoga no es para quienes comen mucho, o no comen" (Upanishads). La dieta afecta directamente a nuestro ánimo y a nuestros cuerpos. Simplemente prestar atención a lo que comemos y a cuánto comemos es otra práctica de yoga, embebido en una práctica más larga. Extrema austeridad respecto a la dieta puede ser una forma que el ego toma para crear más separación. Como sea, comiendo una dieta balanceada con muchos vegetales y frutas es accesible para la casi todo. Disminuir los estimulantes como alcohol, azúcar y posiblemente cafeína creará un ánimo más estable y una atmósfera corporal donde la consciencia plena es más fácil de alcanzar. Comer una dieta vegetariana es considerado prificador por muchas ramas de la tradición del yoga. Comer la carne de un animal que ha sido mantenido en condiciones de vida no dignas será al finalde menor valor nutrimental que animales con libre albedrío. Nosotros, como seres sensibles, podemos elegir qué es útil para nosotros y cómo nuestras elecciones afectan a nuestros seres. Come con amigos o amados tanto como sea posible, dando bendiciones antes de comer.

\subsection{Sueño}
Hay una pulsasión universal llamada "Spanda". Desde esta pulsación viene la manifestación dual de la naturalesa; frío y calor, macho y hembra, luz y oscuridad. Una lugar fresco, oscuro y tranquilo y una cama cómoda para dormir recupera la energía vital del cuerpo y es el complemento natural a la actividad física. El ciclo del día es un modelo de vida, justo como la práctica del yoga y la postura de descanso Savasana es el modelo del día. Vive pleno y duerme profundo. Permite al día previo disolverse, brindando tu atención tan completamente como sea posible al momento presente.

\subsection{Práctica}
Enseñar yoga requiere la asimilación e incorporación de las enseñanzas, presentadas en tu propia voz y en tu propio modo. Conforme crecemos como maestros, agregamos nuestra propia voz al linaje de maestros. Es así como la tradición del yoga se mantiene viva y evoluciona. Al practicar por tu cuenta, tu sabiduría interna comienza a emerger. Al aceptar el conocimiento que ha sido revelado y al honrar a tus maestros, creces individualmente como maestro, y simultáneamente te conectas más con la energía universal. Elije un tiempo y lugar para la práctica diaria. El cuerpo cambia día a día. Algunos días nuestra energía está alta y el cuerpo se siente como una pluma. Otros días la energía corporal está menos disponible y se sentimos como concreto húmedo. Otros días de energía alta, tu práctica puede ser una celebración que es dinámica y retadora. En los días difíciles, desarrolla una secuencia de posturas más restaurativas para ayudar a aliviar la fatiga. Aún así no existe razón para dejar de practicar, y tus habilidades de enseñar a una variedad de estudiantes incrementará.

\subsection{Relaciones}
Todo en el universo es inatamente divino. La práctica de yoga no es algo que pase solamente en el tapete. Permanecer completamente consciente en relación con otros seres humanos es tal vez uno de los aspectos mas difíciles de la práctica. Es aquí el donde nuestra habilidad de recordar nuestra divinidad innata, y la divinidad de otros es revelada. Percibir patrones en tus relaciones. Percibir tus reacciones y deseos habituales. Abrazar las relaciones como parte de, no separarlas de tu práctica. De esta forma observarás la naturaleza sagrada de otros. Honra a aquellos que amas y aquellos que has amado.

\section{El principio de atracción}
\subsection{Yoga y el principio de atracción}
\subsection{Reciviendo}
\subsection{Práctica}
\section{Ética}
\subsection{Yamas y Niyamas}
\subsection{Mas de Ética de un maestro}
\section{Luz y Oscuridad}
\section{Mantras}
Un mantra es una invocación de sonidos sagrados, y como tal es otra forma de vibración en la forma de sonidos organizados. Repetir un mantra es una forma de yoga en sí mismo y es la práctica principal en Mantra Yoga. El efecto de la vibración importa. Incluso un instrumento musical, si está hecho de material orgánico como madera, absorverá las vibraciones resonando a travé de el. Si el instrumento se mantiene en sintonía y se toca regularmente, el tono del instrumento se profundiza, se vuelve más bello y permite acarrear la canción del música completa y acertadamente. El cuerpo humano en el caso de la repetición del mantra, es el instrumento, que toca la canción de la divinidad.

El mantra Gayatri es primero recordado en el \textit{Rig Veda} escrito en Sánscrito hace cerca de 2,500 a 3,500 años, y de acuerdo a algunas fuentas, puede haber sido cantado generaciones antes de eso.

\begin{multicols}{2}
	El Gayatri Mantra:\\*
	Om bhûr bhuva sva\\*
	tát savitúr várenyam\\*
	bhárgo devásya dhîmahi\\*
	dhíyo yó na prachodáyât
	\columnbreak
	\textit{Phonetic Pronunciation}
	Om burr buva-ha sva-ha\\*
	Tat sa-vi-tour vara-en ya-hum\\*
	Bar-go de vas-ya de my-hee\\*
	De yo-yo na pra-show-da-yat
\end{multicols}

\section{Visión general de los Estilos de Yoga}
Una variedad de caminos del yoga continúan entrelazándose e informándose entre sí. Algunos ejemplos son:
\begin{itemize}
	\item \textbf{Anusara Yoga:} Desarrollado por John Friend en 1997; unirica la filosofía enfocada al corazón Tántrica con principios de alineación bio-mecánicos.
	\item \textbf{Ashtanga Vinyasa Yoga:} Desarrollado por T.Krishnamacharya y su estudiante Pattabhi Jois; acercamiento sistemático y secuencial a la práctica de los asanas donde las posturas son separadas en series. Vinyasa, una conexión energetica de un asana a otro, es usado para crear y mantener calor y un movimiento de estado meditativo.
	\item \textbf{Ashtanga Yoga:} Llamado así por Baba Hari Dass, despus del camino de ocho pasos de Patanjali, no confundirlo con Ashtanga Vinyasa Yoga.
	\item \textbf{Bikram Yoga:} Una secuencia de veintiseis posturas en un cuarto calentado a 100 grados Fahrenheit.
	\item \textbf{Iyengar Yoga:} B.K.S. Iyengar, otro de los estudiantes de Krishnamacharya, refinó el aprendizaje de su gurú despus de mudarse a Pune. Avandonó el estilo Vinyasa y se enfocó en las enseñanzas de la salud, alineación estructural y beneficios terapeuticos de las posturas.
	\item \textbf{Kundalini Yoga:} Despertar energía, Kundalini yoga llegó al oeste en 1969, cuando Sikh Yogi Bhajan desafió la tradición y comenzó a enzeñarlo públicamento. Esta práctica es designada para despertar la energía Kundalini, la cuál es almacenada en la base de la espina y en ocasiones representada como una serpiente enroscada. Kundalini mezcla cantos, prácticas de respiración y ejercicios de yoga. El enfasis no es en los asanas, sino en los cantos y respiraciones.
	\item \textbf{Mysore Style:} Nombrado así en honor a la ciudad en india donde Pattabhi Jois enseña el metodo Ashtanga Vinyasa; una práctica autoencaminada con supervición y ajustes físicos de un instructor.
	\item \textbf{Vijnana Yoga:} Una práctica de Hatha Yoga desarrollada por Donna Holleman y Orit Sen-Gupta, basada en siete \"principios vitales\" diseñados para usar el cuerpo para explorar los más profundos niveles de nuestro ser.
	\item \textbf{Viniyoga:} Esta forma gentil de flow yoga pone gran \'enfasis en la respicación y cordina respiración con movimiento. El movimiento fluido de Viniyoga o Vinyasa es similar a la dinámica de la serie de poses de Ashtanga, pero ejecutado con una gran reducción de paz y nivel de estr\'es. Las posturas y secuencias son elejidas para encajar en las habilidades del estudiante. Se enseña al estudiante cómo aplicar las herramientas de yoga: asana, cántos, pranayama (control de la respiración), y meditación, en una práctica individual. Desarrollado por T.K.V. Desikachar, el hijo de Krisnamacharya (maestro de algunos grandes instructores de yoga incluyendo Iyengar y Pattabhi Jois), Viniyoga pone menos estres en uniones y rodillas manteniendo las posturas con una ligera flexión en las rodillas. Viniyoga es considerado excelente para principiantes, y es incrementalmente usado en ambientes rerap\'euticos.
	\item \textbf{Yin Yoga:} Un t\'ermino creado por Paul Grilley para describir una forma de práctica con un \'enfasis en posturas mantenidas por mucho tiempo, usualmente sentados, boca abajo o boca arriba. Yin Yoga se enfoca en fortalecer y alargar el tejido conectivo, que en turno, a partir de líneas meridianas, tiene un efecto óptimo en el funcionamiento de los órganos.
\end{itemize}

\subsection{Viendo la Imagen Completa}

\section{El negocio del Yoga}
\subsection{Marketing}
\subsection{Yoga en casa}
\section{Sencillez Voluntaria}
\subsection{Principios de Sencillez}
\subsection{Acercamiento a Sencillez}
\subsection{Un Acercamiento al Lado Financiero}
