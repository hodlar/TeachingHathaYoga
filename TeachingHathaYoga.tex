\documentclass[a4paper]{article}



\usepackage[T1]{fontenc}
\usepackage[utf8]{inputenc}
\usepackage{booktabs}

\title{Enseñando Hatha Yoga}
\date{2016-01-27}
\author{Daniel Clement with Naomi Clement}

\begin{document}
\pagenumbering{gobble}
\maketitle

\newpage
Copyright c 2007 Daniel Clement
Todos los derechos reservados. Sin limitaciones de derechos bajo derechos de autor, ninguna parte de esta publicación debe ser reproducida, almacenada o introducida en un sistema de
retribución, o transmitida, en cualquier forma o por cualquier medio (electrónico, mecánico, fotocopiado, grabado o de otra forma), sin el previo consentimiento escrito del propietario de los derechos, excepto para cortas reseñas.
Primer impresión Octubre 2007, segunda impresión 2008, tercer impresión 2009, cuarta impresión 2010, quinta impresión 2011.
Contacta al publicador en la web www.opensourceyoga.ca

ISBN: 978-0-9735820-9-3

\newpage
\tableofcontents

\newpage
"Cuando estás inspirado por algún gran propósito, algun proyecto extraordinario, todos tus pensamientos rompen sus ataduras: Tu mente traciende limitaciones, tu conciencia expande en todas direcciones y tu te encuentras en un nuevo, grande y maravilloso mundo. Fuercas dormidas, facultades y talentos cobran vida, y tu descubres que eres una persona mucho mejor de lo que jamás soñaste que serías."

-Patanjali, los Yoga Sutras

\newpage
Acerca de poseer el Yoga

Los materiales presentados en este manual representan una composición e interpretación personal del yoga. Yoga es tanto una ciencia como un arte de profundo entendimiento de la condición humana. En mi investigación en este tema, he tenido la buena fortuna de conocer grandes maestros, cuya sabidiría ha derramado luz en mi propia búsqueda por un entendimiento mayor. Tal acercamiento, con el tiempo, beneficiará tanto a estudiantes como a maestros. Cualquier estilo que sea tu práctica, yoga es una práctica de revelación. Lo que es revelado es nuestra verdadera naturaleza como un aspecto de Fuente.

Esta fuenta, como el océano, subyace en todas nuestras cualidades individuales. Una de las cualidades del ego individual es la noción de pertenencia. Pertenencia es manifestada en la forma de: escrituras de tierras que han estado aquí milenios antes de que pusieramos una cerca alrededor de ellos; el deseo de acumular bienes en nuestro nombre; o el patentar ideas. Ultimamente, dejamos todo atrás excepto el entendimiento que cultivamos, y es nuestro privilegio como maestros compartir ese conocimiento.

El yoga puede no ser poseido, porque el yoga como forma de conciencia adquirida, es una parte intrínseca de nuestra naturaleza. Cada vez que tomas una profunda y conciente respiración, has experimentado el yoga.

Namaste,

Dan Clement

\newpage
\section{FILOSOFÍA, ESTILO DE VIDA Y ÉTICA}
texto
\subsection{Por qué el yoga pudo haber pasado}
\subsubsection{Los upanishads}
\subsubsection{El bagavad Gita}
\subsubsection{Los Yoga Sutras de Patanjali}
\subsubsection{Vedanta - La filosofía No-Dual del Sankara}
\subsubsection{Kashmir Saivism y los Tattvas}
\subsubsection{Yoga Moderno}
\subsubsection{Cronología del Yoga}
\subsection{Dieta y Estilo de Vida}
\subsubsection{Sueño}
\subsubsection{Práctica}
\subsubsection{Relaciones}
\subsection{El principio de atracción}
\subsubsection{Yoga y el principio de atracción}
\subsubsection{Reciviendo}
\subsubsection{Práctica}
\subsection{Ética}
\subsubsection{Yamas y Niyamas}
\subsubsection{Mas de Ética de un maestro}
\subsection{Luz y Oscuridad}
\subsection{Mantras}
\subsection{Resumen de los Estilos de Yoga}
\subsubsection{Viendo la Imagen Completa}
\subsection{El Negocio del Yoga}
\subsubsection{Mercado}
\subsubsection{Yoga en Casa}
\subsection{Sencillez Voluntaria}
\subsubsection{Principios de Sencillez}
\subsubsection{Acercamiento a Sencillez}
\subsubsection{Un Acercamiento al Lado Financiero}

\section{Técnicas de Entrenamiento y Práctica}
\subsection{Técnicas de Asana}
\subsubsection{Alineamientos Fundamentales}
\subsubsection{Biomecánicas Holísticas}
\subsubsection{Terapia Estructural}
\subsubsection{Técnicas de Asana: Categorías de Posturas}
\subsubsection{Forma y Acción}
\subsubsection{Navegando el Tapete}
\subsubsection{Sacro}
\subsubsection{La Práctica de Asana}
\subsubsection{Yin y Yang}
\subsubsection{Polaridades de la Energía Física}
\subsubsection{Fuerzas Opuestas}
\subsection{Téecnicas de Purificación}
\subsection{Meditación}
\subsection{Pranayama}
\subsubsection{Nadi Shodhana}

\section{Anatomía y Psicología}
\subsection{Los Vayus}
\subsection{Comprensión y Tensión}
\subsection{Anatomía Funcional}
\subsubsection{Huesos y Articulaciones}
\subsubsection{La Espina}
\subsubsection{Muscúlos}
\subsubsection{Muscúlos y Posturas}
\subsubsection{Yoga y Posturas}
\newpage
\subsection{Los Bandhas}
Bandha significa \lq\lq candado\rq\rq. Este tipo de candado, en lugar de cerrar, como el tipo de candado donde se requiere una llave para abrir, fue de hecho un termino figurado. Estos candados son como una zanja usada para dirigir el agua a diferentes partes de un campo. Bandhas en el cuerpo son usados para dirigir la energía tanto física como energeticamente. Físicamente, los bandhas funcionan para mantener estimulados y armonizados nuestros órganos internos. Energeticamente, asisten al movimiento del prana, o energía en el cuerpo. Hay tres tipos de bandhas usados en la práctica de las asanas:

\subsubsection{Mulabandha}
Localizado entre el ano y los genitales, es el músculo perineo para los hombres. Para las mujeres está ubicado cerca del límite superior del cuello del útero. La activación del Mulabandha no es una fuerte contracción forzando los músculos que lo rodean, es mas sutil que eso. Mulabandha puede ser experimentado activando los músculos hacia atras, incrementando la curvatura lumbar en la espina, despus permitiendo al coxis alargarse, estimulando al abdomen a activarse y la base de la pelvis a levantarse.

Activar los muslos hacia atras acomoda las cabezas del femur hacia atras y crea una expansión en el área pélvica.
Bajar el sacro reafirma la carne de los glúteos. El abdomen bajo se mueve de la pubis al ombligo.

La sinergía creada por esos dos complementarias, y a la vez opuestas fuerzas, crean Mulabandha. En lugar de endurecer o relajar en el área pélvica, un levantamiento es creado similar a lo que sería el último medio centímetro de una malteada por un popote.

\subsubsection{Uddiyana Bandha}
Localizado ligeramente abajo del ombligo, Uddiyana Bandha significa \lq\lq Volar hacia arriba\rq\rq refiriendose a su efecto en el prana. Este segundo banda es activado en una forma parecida a Mulbandha, con un mínimo de endurecimiento externo o contracción. En el proceso de crear este candado, el centro del plexo solar es llevado adentro y hacia arriba en un levantamiento abdominal y es cuando se encuentra la activación. En expresión completa es llevado acabo exalando completamente y luego llevando el abdomen bajo hacia adentro y arriba, mientras se eleva el diafragma. Este nivel de Uddiyana Bandha será usdo en la práctica de la retención de exalación en Prnayama, pero debido a la incapacidad de inalar mientras se ejecuta este nivel, simplemente mantener tranquilidad cerca de tres dedos debajo del ombligo otorga espacio para que el diafragma baje durante cada exalación. Conforme el diafragma baja, la respiración es impulsada a moverse hacia las costillas, espalda y pecho. En cada exalación los músculos abdominales promueven a completar el vaciado de los plmones. El proceso toma práctica, y las sutilezas de la relación entre respiración y bandhas debe ser explorada experimentalmente.


\subsubsection{Jalandhara Bandha}
Este candado es creado levantando y girando los hombros  hacia atras para primero ampliar y luego levantar el pecho. Despues la parte trasera de la cabeza se extiende hacia el cielo y la barbilla se mueve en una contracción, la cuál es formada donde dos huesos de la clavícula se encuentran. El candado ocurre espontáneamente en algunas posturas como el parado de hombros, pero no es usado tan ampliamente como los otros dos candados.

\subsection{La Respiración}
\subsubsection{Respiración Ujjayi}
\subsection{Elementos de la Naturaleza}
\subsubsection{Cualidades Caracteristicas de los Cinco Elementos}
\subsubsection{Ayurveda}
\subsection{Los Cinco Koshas}

\section{Metodología de Enseñanza}
\subsection{Enseñando a dirigir}
\subsection{Agregando Contenido}
\subsubsection{Nivel 1 - Respiración}
\subsubsection{Nivel 2 - Movimiento del Cuerpo Exterior}
\subsubsection{Nivel 3 - Alineamiento Fisico/Movimiento Energético}
\subsubsection{Nivel 4 - Incorporando Intención}
\subsubsection{Enseñando lo que Observas}
\subsubsection{Temporizando una Clase}
\subsubsection{Saludando/Centrando}
\subsubsection{Secuencia}
\subsubsection{Trabajo en Equipo}
\subsection{Tematizando}
\subsubsection{Tematizando a una Postura Especifíca}
\subsection{Organización del Salón de Clase}
\subsubsection{Líneas Visuales}
\subsubsection{Diseño de Clase}
\subsubsection{Ofreciendo Props}
\subsection{Demostración}
\subsubsection{Demostración Silenciosa}
\subsection{Cuestiones de Salud}
\subsubsection{Estudiantes Lesionados}
\subsubsection{Usando Props}
\subsubsection{Cuidados Específicos de Salud}
\subsubsection{Yoga para Artritis}
\subsubsection{Fibromalgia}
\subsection{Lenguaje}
\subsubsection{Tono de Voz}
\subsubsection{Que Tanto Decir?}
\subsubsection{Se Conciso}
\subsubsection{Volumen/Contenido}
\subsection{Modificación de Postura}
\subsection{Observación: Estudiante Individual}
\subsubsection{Fundación}
\subsubsection{Estado de Animo General}
\subsection{El Rol del Maestro}
\subsection{Fundamentos de Secuenciado}
\subsubsection{Secuenciado Variable y Establecido}
\subsubsection{Principios de Secuenciado}
\subsubsection{Entretenimiento - Centrando la Clase}
\subsubsection{Secuenciando la Clase a Nivel Mixto}
\subsection{Creando Intención}
\subsubsection{La Intención de "Liberar Tensión"}
\subsection{La Práctica y el Servicio de Enseñar Yoga}

\section{Práctico}
\subsection{Tarea}
\subsubsection{Clases en Secuencia}
\subsubsection{Desarrollar Intención para Clases}
\subsubsection{Autoevaluación}

\section{Términos de Sanscrito}
\subsection{Glosario y Términos en Sanscrito}

\section{Un Entrenamiento Ejemplo}
\subsection{Un Entrenamiento Ejemplo - Yoga en Sillas}

\section{Ilustraciones}
\subsection{Flujo de Posturas}
\subsubsection{Surya Namaskara}
\subsubsection{Todos los Niveles de Practica de Asana}
\subsection{Posturas Syllabus}
\subsection{Ajustes Hands-on}



\begin{table}[h!]
	\centering
	\begin{tabular}{l|c||r}
		1 & 2 & 3\\
		\hline
		a & b & c\\
	\end{tabular}
	\caption{Caption for the table}
	\label{tab:table}
\end{table}


\begin{table}[h!]
	\centering
	\begin{tabular}{ccc}
		\toprule
		Some & actual & content\\
		\midrule
		prettifies & the & content\\
		as & well & as \\
		using & the & booktabs package\\
		\bottomrule
	\end{tabular}
	\caption{The other table}
	\label{tab:table2}
\end{table}


\newpage
áéíóú
\'a\'e\'i\'o\'u
ñ
\end{document}
