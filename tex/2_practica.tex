\chapter{Técnicas de Entrenamiento y Práctica}
\section{Técnicas de Asana}
\subsection{Alineamientos Fundamentales}
El cuerpo humano tiene una alineación óptima. Cuando el cuerpo se mueve hacia o en un alineamiento óptimo, hay un incremento en la libertad de movimiento de las articulaciones y más energía disponible en el cuerpo, dado que el cuerpo no está peleando consigo mismo y le es pocible moverse libremente. El dolor es tambi\'en reducido o eliminado cuando los huesos y tejidos del cuerpo se encuentran en una relación cooperativa.

La habilidad para realizar una asana puede variar enormemente, dependiendo de facotes como la estructura osea, lesiones previas, edad, nivel de enería y empeño del estudiante. Comenzar una práctica de asana como disciplina de consciencia y no violencia (especialmente a los practicantes) es un comienzo fundamental para una práctica libre de dolor.


\subsection{Biomecánicas Holísticas}
\begin{itemize}
	\item Holistico - ``visto como un todo, integrar.''
	\item Biomecánico - ``el estudio y la aplicación de fuerzas físicas benficas en seres sensibles.''
	\item O: ``No uses el cuerpo en esa postura\ldots usa la postura para entrar en el cuerpo''
\end{itemize}

\textbf{Biomecánizas holísticas en pocas palabras:}
\begin{itemize}
	\item Reconoce que cada parte del cuerpo humano (cuerpo, mente y emociones) es involucrada en cualquier actividad, sea una postura de yoga, una discusión o comer una comida.
	\item Reconoce el potencial para la salud a trav\'ez de la aplicación apropiada de conciencia y forza la reación de relaciones armoniosas entre articulaciones, músculos, huesos, mente y emociones.
	\item Reconoce las limitaciones estructurales del cuerpo y trabaja dentro de esos límites.
	\item Reconoce que cada persona es diferente (músculos, huesos, mente y corazón). La aplicación de cualquier t\'ecnica debe ser adaptativa a las necesidades individuales de esa persona.
\end{itemize}

Una obstrucción primaria para los estudiantes de Hatha Yoga (yoga físico) es intentar interpretar el lenguaje de un maestro conforme dirige el movimiento de ciertas partes de tu cuerpo en formas específicas. El distinto acercamiento a las posturas de yoga puede dar algunas veces conflicto verbal de instrucciones a elos estudiantes. El siguiente trabajo intenta ayudarte como estudiante a entender el trabajo de tu cuerpo en acción.

Desde el trabajo del maestro T. Krishamacharya y otrs de 1930 en adelante, el potencial de curar al cuerpo y obtener un completo rango de mobilidad y vibración física a partir de la práctica de Hatha Yoga ha sido reconocida. El buen alineamiento y la aplicación de fuerza apropiada en el cuerpo físico es la llave. Sincronizando la mente con el cuerpo e incluso emociones es lo que yo llamo ``Biomecánicas holísticas''.

Las biomecánicas holísticas toman la perspectiva del yoga llamada ``Tantra'', y acepta que el cuerpo no es simplemente un vehículo inherte para un espíritu separado, sino un universo inteligente en miniatura. Fuerzas en nuestro universo interactúan en ciertas formas predecibles que son reflejadas en nuestro cuerpo, mente y emociones. La aparente separación de cuerpo, mente y el mundo exterior es una ilusión.

Los movimientos y direcciones de energía o prana en el cuerpo han sido descritos anteriormente en los Upanishads, una antigua compilación de sabiduría yoguica obtenida por varios autores. Muchas escuelas modernas de yoga utilizan antiguos entendimientos de las energías sutiles en el cuerpo, y las describen en diferentes formas.

La perspectiva oriental de las energías sutiles y de cómo el prana se mueve benficamente dentro de el cuerpo junto con algunos modernas (generalmente occidentales) entendimientos de las relaciones de músculos y huesos, conforman la biomecánica holística.

\textbf{Metodologílsa}
Cuando se realizan los estramientos terrenales que llamamos ``asana'', cualquier forma de nuestro cuerpo tiene el potencial de beneficiar la elasticidad y fuerza, o causar lesiones. Si no hacemos nada, sólo nos sentamos en el sofá, por ejemplo, hay poca probabilidad de algún daño inmediato en el cuerpo.

Existe tambin una posibilidad muy alta de eventualmente perder salud debido al atrofiamiento de tejidos del cuerpo, y correspondiendo contraindicaciones de la mente y el cuerpo emocional. La salud se encuentra entre hacer nada y hacer mucho. Cuando el cuerpo se mueve con habilidad y armonía, la intensidad del ejercicio puede ser incrementado con seguridad y los beneficios incrementan tambin. Un ejercicio hecho sin cuidado tiene más alto potencial de lesiones y menos potencial de beneficio, poque mejoramosen las cosas que practicamos.

Las siguientes decuencias de acciones son una síntesis, tomada del trabajo de maestros de yoga, fisicoterapeutas occidentales, observación de animales y práctica personal. Estas tcnicas no pertenecen a nadie, son parte de nuestra cultura e intelectual común. Cómo puedes saber si la t\'ecnica de un fisioterapeuta está funcionando? Se siente bien. Ese reconocimiento es innato, como el deseo de estirarse y descansar, ese conocimiento es intrínseco en todos los seres. (Opuesto de sedentario y sobre-activo)

\textbf{Samasasthiti (pronunciado, sama stee tee hee)}
Samasthiti es la palabra en sánscrito que significa ``esparcir la luz de la consciencia a travs del cuerpo.'' Samasthiti es el estado de consciencia dentro de la postura inicial de levantarse listo frente al tapete, y en cualqueier postura, demostrar un estado que permite la concentración luminosa y compasiva.

El resultado emocional e intelectual de esta ación es una receptividad, físicamente una falta de armonía exterior y sin embargo una fuerte base de las partes de tu cuerpo que tocan la tierra.

\textbf{Integración}
Como "condensaci\'on de consciencia", el practicante dirije los tejidos del cuerpo como uno hacia el centro, localizado a lo largo del eje de la columna vertebral, entre los hombros y las caderas.

Los hombros y caderas se mueven hacia atras en su "hogar" estructural:
\begin{itemize}
	\item los muslos se mueven hacia los izquiotiviales, suavizando el frente de las ingles; y
	\item los hombros descanzan c\'omoda y fuertemente hacia la parte trasera del cuerpo.
\end{itemize}
(Tadasana de nuevo, energ\'ia moviendose hacia adentro)

\textbf{Expansi\'on}
Una vez que la integración se ha llevado a cabo, la energia sublime del cuerpo que se encuentra en distintas formas, se basa en previa integración muscular, fluye del centro del cuerpo a los ejes de la espina, desde abajo de las piernas como las raíces de un árbol, hasta la cima de la espina, brazos y cabeza, como ramas de un árbol.

(energía movi\'endose hacia afuera)

En pocas palabras, la práctica de esas tres acciones: Suavizar, Flexionar y Estirar. Estas acciones pueden ser realizadas en cualquier postura, posición o actividad dentro de una práctica formal de yoga o en actividades diarias incluso, como lavar los platos.

Los beneficios de una fuerza incrementada, un mayor rango de movilidad y una sensación de serenidad aparecen porque el cuerpo, la mente y las emociones son conducidad en la fábrica de nuestro mundo, de nuestro universo. No estamos separados de todo lo que vemos, así que cuando reconozcamos nuetra conección fundamental, y nos comportemos como todas las cosas en el universo lo hacen (pulsando con expansión y contracción) entonces encontraremos esta relación armoniosa. No una relación de inactividad o separación, sino una de participación con la vida.

\textbf{Dolor contra intensidad:}
Para un practicante principiante, la sensación experimentada durante una práctica de asana es con frecuencia desconocida. Con más experiencia, una diferenciación puede ser encontrada entre dolor e intensidad. Dolor repentino, sensaciones desagradables, especialmente alrededor de las articulaciones no debe ser ignorado. El cuerpo envía una señal de desalineación o desconexión que podría ser dañina.

Sensaciones de estiramientos intensos a un músculo puede ser interpretado como dolor, pero la sensación es muy distinta. Con frecuencia un practicante se encuentra en control de la cantidad de sensaciones experimentadas, como en una flexión sentado hacia el frente. Estas sensaciones de intensidad son intrínsecamente parte de la práctica. Generalmente respirar en las posturas intensas recoge la mente del cuerpo y la resistencia decrementa.

\subsection{Terapia Estructural}

La terapia estructural es la aplicación consciente de las holisticas biomecanicas de un practicante a su cliente, con una completa participación del cliente. Tocar con vigor establece resonancia entre el practicante y el cliente, antes y durante los ajustes realizados. Con práctica, el practicante aprende a diferenciar entre tipos de resistencia: compresión, tensión, músculo y tejido conectivo. Junto con la resistencia física, pueden ser encontrados patrones de resistencia en los cuerpos emocionales y mentales.

La realización de ajustes será de poco valor a largo plazo a menos que se ilustre al cliente ómo realizar una buena alineación, con apropiada integración y expansión. La idea es que el cliente gane consciencia kinestésica de cómo crear alineación curativa para ellos mismos.

\textbf{Qué tanta presión?}
Cuando se aplica un ajuste, primero debes conectar con la parte del cuerpo que vas a ajustar con seguridad, incluso presión. Moverás piel, músculo y hueso como una unidad. Poca presión será inefectiva. El ajuste debe comenzar desde la más baja intensidad e ir incrementando hasta el máximo en un período de 3 o 4 segundos, dl ajuste mismo puede durar desde 5 segundos hasta 1 minuto.

Mantén tu atención en la cara del cliente buscando signos de inconformidad, dolor o alivio. Cuando trabajes con piernas o cadera, tus manos y brazos son usualmente menos fuertes que con lo que estás trabajando. Cada cuerpo es distinto y la sensitividad o sensación es distinta de persona a persona. Mantente comunicado con tu cliente, revisando si está experimentando dolor o alivio.

\textbf{Areas clave de imbalance}

Desalineación ocurre donde los huesos se encuentran. La desalineación habitual ocurre en el estilo de vida: sentado por largos períodos, movimientos repetitivos que crean imbalance muscular, y posiblemente nuestra forma de estar erguidos y caminar como tales.

Es posible que desde una perspectiva evolutiva, aún nos encontremos en evolución física y que no estemos por completo adaptados a una posición erguida. La parte superior de nuestro fémur encaja mejor en el hueco de la cadera en una posición inclinada en lugar de en una orientación vertical, la parte exterior de nuestras piernas se encuentran mas rígidas cuando nos encontramos de pie igualmente. Nuestros muslos interiores son difíciles de involucrar muscularmente para rebalancear la rotación común del femur debido a la rigidez de las piernas exteriores.

Los grupos musculares involucrados en llevar los homóplatos hacia la línea central del cuerpo incluyen a los romboides. Esos músculos son bastante mas complicados de involucrar que otros. Es posible que estos grupos musculares se hayan desarrollado relativamente recientemente en nuestra evolución y aun no se encuentren mapeados del todo en nuestro cerebro. Ese desequilibrio potencial que se manifiesta cerca del eje de nuestro cuerpo puede crear más inestabilidad que la periferia, incluyendo rodillas y muñcas.

\textbf{Compresión y tensión}
Existen dos fuerzas físicas que limitan el rango de movimiento del cuerpo; estas fuerzas limitantes incluyen tensión en el tejido del cuerpo (músculo y tejido conectivo) y compresión. La tensión es fácil de reconocer; es la parte del "estiramiento" en el hatha yoga. Compresión, por otro lado, ocurre cuando dos huesos se encuentran. Un ejemplo sencillo es tu codo. Si lo extiendes por completo, dos huesos se unirán, no importa que tan flexible seas, tu rango de movilidad no va a incrementar. Al aplicar la fuerza apropiada a un hueso, su densidad incrementará, por lo que algo de compresión es buena si el objetivo es estimular la densidad ósea. Intentar forzar cualquier articulación mas alla de su compresión para lograr alguna forma exterior del cuerpo puede causar dolor o lesiones.

Los huesos de cada persona son diferentes en longitud, forma y densida. Los huesos pueden tener rotaciones dentro de sus cabidades que hagan que, en el caso del fémur, un pie puede rotar hacia adentro más hacia adentro o hacia afuera que tro, mientras que la cabeza del fémur se encuentra alineada. La limitación eventual de la profundidad de cualquier postura de yoga será la forma y tamaño de los huesos Los huesos pueden incrementar o decrementar en densidad, adaptandose al estres que se ponga en ellos, pero los huesos adultos no pueden cambiar en tamaño o forma hasta donde sabemos.

\textbf{Hombros - Anatomía y función}
El grupo de los hombros se conforma por la clavícula, escápula y huesos húmeros que contienen también muchos músculos dentro y alrededor de ellos. Lo mas sencillo es verlos como una unidad funcional. El diseño del grupo de los hombros nos permite un mayor rango de movilodad a costa de un poco de estabilidad, comparado con la cadera por ejemplo. El único lugar de conexión ósea es la unión esterno-clavicular, localizada al frente del cuerpo. Si pudieras desabrochar esta unión, serías capáz de casi quitarte los hombros como un abrigo.

La mejor colocación para los hombros es en la parte tasera del cuerpo. Cuando lso hombros se encuentran hacia atrás:
\begin{itemize}
	\item los homóplatos se deslizan hacia atrás sin un "aleteo" significativo;
	\item las clavículas son casi invisibles en el frente del cuerpo; y
	\item los brazos cuelgan de los hombros con las palmas ligeramente rotadas al frente
\end{itemize}

\textbf{Cadera/Espalda baja - Anatomía y función}
La cadera son una unidad funcional que incluye los huesos pélvicos, el fémur y la parte mas baja de la columna, la cual se encuentra colocada entre los huesos pélvicos. Esta area sostiene el peso de la parte superior del cuerpo y lo transfiere abajo hacia las piernas, tal como un puente de pidra arqueado soporta el peso de arriba. El rango de movilidad de el hueco de la cadera se encuentra limitado por la profundidad y tamño de esta misma, su forma y el ángulo del fémur, y posiblemente la tensión articular.

La posición mas ventajosa para la cadera y espalda baja es cuando nos encontramos parados con los pies paralelos y hombros recogidos hacia atrás del cuerpo. Esto refuerza la curva natural de la espina y ayuda a enraizar la parte baja del cuerpo.

La desalineación mas común en las caderas y la espalda baja es una falta de curvatura en la base de la espina. La planicie crea presión en los nervios entre las vértebras y decrementa el rango de movilidad. Esta planicie se encuentra también relacionada a la rotación externa del fémur y a bajar la parte posterior de nuestros piés, lo cuál se convierte a su vez en mas presión a la espalda baja.

Por el caso contrario, demasiada curvatura hacia adentro, y no suficiente alargamiento en la columna, puede a su vez crear dolor debido a la presión en las nervios en las vértebras. Una curvatura excesiva en la espalda baja se encuentra relacionada con la rotación interna del fémur y un exceso de peso en las partes internas de los pies, rotando las rodillas hacia adentro.

\textbf{Muñecas - Anatomía y función}
La articulación de las muñecas incluye los dos huesos de los brazos y los pequeños e irregulares huesos de las manos llamados "carpelos". La compleja construcción de las muñecas nos da una increible destreza y rango de movilidad en los dedos.

Una buena alineación de la muñecas es importante cuando se carga peso en las manos. Cuando nos encontramos en cuatro puntos, se colocan las manos aproximadamente a la distancia de los hombros con los dedos cómodamente separados y los dobleces de las muñecas paralelas al frente de tu tapete de yoga. Muscularmente involucrar las manos creara un hueco en la palma y proveerá soporte estructural.

Dado que utilizamos nuestras manos constantemente, las muñecas son suceptibles a una tensión constante lo que inflama los huecos de los carpelos. Esta tensión repetitiva ocurre seguido en trabajos donde el mismo movimiento de manos y brazos es realizado una y otra vez, como los cajeros de las tiendas, por ejemplo.

\textbf{Alineación del cuerpo exterior:}
Esto se refiere a la estructura básice de alineación del cuerpo dada una postura. Las articulaciones principales del cuerpo (tobillos, rodillas, cadera, hombros, codos, muñecas y cuello) son lugares de gran capacidad de movilidad, y por lo tanto los que tienen mas a una mala alineación. En una práctica de asanas, el cuerpo toma varios formas. Físicamente, los tejidos del cuerpo deben fortalecerse y alargarse para adaptarse a esas formas.

La alineación de una parte del cuerpo (rodillas, por ejemplo) afecta a la alineación de el resto de las partes, tal como una base dispareja en una casa causaría un primer piso inestable, tercer piso inestable y así susecivamente. Los ajustes a nuestro cuerpo exterior en una postura naturalmente comienzan en la base y se mueve secuencialmente hacia arriba.

\textbf{Preparando la fundación}

éáíñúó

\subsection{Técnicas de Asana: Categorías de Posturas}
\subsection{Forma y Acción}
\subsection{Navegando el Tapete}
\subsection{Sacro}
\subsection{La Práctica de Asana}
\subsection{Yin y Yang}
\subsection{Polaridades de la Energía Física}
\subsection{Fuerzas Opuestas}
\section{Téecnicas de Purificación}
\section{Meditación}
\section{Pranayama}
\subsection{Nadi Shodhana}
