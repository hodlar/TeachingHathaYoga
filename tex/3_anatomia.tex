\chapter{Anatomía y Psicología}
\section{Chakras}
Un chakra es el centro de actividad que recive, procesa y expresa energía de fuerzas vitales o prana. La palabra sánscrita chakra se traduce como ``rueda`` o ''disco`` y se refiere a una esfeza giratoria de bio energía. Hay, en este modelo particular, siete chakras posisionados en una columna de energía de la base de la espina a la coronilla de la cabeza. Los siete chakras mayores que se correlacionan con los estados básicos de consciencia. Como transformadores de energía, ellos bajan de la energía universal de consciencia al plano físico. De esta forma nosotros estamos conectados a la fuente de energía, se encuentra disponible para nosotros en diferentes formas de energía. Similar a los conectores elctricos, diferentes formas de energía son apropiadas para diferentes usuarios.\\
Los colores asociados con los chakras son tambi\'en una forma de energía. Color es energía expresada como una onda de luz que podemos ver (existen ondas de luz que no podemos ver con nuestros ojos). Existen tambi\'en sonidos correspondientes, asociados a cada chakra. De nuevo, el sonido es solo otra forma de energía vibrante a distintas frecuencias.\\

\subsection{Primer Chakra: Muladhara}
Tierra, identidad física, orientado a autopreservación. Color rojo. Localizado en la base p\'elvica. Este Chakra forma nuestra base. Está relacionado a nuestos instintos de supervivencia y a nuestro sentido de apego y conexión a nuestros cuerpos y al plano físico. Idealmente este chakra nos otorga salud, prosperidad, seguridad y presencia dinámica.\\
\subsection{Segundo Chakra: Svadhisthana}
Agua, identidad emocional, orientado a autogratificación. Color naranja. Colocado en el área del sacro. Este chakra está relacionado al elemento agua, y a las emociones y sexualidad. Se conecta a nosotros a trav\'es de los sentimientos, deseos, sensaciones y movimiento. Idealmente este chakra nos brinda fluidez y gracia, profundidad en los sentimientos, realización sexual y la habilidad de aceptar el cambio.
\subsection{Tercer Chakra: Manipura}
Fuego, identidad individual, orientacion a autodefinición. Color amarillo. Localizado en el plexo solar. Comanda nuestro poder personal, deseo, autonomía y metabolismo. Cuando se encuentra saludable, este chakra otorga energía, eficiencia, esponaneidad, y poder no-dominante.
\subsection{Cuarto Chakra: Anahata}
Aire, identidad social, orientado a autoaceptación. Color verde. Localizado en el corazón. Está relacionado a la verdadera compasiń y es el integrador de opuestos: izquierdo y derecho, arriba y abajo, hombre y mujer, expansión y contracción. Un cuarto chakra saludable nos permite amar profundamente, sentir empatía y tener una profunda sensación de paz y concentración.
\subsection{Quitno Chakra: Vishudha}
Sonido, identidad creativa, orientado a la autoexpresión. Color azul. Este chakra se encuentra localizado en la garganta y por lo tanto está relacionado con la comunicación y creatividad. Aquí nosotros experimentamos el mundo simbólicamente a trav\'es de vibraciones, tales como la vibración del sonido representando lenguaje.
\subsection{Sexto Chakra: Ajna}
Luz, identidad de estereotipo, orientado a autoreflexión. Localizado en el entrecejo (tercer ojo). Color índigo. Está relacionado al acto de ver, tanto física como intuitivamente. Como tal abre nuestras facultades psíquicas. Cuando se encuentra saludable nos permite ver claramente, completamente, permitiendonos \"ver la imagen completa\".
\subsection{S\'eptimo Chakra: Sahasrara}
Pensamiento, identidad universal, orientado al autoconocimiento. Color violeta. Localizado en la coronilla de la cabeza. Este chakra se relaciona con la consciencia como consciencia pura. Es nuestra conexión con la Consciencia pura al nivel universal. Cuando se desarrolla, este chakra nos otorga conocimiento, sabiduría, entendimiento, conexión espiritual y dicha.
\\
Una práctica apropiada de asanas puede ayudarnos a balancear estas energías sutiles del cuerpo. Los chakras se balancean por medio de unir las energías de Siva (consciencia) y Shakri (creción). Cuando se encuentran balanceados, cada chakra funciona óptimamente, dándonos acceso espontáneo a todas las formas de energía corporales. La meditación de chakra es una excelente forma de mejorar tu entendimiento de esos centros, al igual que una dieta propia y elecciones de estilo de vida.

\section{Los Vayus}
El entendimiento yoguico del cuerpo es experienciado en lugar de teótico. El entendimiento fundamental es que el cuerpo es una expresión de la fuenta universal, como una onda es una expresión del oc\'eano.  Dentro de esta expresión universal hay formas en que la energía vital fluye, como corrientes en un cuerpo de agua. Los yoguis dieron nombres a estas corrientes, y varias escuelas de yoga suelen nombrar a esas expresiones de energía de diferentes formas. La fuerza esencial del cuerpo es conocida como prana, la menor unidad de fuerza vital.\\
Prana y la respiración se encuentran intimamente unidos. Prana mueve la respiración. Sin fuerza vital, no hay respiración, no hay otra forma. Podemos interactuar de cierta forma con esta fuerza vital conocida como prana al sentir y manipular la respiración (incluso deteniendo la respiración por un período de tiempo). Dentro de un ciclo respiratorio, prana se vuelve perceptible.\\
Los cinco Vayus principales:
\begin{itemize}
	\item Prana - la asendencia del flujo de energía, que puede ser sentido en la inalación.
	\item Apana - la desendencia del flujo de energía, percibido en la exalación.
	\item Samana - la corriente de energía que se digire, se representa hacia nuestro centro.
	\item Udana - la corriente de energía que se consume conforme se expande a las extremidades desde nuestro centro.
	\item Yyana - la corriente integradora de energía que mantiene el equilibrio.
\end {itemize}
Conforme un principiante mueve el cuerpo en una práctica de asanas, es en un inicio usualmente una experiencia ``corporal exterior''. Posturas de formas básicas, sentimientos de rigidez o fatiga en partes del cuerpo son notables. Conforme la práctica continúa, se hacen conscientes  sensaciones más sublimes. Es aquí cuando el trabajo con los vayus puede comenzar.\\
Prana puede ser sentido como una fuerza que sube en la inalación cuando los brazos son levantados sobre la cabeza.\\
Apana puede ser sentido hacia abajo en la exalación cuando los brazos se colocan hacia los lados del cuerpo.\\
Samana puede sentirse como una fuerza integradora, dibujando flechas en el cuerpo indicando al centro.\\
Udana puede sentirse como una expresión sublime de expansión, o hacia afuera muviendo la energía desde el centro del cuerpo.\\
Vyana puede ser experimentado al esparcir consciencia atrav\'es del cuerpo, notando cómo diferentes partes pueden comunicarse y son mantenidas juntas.
\section{Compresión y Tensión}
\section{Anatomía Funcional}
\subsection{Huesos y Articulaciones}
\subsection{La Espina}
\subsection{Muscúlos}
\subsection{Muscúlos y Posturas}
\subsection{Yoga y Posturas}
\newpage
\section{Los Bandhas}
Bandha significa \lq\lq candado\rq\rq. Este tipo de candado, en lugar de cerrar, como el tipo de candado donde se requiere una llave para abrir, fue de hecho un termino figurado. Estos candados son como una zanja usada para dirigir el agua a diferentes partes de un campo. Bandhas en el cuerpo son usados para dirigir la energía tanto física como energeticamente. Físicamente, los bandhas funcionan para mantener estimulados y armonizados nuestros órganos internos. Energeticamente, asisten al movimiento del prana, o energía en el cuerpo. Hay tres tipos de bandhas usados en la práctica de las asanas:

\subsection{Mulabandha}
Localizado entre el ano y los genitales, es el músculo perineo para los hombres. Para las mujeres está ubicado cerca del límite superior del cuello del útero. La activación del Mulabandha no es una fuerte contracción forzando los músculos que lo rodean, es mas sutil que eso. Mulabandha puede ser experimentado activando los músculos hacia atras, incrementando la curvatura lumbar en la espina, despus permitiendo al coxis alargarse, estimulando al abdomen a activarse y la base de la pelvis a levantarse.

Activar los muslos hacia atras acomoda las cabezas del femur hacia atras y crea una expansión en el área pélvica.
Bajar el sacro reafirma la carne de los glúteos. El abdomen bajo se mueve de la pubis al ombligo.

La sinergía creada por esos dos complementarias, y a la vez opuestas fuerzas, crean Mulabandha. En lugar de endurecer o relajar en el área pélvica, un levantamiento es creado similar a lo que sería el último medio centímetro de una malteada por un popote.

\subsection{Uddiyana Bandha}
Localizado ligeramente abajo del ombligo, Uddiyana Bandha significa \lq\lq Volar hacia arriba\rq\rq refiriendose a su efecto en el prana. Este segundo banda es activado en una forma parecida a Mulbandha, con un mínimo de endurecimiento externo o contracción. En el proceso de crear este candado, el centro del plexo solar es llevado adentro y hacia arriba en un levantamiento abdominal y es cuando se encuentra la activación. En expresión completa es llevado acabo exalando completamente y luego llevando el abdomen bajo hacia adentro y arriba, mientras se eleva el diafragma. Este nivel de Uddiyana Bandha será usdo en la práctica de la retención de exalación en Prnayama, pero debido a la incapacidad de inalar mientras se ejecuta este nivel, simplemente mantener tranquilidad cerca de tres dedos debajo del ombligo otorga espacio para que el diafragma baje durante cada exalación. Conforme el diafragma baja, la respiración es impulsada a moverse hacia las costillas, espalda y pecho. En cada exalación los músculos abdominales promueven a completar el vaciado de los plmones. El proceso toma práctica, y las sutilezas de la relación entre respiración y bandhas debe ser explorada experimentalmente.


\subsection{Jalandhara Bandha}
Este candado es creado levantando y girando los hombros  hacia atras para primero ampliar y luego levantar el pecho. Despues la parte trasera de la cabeza se extiende hacia el cielo y la barbilla se mueve en una contracción, la cuál es formada donde dos huesos de la clavícula se encuentran. El candado ocurre espontáneamente en algunas posturas como el parado de hombros, pero no es usado tan ampliamente como los otros dos candados.

\section{La Respiración}
\subsection{Respiración Ujjayi}
\section{Elementos de la Naturaleza}
\subsection{Cualidades Caracteristicas de los Cinco Elementos}
\subsection{Ayurveda}
\section{Los Cinco Koshas}


