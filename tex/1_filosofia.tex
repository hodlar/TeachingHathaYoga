\chapter{FILOSOFÍA, ESTILO DE VIDA Y ÉTICA}
texto
\section{Por qué el yoga pudo haber pasado}
\subsection{Los upanishads}
\subsection{El bagavad Gita}
\subsection{Los Yoga Sutras de Patanjali}
\subsection{Vedanta - La filosofía No-Dual del Sankara}
\subsection{Kashmir Saivism y los Tattvas}
\subsection{Yoga Moderno}
\subsection{Cronología del Yoga}
\section{Dieta y Estilo de Vida}
Es difícilmente necesario meditar en una caverna remota para crear un estilo de vida que ayude la práctica. Las prácticas de yoga presentadas en este manual son pensadas para ser integrales y adaptativas al mundo moderno. Entre el 1000 y 1400 DC, en Índia, la región de Kashmir vió una revolución en el penzamiento yóguico. Esos practicantes de yoga fueron, en realidad, amas de casa, esposos y esposas que tomaron la práctica de yoga y la convirtieron en la fábrica de sus vidas. Esta aproximación fue una no renuncia, entre otras prácticas, y fue después llamada tantra, que literalmente significa "tejido". Los yoguis tomaron el entendimiento moral más clásico de yoga y lo recrearon, redefiniéndolo en una práctica más significativa para ellos mismos.

"Una mente aguda, un corazón suave y un cuerpo vibrante" (John Friend) son entre otras cualidades de un maestro de yoga competente. Estas cualidades son apoyadas y fomentadas por:
\begin{itemize}
	\item descanso adecuado;
	\item relaciones alentadoras;
	\item dieta apropiada; y
	\item práctica personal y estudio.
\end{itemize}

Enseñanr una clase mientras te encuentras cansado, hambriendo, agitado o desconectado de tu práctica no es una experiencia de crecimiento. Si estás apropiadamente preparado, nutrido con comida, lleno de prana después de una buena práctica de yoga, bien descansado y sintiéndote apoyado, con seguridad darás una buena clase. El negocio del yoga y tu estilo de vida no pueden ser separados tal como en alguna otra ocupación. Como maestro, eres un ejemplo viviente de cómo la práctica de yoga funciona.

\subsection{Dieta}
La ecuanimidad de la mente ha sido siempre referida como una de las claves del yoga. "Yoga no es para quienes comen mucho, o no comen" (Upanishads). La dieta afecta directamente a nuestro ánimo y a nuestros cuerpos. Simplemente prestar atención a lo que comemos y a cuánto comemos es otra práctica de yoga, embebido en una práctica más larga. Extrema austeridad respecto a la dieta puede ser una forma que el ego toma para crear más separación. Como sea, comiendo una dieta balanceada con muchos vegetales y frutas es accesible para la casi todo. Disminuir los estimulantes como alcohol, azúcar y posiblemente cafeína creará un ánimo más estable y una atmósfera corporal donde la consciencia plena es más fácil de alcanzar. Comer una dieta vegetariana es considerado prificador por muchas ramas de la tradición del yoga. Comer la carne de un animal que ha sido mantenido en condiciones de vida no dignas será al finalde menor valor nutrimental que animales con libre albedrío. Nosotros, como seres sensibles, podemos elegir qué es útil para nosotros y cómo nuestras elecciones afectan a nuestros seres. Come con amigos o amados tanto como sea posible, dando bendiciones antes de comer.

\subsection{Sueño}
Hay una pulsasión universal llamada "Spanda". Desde esta pulsación viene la manifestación dual de la naturalesa; frío y calor, macho y hembra, luz y oscuridad. Una lugar fresco, oscuro y tranquilo y una cama cómoda para dormir recupera la energía vital del cuerpo y es el complemento natural a la actividad física. El ciclo del día es un modelo de vida, justo como la práctica del yoga y la postura de descanso Savasana es el modelo del día. Vive pleno y duerme profundo. Permite al día previo disolverse, brindando tu atención tan completamente como sea posible al momento presente.

\subsection{Práctica}
Enseñar yoga requiere la asimilación e incorporación de las enseñanzas, presentadas en tu propia voz y en tu propio modo. Conforme crecemos como maestros, agregamos nuestra propia voz al linaje de maestros. Es así como la tradición del yoga se mantiene viva y evoluciona. Al practicar por tu cuenta, tu sabiduría interna comienza a emerger. Al aceptar el conocimiento que ha sido revelado y al honrar a tus maestros, creces individualmente como maestro, y simultáneamente te conectas más con la energía universal. Elije un tiempo y lugar para la práctica diaria. El cuerpo cambia día a día. Algunos días nuestra energía está alta y el cuerpo se siente como una pluma. Otros días la energía corporal está menos disponible y se sentimos como concreto húmedo. Otros días de energía alta, tu práctica puede ser una celebración que es dinámica y retadora. En los días difíciles, desarrolla una secuencia de posturas más restaurativas para ayudar a aliviar la fatiga. Aún así no existe razón para dejar de practicar, y tus habilidades de enseñar a una variedad de estudiantes incrementará.

\subsection{Relaciones}
Todo en el universo es inatamente divino. La práctica de yoga no es algo que pase solamente en el tapete. Permanecer completamente consciente en relación con otros seres humanos es tal vez uno de los aspectos mas difíciles de la práctica. Es aquí el donde nuestra habilidad de recordar nuestra divinidad innata, y la divinidad de otros es revelada. Percibir patrones en tus relaciones. Percibir tus reacciones y deseos habituales. Abrazar las relaciones como parte de, no separarlas de tu práctica. De esta forma observarás la naturaleza sagrada de otros. Honra a aquellos que amas y aquellos que has amado.

\section{El principio de atracción}
\subsection{Yoga y el principio de atracción}
\subsection{Reciviendo}
\subsection{Práctica}
\section{Ética}
\subsection{Yamas y Niyamas}
\subsection{Mas de Ética de un maestro}
\section{Luz y Oscuridad}
\section{Mantras}
Un mantra es una invocación de sonidos sagrados, y como tal es otra forma de vibración en la forma de sonidos organizados. Repetir un mantra es una forma de yoga en sí mismo y es la práctica principal en Mantra Yoga. El efecto de la vibración importa. Incluso un instrumento musical, si está hecho de material orgánico como madera, absorverá las vibraciones resonando a travé de el. Si el instrumento se mantiene en sintonía y se toca regularmente, el tono del instrumento se profundiza, se vuelve más bello y permite acarrear la canción del música completa y acertadamente. El cuerpo humano en el caso de la repetición del mantra, es el instrumento, que toca la canción de la divinidad.

El mantra Gayatri es primero recordado en el \textit{Rig Veda} escrito en Sánscrito hace cerca de 2,500 a 3,500 años, y de acuerdo a algunas fuentas, puede haber sido cantado generaciones antes de eso.

\begin{multicols}{2}
	El Gayatri Mantra:\\*
	Om bhûr bhuva sva\\*
	tát savitúr várenyam\\*
	bhárgo devásya dhîmahi\\*
	dhíyo yó na prachodáyât
	\columnbreak
	\textit{Phonetic Pronunciation}
	Om burr buva-ha sva-ha\\*
	Tat sa-vi-tour vara-en ya-hum\\*
	Bar-go de vas-ya de my-hee\\*
	De yo-yo na pra-show-da-yat
\end{multicols}

\section{Visión general de los Estilos de Yoga}
Una variedad de caminos del yoga continúan entrelazándose e informándose entre sí. Algunos ejemplos son:
\begin{itemize}
	\item \textbf{Anusara Yoga:} Desarrollado por John Friend en 1997; unirica la filosofía enfocada al corazón Tántrica con principios de alineación bio-mecánicos.
	\item \textbf{Ashtanga Vinyasa Yoga:} Desarrollado por T.Krishnamacharya y su estudiante Pattabhi Jois; acercamiento sistemático y secuencial a la práctica de los asanas donde las posturas son separadas en series. Vinyasa, una conexión energetica de un asana a otro, es usado para crear y mantener calor y un movimiento de estado meditativo.
	\item \textbf{Ashtanga Yoga:} Llamado así por Baba Hari Dass, despus del camino de ocho pasos de Patanjali, no confundirlo con Ashtanga Vinyasa Yoga.
	\item \textbf{Bikram Yoga:} Una secuencia de veintiseis posturas en un cuarto calentado a 100 grados Fahrenheit.
	\item \textbf{Iyengar Yoga:} B.K.S. Iyengar, otro de los estudiantes de Krishnamacharya, refinó el aprendizaje de su gurú despus de mudarse a Pune. Avandonó el estilo Vinyasa y se enfocó en las enseñanzas de la salud, alineación estructural y beneficios terapeuticos de las posturas.
	\item \textbf{Kundalini Yoga:} Despertar energía, Kundalini yoga llegó al oeste en 1969, cuando Sikh Yogi Bhajan desafió la tradición y comenzó a enzeñarlo públicamento. Esta práctica es designada para despertar la energía Kundalini, la cuál es almacenada en la base de la espina y en ocasiones representada como una serpiente enroscada. Kundalini mezcla cantos, prácticas de respiración y ejercicios de yoga. El enfasis no es en los asanas, sino en los cantos y respiraciones.
	\item \textbf{Mysore Style:} Nombrado así en honor a la ciudad en india donde Pattabhi Jois enseña el metodo Ashtanga Vinyasa; una práctica autoencaminada con supervición y ajustes físicos de un instructor.
	\item \textbf{Vijnana Yoga:} Una práctica de Hatha Yoga desarrollada por Donna Holleman y Orit Sen-Gupta, basada en siete \"principios vitales\" diseñados para usar el cuerpo para explorar los más profundos niveles de nuestro ser.
	\item \textbf{Viniyoga:} Esta forma gentil de flow yoga pone gran \'enfasis en la respicación y cordina respiración con movimiento. El movimiento fluido de Viniyoga o Vinyasa es similar a la dinámica de la serie de poses de Ashtanga, pero ejecutado con una gran reducción de paz y nivel de estr\'es. Las posturas y secuencias son elejidas para encajar en las habilidades del estudiante. Se enseña al estudiante cómo aplicar las herramientas de yoga: asana, cántos, pranayama (control de la respiración), y meditación, en una práctica individual. Desarrollado por T.K.V. Desikachar, el hijo de Krisnamacharya (maestro de algunos grandes instructores de yoga incluyendo Iyengar y Pattabhi Jois), Viniyoga pone menos estres en uniones y rodillas manteniendo las posturas con una ligera flexión en las rodillas. Viniyoga es considerado excelente para principiantes, y es incrementalmente usado en ambientes rerap\'euticos.
	\item \textbf{Yin Yoga:} Un t\'ermino creado por Paul Grilley para describir una forma de práctica con un \'enfasis en posturas mantenidas por mucho tiempo, usualmente sentados, boca abajo o boca arriba. Yin Yoga se enfoca en fortalecer y alargar el tejido conectivo, que en turno, a partir de líneas meridianas, tiene un efecto óptimo en el funcionamiento de los órganos.
\end{itemize}

\subsection{Viendo la Imagen Completa}

\section{El negocio del Yoga}
\subsection{Marketing}
\subsection{Yoga en casa}
\section{Sencillez Voluntaria}
\subsection{Principios de Sencillez}
\subsection{Acercamiento a Sencillez}
\subsection{Un Acercamiento al Lado Financiero}
