\chapter{Preface}
"Cuando estás inspirado por algún gran propósito, algun proyecto extraordinario, todos tus pensamientos rompen sus ataduras: Tu mente traciende limitaciones, tu conciencia expande en todas direcciones y tu te encuentras en un nuevo, grande y maravilloso mundo. Fuercas dormidas, facultades y talentos cobran vida, y tu descubres que eres una persona mucho mejor de lo que jamás soñaste que serías."

-Patanjali, los Yoga Sutras

\newpage
Acerca de poseer el Yoga

Los materiales presentados en este manual representan una composición e interpretación personal del yoga. Yoga es tanto una ciencia como un arte de profundo entendimiento de la condición humana. En mi investigación en este tema, he tenido la buena fortuna de conocer grandes maestros, cuya sabidiría ha derramado luz en mi propia búsqueda por un entendimiento mayor. Tal acercamiento, con el tiempo, beneficiará tanto a estudiantes como a maestros. Cualquier estilo que sea tu práctica, yoga es una práctica de revelación. Lo que es revelado es nuestra verdadera naturaleza como un aspecto de Fuente.

Esta fuenta, como el océano, subyace en todas nuestras cualidades individuales. Una de las cualidades del ego individual es la noción de pertenencia. Pertenencia es manifestada en la forma de: escrituras de tierras que han estado aquí milenios antes de que pusieramos una cerca alrededor de ellos; el deseo de acumular bienes en nuestro nombre; o el patentar ideas. Ultimamente, dejamos todo atrás excepto el entendimiento que cultivamos, y es nuestro privilegio como maestros compartir ese conocimiento.

El yoga puede no ser poseido, porque el yoga como forma de conciencia adquirida, es una parte intrínseca de nuestra naturaleza. Cada vez que tomas una profunda y conciente respiración, has experimentado el yoga.

Namaste,

Dan Clement

\newpage
Nota de traductor.

He decidido traducir este texto como un intento de proporcionar las herramientas necesarias a quienes quieren profundizar sus conocimientos en la disciplina del yoga, agradezco a Daniel por compartir sus conocimientos por medio de esta obra y al internet por permitirme distribuírla (=

Cabe mencionar que algunos capítulos han sido excluídos de la obra original, dado que mi opinión al respecto de esos temas difiere un poco, pero tal vez en un futuro decida agregarlos.

Cualquier comentario o sugerencia, será bien recibida a mi correo carlosrdz.isd@gmail.com, y una disculpa por los errores gramaticales, semánticos etc.

Carlos Rodríguez
