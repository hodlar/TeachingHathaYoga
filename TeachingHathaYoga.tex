\documentclass[a4paper]{book}



\usepackage[T1]{fontenc}
\usepackage[utf8]{inputenc}
\usepackage{booktabs}

\title{Enseñando Hatha Yoga}
\date{2016-01-27}
\author{Daniel Clement with Naomi Clement}

\begin{document}
\pagenumbering{gobble}
\maketitle

\newpage
Copyright c 2007 Daniel Clement
Todos los derechos reservados. Sin limitaciones de derechos bajo derechos de autor, ninguna parte de esta publicación debe ser reproducida, almacenada o introducida en un sistema de
retribución, o transmitida, en cualquier forma o por cualquier medio (electrónico, mecánico, fotocopiado, grabado o de otra forma), sin el previo consentimiento escrito del propietario de los derechos, excepto para cortas reseñas.
Primer impresión Octubre 2007, segunda impresión 2008, tercer impresión 2009, cuarta impresión 2010, quinta impresión 2011.
Contacta al publicador en la web www.opensourceyoga.ca

ISBN: 978-0-9735820-9-3

\newpage
\tableofcontents

\newpage
"Cuando estás inspirado por algún gran propósito, algun proyecto extraordinario, todos tus pensamientos rompen sus ataduras: Tu mente traciende limitaciones, tu conciencia expande en todas direcciones y tu te encuentras en un nuevo, grande y maravilloso mundo. Fuercas dormidas, facultades y talentos cobran vida, y tu descubres que eres una persona mucho mejor de lo que jamás soñaste que serías."

-Patanjali, los Yoga Sutras

\newpage
Acerca de poseer el Yoga

Los materiales presentados en este manual representan una composición e interpretación personal del yoga. Yoga es tanto una ciencia como un arte de profundo entendimiento de la condición humana. En mi investigación en este tema, he tenido la buena fortuna de conocer grandes maestros, cuya sabidiría ha derramado luz en mi propia búsqueda por un entendimiento mayor. Tal acercamiento, con el tiempo, beneficiará tanto a estudiantes como a maestros. Cualquier estilo que sea tu práctica, yoga es una práctica de revelación. Lo que es revelado es nuestra verdadera naturaleza como un aspecto de Fuente.

Esta fuenta, como el océano, subyace en todas nuestras cualidades individuales. Una de las cualidades del ego individual es la noción de pertenencia. Pertenencia es manifestada en la forma de: escrituras de tierras que han estado aquí milenios antes de que pusieramos una cerca alrededor de ellos; el deseo de acumular bienes en nuestro nombre; o el patentar ideas. Ultimamente, dejamos todo atrás excepto el entendimiento que cultivamos, y es nuestro privilegio como maestros compartir ese conocimiento.

El yoga puede no ser poseido, porque el yoga como forma de conciencia adquirida, es una parte intrínseca de nuestra naturaleza. Cada vez que tomas una profunda y conciente respiración, has experimentado el yoga.

Namaste,

Dan Clement

\newpage

\chapter{FILOSOFÍA, ESTILO DE VIDA Y ÉTICA}
texto
\section{Por qué el yoga pudo haber pasado}
\subsection{Los upanishads}
\subsection{El bagavad Gita}
\subsection{Los Yoga Sutras de Patanjali}
\subsection{Vedanta - La filosofía No-Dual del Sankara}
\subsection{Kashmir Saivism y los Tattvas}
\subsection{Yoga Moderno}
\subsection{Cronología del Yoga}
\section{Dieta y Estilo de Vida}
\subsection{Sueño}
\subsection{Práctica}
\subsection{Relaciones}
\section{El principio de atracción}
\subsection{Yoga y el principio de atracción}
\subsection{Reciviendo}
\subsection{Práctica}
\section{Ética}
\subsection{Yamas y Niyamas}
\subsection{Mas de Ética de un maestro}
\section{Luz y Oscuridad}
\section{Mantras}
\section{Visión general de los Estilos de Yoga}
Una variedad de caminos del yoga continúan entrelazándose e informándose entre sí. Algunos ejemplos son:
\begin{itemize}
	\item \textbf{Anusara Yoga:} Desarrollado por John Friend en 1997; unirica la filosofía enfocada al corazón Tántrica con principios de alineación bio-mecánicos.
	\item \textbf{Ashtanga Vinyasa Yoga:} Desarrollado por T.Krishnamacharya y su estudiante Pattabhi Jois; acercamiento sistemático y secuencial a la práctica de los asanas donde las posturas son separadas en series. Vinyasa, una conexión energetica de un asana a otro, es usado para crear y mantener calor y un movimiento de estado meditativo.
	\item \textbf{Ashtanga Yoga:} Llamado así por Baba Hari Dass, despus del camino de ocho pasos de Patanjali, no confundirlo con Ashtanga Vinyasa Yoga.
	\item \textbf{Bikram Yoga:} Una secuencia de veintiseis posturas en un cuarto calentado a 100 grados Fahrenheit.
	\item \textbf{Iyengar Yoga:} B.K.S. Iyengar, otro de los estudiantes de Krishnamacharya, refinó el aprendizaje de su gurú despus de mudarse a Pune. Avandonó el estilo Vinyasa y se enfocó en las enseñanzas de la salud, alineación estructural y beneficios terapeuticos de las posturas.
	\item \textbf{Kundalini Yoga:} Despertar energía, Kundalini yoga llegó al oeste en 1969, cuando Sikh Yogi Bhajan desafió la tradición y comenzó a enzeñarlo públicamento. Esta práctica es designada para despertar la energía Kundalini, la cuál es almacenada en la base de la espina y en ocasiones representada como una serpiente enroscada. Kundalini mezcla cantos, prácticas de respiración y ejercicios de yoga. El enfasis no es en los asanas, sino en los cantos y respiraciones.
	\item \textbf{Mysore Style:} Nombrado así en honor a la ciudad en india donde Pattabhi Jois enseña el metodo Ashtanga Vinyasa; una práctica autoencaminada con supervición y ajustes físicos de un instructor.
	\item \textbf{Vijnana Yoga:} Una práctica de Hatha Yoga desarrollada por Donna Holleman y Orit Sen-Gupta, basada en siete \"principios vitales\" diseñados para usar el cuerpo para explorar los más profundos niveles de nuestro ser.
	\item \textbf{Viniyoga:} Esta forma gentil de flow yoga pone gran \'enfasis en la respicación y cordina respiración con movimiento. El movimiento fluido de Viniyoga o Vinyasa es similar a la dinámica de la serie de poses de Ashtanga, pero ejecutado con una gran reducción de paz y nivel de estr\'es. Las posturas y secuencias son elejidas para encajar en las habilidades del estudiante. Se enseña al estudiante cómo aplicar las herramientas de yoga: asana, cántos, pranayama (control de la respiración), y meditación, en una práctica individual. Desarrollado por T.K.V. Desikachar, el hijo de Krisnamacharya (maestro de algunos grandes instructores de yoga incluyendo Iyengar y Pattabhi Jois), Viniyoga pone menos estres en uniones y rodillas manteniendo las posturas con una ligera flexión en las rodillas. Viniyoga es considerado excelente para principiantes, y es incrementalmente usado en ambientes rerap\'euticos.
	\item \textbf{Yin Yoga:} Un t\'ermino creado por Paul Grilley para describir una forma de práctica con un \'enfasis en posturas mantenidas por mucho tiempo, usualmente sentados, boca abajo o boca arriba. Yin Yoga se enfoca en fortalecer y alargar el tejido conectivo, que en turno, a partir de líneas meridianas, tiene un efecto óptimo en el funcionamiento de los órganos.
\end{itemize}

\subsection{Viendo la Imagen Completa}
ga}


\section{Sencillez Voluntaria}
\subsection{Principios de Sencillez}
\subsection{Acercamiento a Sencillez}
\subsection{Un Acercamiento al Lado Financiero}

lakoae dwomcsdkjfcnerdIJERKLMDSCIOEJDM

\chapter{Técnicas de Entrenamiento y Práctica}
\section{Técnicas de Asana}
\subsection{Alineamientos Fundamentales}
El cuerpo humano tiene una alineación óptima. Cuando el cuerpo se mueve hacia o en un alineamiento óptimo, hay un incremento en la libertad de movimiento de las articulaciones y más energía disponible en el cuerpo, dado que el cuerpo no está peleando consigo mismo y le es pocible moverse libremente. El dolor es tambi\'en reducido o eliminado cuando los huesos y tejidos del cuerpo se encuentran en una relación cooperativa.

La habilidad para realizar una asana puede variar enormemente, dependiendo de facotes como la estructura osea, lesiones previas, edad, nivel de enería y empeño del estudiante. Comenzar una práctica de asana como disciplina de consciencia y no violencia (especialmente a los practicantes) es un comienzo fundamental para una práctica libre de dolor.


\subsection{Biomecánicas Holísticas}
\begin{itemize}
	\item Holistico - ``visto como un todo, integrar.''
	\item Biomecánico - ``el estudio y la aplicación de fuerzas físicas benficas en seres sensibles.''
	\item O: ``No uses el cuerpo en esa postura\ldots usa la postura para entrar en el cuerpo''
\end{itemize}

\subsubsection{Biomecánizas holísticas en pocas palabras:}
\begin{itemize}
	\item Reconoce que cada parte del cuerpo humano (cuerpo, mente y emociones) es involucrada en cualquier actividad, sea una postura de yoga, una discusión o comer una comida.
	\item Reconoce el potencial para la salud a trav\'ez de la aplicación apropiada de conciencia y forza la reación de relaciones armoniosas entre articulaciones, músculos, huesos, mente y emociones.
	\item Reconoce las limitaciones estructurales del cuerpo y trabaja dentro de esos límites.
	\item Reconoce que cada persona es diferente (músculos, huesos, mente y corazón). La aplicación de cualquier t\'ecnica debe ser adaptativa a las necesidades individuales de esa persona.
\end{itemize}

Una obstrucción primaria para los estudiantes de Hatha Yoga (yoga físico) es intentar interpretar el lenguaje de un maestro conforme dirige el movimiento de ciertas partes de tu cuerpo en formas específicas. El distinto acercamiento a las posturas de yoga puede dar algunas veces conflicto verbal de instrucciones a elos estudiantes. El siguiente trabajo intenta ayudarte como estudiante a entender el trabajo de tu cuerpo en acción.

Desde el trabajo del maestro T. Krishamacharya y otrs de 1930 en adelante, el potencial de curar al cuerpo y obtener un completo rango de mobilidad y vibración física a partir de la práctica de Hatha Yoga ha sido reconocida. El buen alineamiento y la aplicación de fuerza apropiada en el cuerpo físico es la llave. Sincronizando la mente con el cuerpo e incluso emociones es lo que yo llamo ``Biomecánicas holísticas''.

Las biomecánicas holísticas toman la perspectiva del yoga llamada ``Tantra'', y acepta que el cuerpo no es simplemente un vehículo inherte para un espíritu separado, sino un universo inteligente en miniatura. Fuerzas en nuestro universo interactúan en ciertas formas predecibles que son reflejadas en nuestro cuerpo, mente y emociones. La aparente separación de cuerpo, mente y el mundo exterior es una ilusión.

Los movimientos y direcciones de energía o prana en el cuerpo han sido descritos anteriormente en los Upanishads, una antigua compilación de sabiduría yoguica obtenida por varios autores. Muchas escuelas modernas de yoga utilizan antiguos entendimientos de las energías sutiles en el cuerpo, y las describen en diferentes formas.

La perspectiva oriental de las energías sutiles y de cómo el prana se mueve benficamente dentro de el cuerpo junto con algunos modernas (generalmente occidentales) entendimientos de las relaciones de músculos y huesos, conforman la biomecánica holística.

\subsubsection{Metodologílsa}
Cuando se realizan los estramientos terrenales que llamamos ``asana'', cualquier forma de nuestro cuerpo tiene el potencial de beneficiar la elasticidad y fuerza, o causar lesiones. Si no hacemos nada, sólo nos sentamos en el sofá, por ejemplo, hay poca probabilidad de algún daño inmediato en el cuerpo.

Existe tambin una posibilidad muy alta de eventualmente perder salud debido al atrofiamiento de tejidos del cuerpo, y correspondiendo contraindicaciones de la mente y el cuerpo emocional. La salud se encuentra entre hacer nada y hacer mucho. Cuando el cuerpo se mueve con habilidad y armonía, la intensidad del ejercicio puede ser incrementado con seguridad y los beneficios incrementan tambin. Un ejercicio hecho sin cuidado tiene más alto potencial de lesiones y menos potencial de beneficio, poque mejoramosen las cosas que practicamos.

Las siguientes decuencias de acciones son una síntesis, tomada del trabajo de maestros de yoga, fisicoterapeutas occidentales, observación de animales y práctica personal. Estas tcnicas no pertenecen a nadie, son parte de nuestra cultura e intelectual común. Cómo puedes saber si la t\'ecnica de un fisioterapeuta está funcionando? Se siente bien. Ese reconocimiento es innato, como el deseo de estirarse y descansar, ese conocimiento es intrínseco en todos los seres. (Opuesto de sedentario y sobre-activo)

\subsubsection{Samasasthiti (pronunciado, sama stee tee hee)}
Samasthiti es la palabra en sánscrito que significa ``esparcir la luz de la consciencia a travs del cuerpo.'' Samasthiti es el estado de consciencia dentro de la postura inicial de levantarse listo frente al tapete, y en cualqueier postura, demostrar un estado que permite la concentración luminosa y compasiva.

El resultado emocional e intelectual de esta ación es una receptividad, físicamente una falta de armonía exterior y sin embargo una fuerte base de las partes de tu cuerpo que tocan la tierra.

\subsubsection{Integración}
Como "condensaci\'on de consciencia", el practicante dirije los tejidos del cuerpo como uno hacia el centro, localizado a lo largo del eje de la columna vertebral, entre los hombros y las caderas.

Los hombros y caderas se mueven hacia atras en su "hogar" estructural:
\begin{itemize}
	\item los muslos se mueven hacia los izquiotiviales, suavizando el frente de las ingles; y
	\item los hombros descanzan c\'omoda y fuertemente hacia la parte trasera del cuerpo.
\end{itemize}
(Tadasana de nuevo, energ\'ia moviendose hacia adentro)

\subsubsection{Expansi\'on}
Una vez que la integración se ha llevado a cabo, la energia sublime del cuerpo que se encuentra en distintas formas, se basa en previa integración muscular, fluye del centro del cuerpo a los ejes de la espina, desde abajo de las piernas como las raíces de un árbol, hasta la cima de la espina, brazos y cabeza, como ramas de un árbol.

(energía movi\'endose hacia afuera)

En pocas palabras, la práctica de esas tres acciones: Suavizar, Flexionar y Estirar. Estas acciones pueden ser realizadas en cualquier postura, posición o actividad dentro de una práctica formal de yoga o en actividades diarias incluso, como lavar los platos.

Los beneficios de una fuerza incrementada, un mayor rango de movilidad y una sensación de serenidad aparecen porque el cuerpo, la mente y las emociones son conducidad en la fábrica de nuestro mundo, de nuestro universo. No estamos separados de todo lo que vemos, así que cuando reconozcamos nuetra conección fundamental, y nos comportemos como todas las cosas en el universo lo hacen (pulsando con expansión y contracción) entonces encontraremos esta relación armoniosa. No una relación de inactividad o separación, sino una de participación con la vida.

\subsubsection{Dolor contra intensidad:}
Para un practicante principiante, la sensación experimentada durante una práctica de asana es con frecuencia desconocida. Con más experiencia, una diferenciación puede ser encontrada entre dolor e intensidad. Dolor repentino, sensaciones desagradables, especialmente alrededor de las articulaciones no debe ser ignorado. El cuerpo envía una señal de desalineación o desconexión que podría ser dañina.

Sensaciones de estiramientos intensos a un músculo puede ser interpretado como dolor, pero la sensación es muy distinta. Con frecuencia un practicante se encuentra en control de la cantidad de sensaciones experimentadas, como en una flexión sentado hacia el frente. Estas sensaciones de intensidad son intrínsecamente parte de la práctica. Generalmente respirar en las posturas intensas recoge la mente del cuerpo y la resistencia decrementa.

\subsection{Terapia Estructural}

La terapia estructural es la aplicación consciente de las holisticas biomecanicas de un practicante a su cliente, con una completa participación del cliente. Tocar con vigor establece resonancia entre el practicante y el cliente, antes y durante los ajustes realizados. Con práctica, el practicante aprende a diferenciar entre tipos de resistencia: compresión, tensión, músculo y tejido conectivo. Junto con la resistencia física, pueden ser encontrados patrones de resistencia en los cuerpos emocionales y mentales.

La realización de ajustes será de poco valor a largo plazo a menos que se ilustre al cliente ómo realizar una buena alineación, con apropiada integración y expansión. La idea es que el cliente gane consciencia kinestésica de cómo crear alineación curativa para ellos mismos.

\subsubsection{Qué tanta presión?}
Cuando se aplica un ajuste, primero debes conectar con la parte del cuerpo que vas a ajustar con seguridad, incluso presión. Moverás piel, músculo y hueso como una unidad. Poca presión será inefectiva. El ajuste debe comenzar desde la más baja intensidad e ir incrementando hasta el máximo en un período de 3 o 4 segundos, dl ajuste mismo puede durar desde 5 segundos hasta 1 minuto.

Mantén tu atención en la cara del cliente buscando signos de inconformidad, dolor o alivio. Cuando trabajes con piernas o cadera, tus manos y brazos son usualmente menos fuertes que con lo que estás trabajando. Cada cuerpo es distinto y la sensitividad o sensación es distinta de persona a persona. Mantente comunicado con tu cliente, revisando si está experimentando dolor o alivio.

\subsubsection{Areas clave de imbalance}

Desalineación ocurre donde los huesos se encuentran. La desalineación habitual ocurre en el estilo de vida: sentado por largos períodos, movimientos repetitivos que crean imbalance muscular, y posiblemente nuestra forma de estar erguidos y caminar como tales.

Es posible que desde una perspectiva evolutiva, aún nos encontremos en evolución física y que no estemos por completo adaptados a una posición erguida. La parte superior de nuestro fémur encaja mejor en el hueco de la cadera en una posición inclinada en lugar de en una orientación vertical, la parte exterior de nuestras piernas se encuentran mas rígidas cuando nos encontramos de pie igualmente. Nuestros muslos interiores son difíciles de involucrar muscularmente para rebalancear la rotación común del femur debido a la rigidez de las piernas exteriores.

Los grupos musculares involucrados en llevar los homóplatos hacia la línea central del cuerpo incluyen a los romboides. Esos músculos son bastante mas complicados de involucrar que otros. Es posible que estos grupos musculares se hayan desarrollado relativamente recientemente en nuestra evolución y aun no se encuentren mapeados del todo en nuestro cerebro. Ese desequilibrio potencial que se manifiesta cerca del eje de nuestro cuerpo puede crear más inestabilidad que la periferia, incluyendo rodillas y muñcas.

\textbf{Compresión y tensión}
Existen dos fuerzas físicas que limitan el rango de movimiento del cuerpo; estas fuerzas limitantes incluyen tensión en el tejido del cuerpo (músculo y tejido conectivo) y compresión. La tensión es fácil de reconocer; es la parte del "estiramiento" en el hatha yoga. Compresión, por otro lado, ocurre cuando dos huesos se encuentran. Un ejemplo sencillo es tu codo. Si lo extiendes por completo, dos huesos se unirán, no importa que tan flexible seas, tu rango de movilidad no va a incrementar. Al aplicar la fuerza apropiada a un hueso, su densidad incrementará, por lo que algo de compresión es buena si el objetivo es estimular la densidad ósea. Intentar forzar cualquier articulación mas alla de su compresión para lograr alguna forma exterior del cuerpo puede causar dolor o lesiones.

Los huesos de cada persona son diferentes en longitud, forma y densida. Los huesos pueden tener rotaciones dentro de sus cabidades que hagan que, en el caso del fémur, un pie puede rotar hacia adentro más hacia adentro o hacia afuera que tro, mientras que la cabeza del fémur se encuentra alineada. La limitación eventual de la profundidad de cualquier postura de yoga será la forma y tamaño de los huesos Los huesos pueden incrementar o decrementar en densidad, adaptandose al estres que se ponga en ellos, pero los huesos adultos no pueden cambiar en tamaño o forma hasta donde sabemos.

\subsubsection{Hombros - Anatomía y función}
El grupo de los hombros se conforma por la clavícula, escápula y huesos húmeros que contienen también muchos músculos dentro y alrededor de ellos. Lo mas sencillo es verlos como una unidad funcional. El diseño del grupo de los hombros nos permite un mayor rango de movilodad a costa de un poco de estabilidad, comparado con la cadera por ejemplo. El único lugar de conexión ósea es la unión esterno-clavicular, localizada al frente del cuerpo. Si pudieras desabrochar esta unión, serías capáz de casi quitarte los hombros como un abrigo.

La mejor colocación para los hombros es en la parte tasera del cuerpo. Cuando lso hombros se encuentran hacia atrás:
\begin{itemize}
	\item los homóplatos se deslizan hacia atrás sin un "aleteo" significativo;
	\item las clavículas son casi invisibles en el frente del cuerpo; y
	\item los brazos cuelgan de los hombros con las palmas ligeramente rotadas al frente
\end{itemize}

\subsubsection{Cadera/Espalda baja - Anatomía y función}
La cadera son una unidad funcional que incluye los huesos pélvicos, el fémur y la parte mas baja de la columna, la cual se encuentra colocada entre los huesos pélvicos. Esta area sostiene el peso de la parte superior del cuerpo y lo transfiere abajo hacia las piernas, tal como un puente de pidra arqueado soporta el peso de arriba. El rango de movilidad de el hueco de la cadera se encuentra limitado por la profundidad y tamño de esta misma, su forma y el ángulo del fémur, y posiblemente la tensión articular.

La posición mas ventajosa para la cadera y espalda baja es cuando nos encontramos parados con los pies paralelos y hombros recogidos hacia atrás del cuerpo. Esto refuerza la curva natural de la espina y ayuda a enraizar la parte baja del cuerpo.

La desalineación mas común en las caderas y la espalda baja es una falta de curvatura en la base de la espina. La planicie crea presión en los nervios entre las vértebras y decrementa el rango de movilidad. Esta planicie se encuentra también relacionada a la rotación externa del fémur y a bajar la parte posterior de nuestros piés, lo cuál se convierte a su vez en mas presión a la espalda baja.

Por el caso contrario, demasiada curvatura hacia adentro, y no suficiente alargamiento en la columna, puede a su vez crear dolor debido a la presión en las nervios en las vértebras. Una curvatura excesiva en la espalda baja se encuentra relacionada con la rotación interna del fémur y un exceso de peso en las partes internas de los pies, rotando las rodillas hacia adentro.

\subsubsection{Muñecas - Anatomía y función}
La articulación de las muñecas incluye los dos huesos de los brazos y los pequeños e irregulares huesos de las manos llamados "carpelos". La compleja construcción de las muñecas nos da una increible destreza y rango de movilidad en los dedos.

Una buena alineación de la muñecas es importante cuando se carga peso en las manos. Cuando nos encontramos en cuatro puntos, se colocan las manos aproximadamente a la distancia de los hombros con los dedos cómodamente separados y los dobleces de las muñecas paralelas al frente de tu tapete de yoga. Muscularmente involucrar las manos creara un hueco en la palma y proveerá soporte estructural.

Dado que utilizamos nuestras manos constantemente, las muñecas son suceptibles a una tensión constante lo que inflama los huecos de los carpelos. Esta tensión repetitiva ocurre seguido en trabajos donde el mismo movimiento de manos y brazos es realizado una y otra vez, como los cajeros de las tiendas, por ejemplo.

\subsubsection{Alineación del cuerpo exterior:}
Esto se refiere a la estructura básice de alineación del cuerpo dada una postura. Las articulaciones principales del cuerpo (tobillos, rodillas, cadera, hombros, codos, muñecas y cuello) son lugares de gran capacidad de movilidad, y por lo tanto los que tienen mas a una mala alineación. En una práctica de asanas, el cuerpo toma varios formas. Físicamente, los tejidos del cuerpo deben fortalecerse y alargarse para adaptarse a esas formas.

La alineación de una parte del cuerpo (rodillas, por ejemplo) afecta a la alineación de el resto de las partes, tal como una base dispareja en una casa causaría un primer piso inestable, tercer piso inestable y así susecivamente. Los ajustes a nuestro cuerpo exterior en una postura naturalmente comienzan en la base y se mueve secuencialmente hacia arriba.

\subsubsection{Preparando la base:}
En cualquier postura, existe una base, o esa parte del cuerpo que conecta a la tierra. Nos encontramos dentro de un campo gravitacional, una fuerza constante que jala el cuerpo hacia abajo y provee la estabilidad con la cuál nos levantamos, física y energéticamente. La práctica de asanas no sería benéfica realmente sin gravedad. La gravedad enraiza al cuerpo, otorgando la resistencia que crea fuerza en los tejidos musculares, Esto es comprensible, condensando la fuerza con la que nos expandemos.

\subsubsection{Alineando los pies:}
Existen cuatro esquinas para cada pie:
\begin{enumerate}
	\item el montículo del dedo gordo
	\item la parte trasera del talón interior
	\item el montículo del dedo pequeño
	\item la parte trasera del talón exterior.
\end{enumerate}

Para crear conexión con la tierra a través de los pies se debe seguir esta secuencia.

En Tadasana (postura de la montaña) dibuja una línea imaginaria del centro del tobillo a la mitad del segundo dedo del pie. Crea esta línea en cada pie paralelo con el otro. Los pies deben estar separados a la distancia de los huesos exteriores de la cadera.

\textbf{Alineando las manos:}
Cuando las manos son parte de la base, como en perro boca abajo o chaturanga dandasana, y la mayoría de los balances en brazos; las manos igual deben estar alineadas para estar conectadas.
\begin{enumerate}
	\item el montículo del dedo índice;
	\item la parte interior de la base de la muñeca;
	\item el montículo del dedo meñique;
	\item la parte exterior de la base de la muñeca;
\end{enumerate}

Separa y enraiza los dedos y acerca la palma a la ierra, creando una forma de domo en el centro de la palma.

Cuando las manos son parte de la base, las crestas de las muñecas se alínean mutuamente y con la parte frontal. Si te encuentras mirando hacia abajo en el tapete (como en perro boca abajo) las crestas de las muñecas deben estar paralelas al borde frontal de tu tapete. Las manos deben ser colocadas a la distancia de los hombros.

\subsubsection{Alineando las rodillas:}
En Samasthiti (Tadasana), la parte superior de las piernas (fémur) y los huesos de bajos e encuentran verticales y en l ínea el uno con el otro. Las rodillas no se encuentran dobladas o hiper extendidas (recogidas hacia atras mas allá de una línea recta. Las cuatro esquinas de cada rótula se encuentran cuadradas. Los lados de las rótulas se encuentran verticales. En posturas de pie, la rodilla (si se encuentra flexionada en una postura como Parsvakonasana y Virabahdrasana 1 y 2) va directamente sobre el tobillo, y no más allá.

\subsubsection{Alineando los hombros:}
El grupoos de los hombros (clavículas, los huesos superior de los brazos y homóplatos) deben ser colocados tal que los homóplatos caigan hacia atrás del cuerpo. Para lograr esto, parados en Tadasana, alarga los lados del torso hacia arriba, eleva y rota hacia atrás los hombros, relajando los brazos a los lados del cuerpo.

\subsubsection{Alineando la cabeza y el cuello:}
En la cabeza existe mas masa en la parte frontal de la columna vertebral que en la parte trasera. Esto es evidente si es observado desde un lado del cuerpo. Para muchos de nosotros, la cabeza se coloca hacia el frente de su posición anatómicamente neutral, creando un acortamiento y tensión en los hombros y el cuello. Para liberar esta tensión habitual y traer la cabeza a una posición neutral arriba de la columna vertebral, mueve la cima de la garganta (donde el cuello se encuentra con la cabez) hacia atrás y hacia arriba. Esto puede ser generado al alargar la parte trasera de la cabeza hacia arriba. Una referencia de una posición neutral es observada cuando los párpados superior e inferior  se encuentran en el mismo plano vertical (visto desde un lado).

\subsubsection{Etudio de alineación: perro boca abajo}
Para crear la forma exterior del perro boca abajo, primero encuentra la distancia óptima de tu cuerpo de las manos a los pies:
\begin{itemize}
	\item Comienza acostandote boca abajo (en tu estómago).
	\item Coloca tus manos bajo tus hombros.
	\item Coloca los dedos de tus pies de tal manera que las huellas digitales se encuentren bajo tus talones.
	\item Sin cambiar la colocación de pies o manos, entra en una posición arrodillada y luego empuja las rodilals hacia perro boca abajo.
\end{itemize}

Esto te otorgará una idea de la separación aproximada de tu postura basado en tu altura. Anchura de tu fundación:
\begin{itemize}
	\itemize Las manos se encuentran a la distancia de los hombros.
	\itemize Los pies se encuentran a la distancia de los huesos de la cadera.
	\itemize La cabeza tiene permitido colgar naturalmente, y los ojos se encuentran suavizados.

Lo anterior etablece la forma exterior de la postura.
\end{itemize}

\subsection{Técnicas de Asana: Categorías de Posturas}
Las posturas son clasificadas dependiendo de su función:

\begin{itemize}
	\item Posturas de pie
	\item Arcos
	\item Flexiones frontales
	\item Torciones
	\item Inversiones
	\item Aperturas de cadera
	\item Restaurativas
	\item Balance de brazos
\end{itemize}

\subsubsection{Posturas de pie:}
Esas posturas son esencialmente para desarrollar consciencia corporal, fuersa muscular (especialmente en las piernas) y equilibrio. El cuerpo entero es afectado fuertemente por la fuerza de gravedad y por lo tanto tiene que trabajar enormemente. La circulación es estimulada al igual que el flujo balanceado del prana. Para encontrar una anchura apropiada (distancia entre los pies), es útul comenzar con un desplante, usando esa colocación óptima de los pies como una plantilla para generar las posturas de pie

\subsubsection{Arcos:}
Los arcos abren la parte frontal del cuerpo. La gravedad y la cerradura abitual de la parte frontal del cuerpo debido a la postura puede encorvar el cuerpo y cerrarlo hacia el frente, tanto física como emocionalmente. Arquear empodera al sistema nervioso y puede alludar a liberar emociones guardadas. Dado que esas posturas avivan al sistema nervioso, se debe considerar realizar arcos en el dia dado que su practica puede causar insomnio si se realizan en la noche.

\subsubsection{Flexiones frontales:}
Estas posturas estiran la parte trasera del cuerpo, cerrando el frente donde se encuentran nuestros órganos de percepción. Los efectos son generalmente más pasivos, el sistema nervioso se tranquiliza. Para realizar una flexión frontal, la espalda baja debe encontrarse ligeramente cóncava y la columna extendida, enviando la pelvis hacia el frente, antes de arrojar el torso hacia el frente. Puede ser necesario que los estudiantes se sienten en un block o una cobija para lograr esto. En caso de estar de pie, se deben mantener las piernas firmes y flexionar ligeramente, con las manos apoyadas en las piernas.

\subsubsection{Torsiones:}
Las torsiones invitan a los órganos internos a desintoxicarse y ayudan a energizar y balancear al sistema nervioso a la vez. Para ser más provechosas, una parte de la torcion debe mantenerse estable mientras la otra parte rota, articulando las vertebras en la columna espinal. En ocasiones la parte mas movil del cuerpo (el cuello) se moverá inconscientemente antes que las partes menos móciles de la columna. Mueve el torso a ambos lados equitativamente en toda torsión, y realiza la profundidad de la postura desde el pecho. Mantén una base firme  durante la torsión

\subsubsection{Inversiones:}
Estar boca abajo literalmente, cambia tu punto de vista. Las inversiones estimulan la circulación al permitir que la gravedd invierta el flujo de la sangre. Si un estudiante tiene alta presión sanguínea, las inversiones como el parado de manos debe ser realizado con cuidado. Existe generalmente un miedo relacionado a encontrarse boca abajo dado que esto desorienta en un inicio. Se debe incentivar a los alumnos a moverse lentamente a posturas como el parado de manos y evitar patear hacia arriba o golpear con la pared.

\subsubsection{Aperturas de cadera}
Las aperturas de cadera incentiban al chakra raíz a funcionar y puede liberar dolor en la espalda baja y desalineación en las piernas. Debido a nuestra preferencia de sentarnos en sillas, los músculos y tejidos conectores de las caderas se endurecen con el tiempo y limitan el rango de movilidad. Esto junto con músculos abdominales débiles (de nuevo, la espalda baja sosteniendo el peso no ayuda a la participación del área abdominal) crean una situación donde se vuelve dificil para muchos adultos incluso sentarse cómodamente en el suelo. Para permitir la apertura de la parte frontal y trasera de la cadera, la clave es balancear la fuerza y la flexibilidad. Las posturas de pie como Virabhadrasana 1 y 2 al igual que Parsvakonasana proveen un camino accesible para la apertura de cadera. Estudiantes nuevos o muy rígidos pueden encontrar poturas las posturas tradicionales sentadas como Rajakapotasana extremadamente retadoras y difíciles de realizar. La estructura ósea en el área pélvica (cabezas del fémur, longitud del trocánter mayor y la forma del acetábulo) pueden variar enormemente de estudiante a estudiante, permitiendo a algunos tener mayor libertad de movimiento que otros.

\subsubsection{Restaurativas:}
Estas posturas son por naturaleza diseñadas para relajar y restaurar energía. Savasana es por exelencia la postura restaurativa, en la que la verdader relajación es el verdadero (una liberación de tensión muscular y una respiración controlada mientras se permanece consciente. El cuerpo tiene una habilidad innata para curarse a si mismo. Las posturas restaurativas permiten un mejor flujo de prana en las áreas enfocadas. Debido al total de la relajación necesitada, las posturas restaurativas deben ser principalmente supinas (sobre la espalda) y posiblemente ayudadas por props.

\subsubsection{Balances de brazos:}
Requieren coraje, fuerza y stamina, los balances de brazos son probablemente la clase de posturas mas vigorizantes. Debido a su intensidad, solamente pueden ser realizadas normalmente por "fuerza bruta". Un punto importante a recordar es que el cuerpo entero se encuentra activo en un balance de brazos, brindando estabilidad y distribuyendo el trabajo a lo largo del cuerpo, no solamente sobre los brazos.

\subsection{Forma y Acción}
Cuando observa una postura, la figura general o forma de la postura es lo que aparenta al inicio. Dentro de la forma de cualquier postura, existen también acciones internas del cuerpo (los músculos, huesos, respiración y movimiento del prana. La forma exterior y la forma interior en alguna postura pueden ser diferentes e incluso opuestas, creando un balance.

La práctica de asana es la práctica de reunir partes disjuntas del cuerpo para lograr un fin común. Nosotros usamos la forma y acción unidas para ayudar a lograr esto.

Tomando una postura básica de desplante por ejemplo, la forma exterior es clara. Sin embargo, lo que puede no ser visible es la estabilización alcanzada como resultado de aplicar energía muscular de forma equitativa en toda la pierna, centrando el fémur (hueso superior de la pierna) en el hueco de la cadera. Si la forma exterior fuera realizada sin acción muscular, al mantener el fémur de la pierna trasera hacia arriba, la fuerza de gravedad comenzaría eventualmente a mover el fémur hacia el suelo y hacia los músculos de los cuadriceps, por lo tanto, lejos de su lugar óptimo.

El cuerpo, al sentir desalineación, tiene varios trucos para analizar y ptorejer el cuerpo de lesiones. Es posible que ocurra una contracción, o calambre de los músculos alrededor del fémur asociado con incomodidad y rango de movilidad limitado, exactamente lo opuesto del objetio de realizar una postura en primer lugar.


éáíñúó

\subsection{Navegando el Tapete}
\subsection{Sacro}
\subsection{La Práctica de Asana}
\subsection{Yin y Yang}
\subsection{Polaridades de la Energía Física}
\subsection{Fuerzas Opuestas}
\section{Téecnicas de Purificación}
\section{Meditación}
\section{Pranayama}
\subsection{Nadi Shodhana}































\chapter{Anatomía y Psicología}
\section{Chakras}
Un chakra es el centro de actividad que recive, procesa y expresa energía de fuerzas vitales o prana. La palabra sánscrita chakra se traduce como ``rueda`` o ''disco`` y se refiere a una esfeza giratoria de bio energía. Hay, en este modelo particular, siete chakras posisionados en una columna de energía de la base de la espina a la coronilla de la cabeza. Los siete chakras mayores que se correlacionan con los estados básicos de consciencia. Como transformadores de energía, ellos bajan de la energía universal de consciencia al plano físico. De esta forma nosotros estamos conectados a la fuente de energía, se encuentra disponible para nosotros en diferentes formas de energía. Similar a los conectores elctricos, diferentes formas de energía son apropiadas para diferentes usuarios.\\
Los colores asociados con los chakras son tambi\'en una forma de energía. Color es energía expresada como una onda de luz que podemos ver (existen ondas de luz que no podemos ver con nuestros ojos). Existen tambi\'en sonidos correspondientes, asociados a cada chakra. De nuevo, el sonido es solo otra forma de energía vibrante a distintas frecuencias.\\

\subsection{Primer Chakra: Muladhara}
Tierra, identidad física, orientado a autopreservación. Color rojo. Localizado en la base p\'elvica. Este Chakra forma nuestra base. Está relacionado a nuestos instintos de supervivencia y a nuestro sentido de apego y conexión a nuestros cuerpos y al plano físico. Idealmente este chakra nos otorga salud, prosperidad, seguridad y presencia dinámica.\\
\subsection{Segundo Chakra: Svadhisthana}
Agua, identidad emocional, orientado a autogratificación. Color naranja. Colocado en el área del sacro. Este chakra está relacionado al elemento agua, y a las emociones y sexualidad. Se conecta a nosotros a trav\'es de los sentimientos, deseos, sensaciones y movimiento. Idealmente este chakra nos brinda fluidez y gracia, profundidad en los sentimientos, realización sexual y la habilidad de aceptar el cambio.
\subsection{Tercer Chakra: Manipura}
Fuego, identidad individual, orientacion a autodefinición. Color amarillo. Localizado en el plexo solar. Comanda nuestro poder personal, deseo, autonomía y metabolismo. Cuando se encuentra saludable, este chakra otorga energía, eficiencia, esponaneidad, y poder no-dominante.
\subsection{Cuarto Chakra: Anahata}
Aire, identidad social, orientado a autoaceptación. Color verde. Localizado en el corazón. Está relacionado a la verdadera compasiń y es el integrador de opuestos: izquierdo y derecho, arriba y abajo, hombre y mujer, expansión y contracción. Un cuarto chakra saludable nos permite amar profundamente, sentir empatía y tener una profunda sensación de paz y concentración.
\subsection{Quitno Chakra: Vishudha}
Sonido, identidad creativa, orientado a la autoexpresión. Color azul. Este chakra se encuentra localizado en la garganta y por lo tanto está relacionado con la comunicación y creatividad. Aquí nosotros experimentamos el mundo simbólicamente a trav\'es de vibraciones, tales como la vibración del sonido representando lenguaje.
\subsection{Sexto Chakra: Ajna}
Luz, identidad de estereotipo, orientado a autoreflexión. Localizado en el entrecejo (tercer ojo). Color índigo. Está relacionado al acto de ver, tanto física como intuitivamente. Como tal abre nuestras facultades psíquicas. Cuando se encuentra saludable nos permite ver claramente, completamente, permitiendonos \"ver la imagen completa\".
\subsection{S\'eptimo Chakra: Sahasrara}
Pensamiento, identidad universal, orientado al autoconocimiento. Color violeta. Localizado en la coronilla de la cabeza. Este chakra se relaciona con la consciencia como consciencia pura. Es nuestra conexión con la Consciencia pura al nivel universal. Cuando se desarrolla, este chakra nos otorga conocimiento, sabiduría, entendimiento, conexión espiritual y dicha.
\\
Una práctica apropiada de asanas puede ayudarnos a balancear estas energías sutiles del cuerpo. Los chakras se balancean por medio de unir las energías de Siva (consciencia) y Shakri (creción). Cuando se encuentran balanceados, cada chakra funciona óptimamente, dándonos acceso espontáneo a todas las formas de energía corporales. La meditación de chakra es una excelente forma de mejorar tu entendimiento de esos centros, al igual que una dieta propia y elecciones de estilo de vida.

\section{Los Vayus}
El entendimiento yoguico del cuerpo es experienciado en lugar de teótico. El entendimiento fundamental es que el cuerpo es una expresión de la fuenta universal, como una onda es una expresión del oc\'eano.  Dentro de esta expresión universal hay formas en que la energía vital fluye, como corrientes en un cuerpo de agua. Los yoguis dieron nombres a estas corrientes, y varias escuelas de yoga suelen nombrar a esas expresiones de energía de diferentes formas. La fuerza esencial del cuerpo es conocida como prana, la menor unidad de fuerza vital.\\
Prana y la respiración se encuentran intimamente unidos. Prana mueve la respiración. Sin fuerza vital, no hay respiración, no hay otra forma. Podemos interactuar de cierta forma con esta fuerza vital conocida como prana al sentir y manipular la respiración (incluso deteniendo la respiración por un período de tiempo). Dentro de un ciclo respiratorio, prana se vuelve perceptible.\\
Los cinco Vayus principales:
\begin{itemize}
	\item Prana - la asendencia del flujo de energía, que puede ser sentido en la inalación.
	\item Apana - la desendencia del flujo de energía, percibido en la exalación.
	\item Samana - la corriente de energía que se digire, se representa hacia nuestro centro.
	\item Udana - la corriente de energía que se consume conforme se expande a las extremidades desde nuestro centro.
	\item Yyana - la corriente integradora de energía que mantiene el equilibrio.
\end {itemize}
Conforme un principiante mueve el cuerpo en una práctica de asanas, es en un inicio usualmente una experiencia ``corporal exterior''. Posturas de formas básicas, sentimientos de rigidez o fatiga en partes del cuerpo son notables. Conforme la práctica continúa, se hacen conscientes  sensaciones más sublimes. Es aquí cuando el trabajo con los vayus puede comenzar.\\
Prana puede ser sentido como una fuerza que sube en la inalación cuando los brazos son levantados sobre la cabeza.\\
Apana puede ser sentido hacia abajo en la exalación cuando los brazos se colocan hacia los lados del cuerpo.\\
Samana puede sentirse como una fuerza integradora, dibujando flechas en el cuerpo indicando al centro.\\
Udana puede sentirse como una expresión sublime de expansión, o hacia afuera muviendo la energía desde el centro del cuerpo.\\
Vyana puede ser experimentado al esparcir consciencia atrav\'es del cuerpo, notando cómo diferentes partes pueden comunicarse y son mantenidas juntas.
\section{Compresión y Tensión}
\section{Anatomía Funcional}
\subsection{Huesos y Articulaciones}
\subsection{La Espina}
\subsection{Muscúlos}
\subsection{Muscúlos y Posturas}
\subsection{Yoga y Posturas}
\newpage
\section{Los Bandhas}
Bandha significa \lq\lq candado\rq\rq. Este tipo de candado, en lugar de cerrar, como el tipo de candado donde se requiere una llave para abrir, fue de hecho un termino figurado. Estos candados son como una zanja usada para dirigir el agua a diferentes partes de un campo. Bandhas en el cuerpo son usados para dirigir la energía tanto física como energeticamente. Físicamente, los bandhas funcionan para mantener estimulados y armonizados nuestros órganos internos. Energeticamente, asisten al movimiento del prana, o energía en el cuerpo. Hay tres tipos de bandhas usados en la práctica de las asanas:

\subsection{Mulabandha}
Localizado entre el ano y los genitales, es el músculo perineo para los hombres. Para las mujeres está ubicado cerca del límite superior del cuello del útero. La activación del Mulabandha no es una fuerte contracción forzando los músculos que lo rodean, es mas sutil que eso. Mulabandha puede ser experimentado activando los músculos hacia atras, incrementando la curvatura lumbar en la espina, despus permitiendo al coxis alargarse, estimulando al abdomen a activarse y la base de la pelvis a levantarse.

Activar los muslos hacia atras acomoda las cabezas del femur hacia atras y crea una expansión en el área pélvica.
Bajar el sacro reafirma la carne de los glúteos. El abdomen bajo se mueve de la pubis al ombligo.

La sinergía creada por esos dos complementarias, y a la vez opuestas fuerzas, crean Mulabandha. En lugar de endurecer o relajar en el área pélvica, un levantamiento es creado similar a lo que sería el último medio centímetro de una malteada por un popote.

\subsection{Uddiyana Bandha}
Localizado ligeramente abajo del ombligo, Uddiyana Bandha significa \lq\lq Volar hacia arriba\rq\rq refiriendose a su efecto en el prana. Este segundo banda es activado en una forma parecida a Mulbandha, con un mínimo de endurecimiento externo o contracción. En el proceso de crear este candado, el centro del plexo solar es llevado adentro y hacia arriba en un levantamiento abdominal y es cuando se encuentra la activación. En expresión completa es llevado acabo exalando completamente y luego llevando el abdomen bajo hacia adentro y arriba, mientras se eleva el diafragma. Este nivel de Uddiyana Bandha será usdo en la práctica de la retención de exalación en Prnayama, pero debido a la incapacidad de inalar mientras se ejecuta este nivel, simplemente mantener tranquilidad cerca de tres dedos debajo del ombligo otorga espacio para que el diafragma baje durante cada exalación. Conforme el diafragma baja, la respiración es impulsada a moverse hacia las costillas, espalda y pecho. En cada exalación los músculos abdominales promueven a completar el vaciado de los plmones. El proceso toma práctica, y las sutilezas de la relación entre respiración y bandhas debe ser explorada experimentalmente.


\subsection{Jalandhara Bandha}
Este candado es creado levantando y girando los hombros  hacia atras para primero ampliar y luego levantar el pecho. Despues la parte trasera de la cabeza se extiende hacia el cielo y la barbilla se mueve en una contracción, la cuál es formada donde dos huesos de la clavícula se encuentran. El candado ocurre espontáneamente en algunas posturas como el parado de hombros, pero no es usado tan ampliamente como los otros dos candados.

\section{La Respiración}
\subsection{Respiración Ujjayi}
\section{Elementos de la Naturaleza}
\subsection{Cualidades Caracteristicas de los Cinco Elementos}
\subsection{Ayurveda}
\section{Los Cinco Koshas}



\chapter{Metodología de Enseñanza}
\section{Enseñando a dirigir}
\section{Agregando Contenido}
\subsection{Nivel 1 - Respiración}
\subsection{Nivel 2 - Movimiento del Cuerpo Exterior}
\subsection{Nivel 3 - Alineamiento Fisico/Movimiento Energético}
\subsection{Nivel 4 - Incorporando Intención}
\subsection{Enseñando lo que Observas}
\subsection{Temporizando una Clase}
\subsection{Saludando/Centrando}
\subsection{Secuencia}
\subsection{Trabajo en Equipo}
\section{Tematizando}
\subsection{Tematizando a una Postura Especifíca}
\section{Organización del Salón de Clase}
\subsection{Líneas Visuales}
\subsection{Diseño de Clase}
\subsection{Ofreciendo Props}
\section{Demostración}
\subsection{Demostración Silenciosa}
\section{Cuestiones de Salud}
\subsection{Estudiantes Lesionados}
\subsection{Usando Props}
\subsection{Cuidados Específicos de Salud}
\subsection{Yoga para Artritis}
\subsection{Fibromalgia}
\section{Lenguaje}
\subsection{Tono de Voz}
\subsection{Que Tanto Decir?}
\subsection{Se Conciso}
\subsection{Volumen/Contenido}
\section{Modificación de Postura}
\section{Observación: Estudiante Individual}
\subsection{Fundación}
\subsection{Estado de Animo General}
\section{El Rol del Maestro}
\section{Fundamentos de Secuenciado}
\subsection{Secuenciado Variable y Establecido}
\subsection{Principios de Secuenciado}
\subsection{Entretenimiento - Centrando la Clase}
\subsection{Secuenciando la Clase a Nivel Mixto}
\section{Creando Intención}
\subsection{La Intención de "Liberar Tensión"}
\section{La Práctica y el Servicio de Enseñar Yoga}



\chapter{Práctico}
\section{Tarea}
\subsection{Clases en Secuencia}
\subsection{Desarrollar Intención para Clases}
\subsection{Autoevaluación}



\include{./tex/6_sanscrito}
\include{./tex/7_ejemplo}
\include{./tex/8_ilustraciones}

\end{document}
