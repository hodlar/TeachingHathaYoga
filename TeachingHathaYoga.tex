%	First file
\documentclass[a4paper]{article}

\usepackage{mathtools}
\usepackage[]{graphicx}
\usepackage{booktabs}

\title{Enseñando Hatha Yoga}
\date{2016-01-27}
\author{Daniel Clement with Naomi Clement}

\begin{document}
\pagenumbering{gobble}
\maketitle

\newpage
Copyright c 2007 Daniel Clement
Todos los derechos reservados. Sin limitaciones de derechos bajo 
derechos de autor, ninguna parte de esta publicación debe ser 
reproducida, almacenada o introducida en un sistema de 
retribución, o transmitida, en cualquier forma o por cualquier 
medio (electrónico, mecánico, fotocopiado, grabado o de otra 
forma), sin el previo consentimiento escrito del propietario de 
los derechos, excepto para cortas reseñas.
Primer impresión Octubre 2007, segunda impresión 2008, tercer 
impresión 2009, cuarta impresión 2010, quinta impresión 2011.
Contacta al publicador en la web www.opensourceyoga.ca

ISBN: 978-0-9735820-9-3

\newpage
\tableofcontents
\newpage


\begin{table}[h!]
	\centering
	\begin{tabular}{l|c||r}
		1 & 2 & 3\\
		\hline
		a & b & c\\
	\end{tabular}
	\caption{Caption for the table}
	\label{tab:table}
\end{table}


\begin{table}[h!]
	\centering
	\begin{tabular}{ccc}
		\toprule
		Some & actual & content\\
		\midrule
		prettifies & the & content\\
		as & well & as \\
		using & the & booktabs package\\
		\bottomrule
	\end{tabular}
	\caption{The other table}
	\label{tab:table2}
\end{table}

\section{FILOSOF\'{I}A, ESTILO DE VIDA Y \'{E}TICA}
texto
\subsection{Por qu\'{e} el yoga pudo haber pasado}
\susubbsection{Los upanishads}
\subsubsection{El bagavad Gita}
\subsubsection{Los Yoga Sutras de Patanjali}
\subsubsection{Vedanta - La filosof\'{i}a No-Dual del Sankara}
\subsubsection{Kashmir Saivism y los Tattvas}
\subsubsection{Yoga Moderno}
\subsubsection{Cronolog\'{i}a del Yoga}
\subsection{Dieta y Estilo de Vida}
\subsubsection{Sueño}
\subsubsection{Pr\'{a}ctica}
\subsubsection{Relaciones}
\subsection{El principio de atracci\'{o}n}
\subsubsection{Yoga y el principio de atracci\'{o}n}
\subsubsection{Reciviendo}
\subsubsection{Pr\'{a}ctica}
\subsection{\'{E}tica}
\subsubsection{Yamas y Niyamas}
\subsubsection{Mas de \'{E}tica de un maestro}
\subsection{Luz y Oscuridad}
\subsection{Mantras}
\subsection{Resumen de los Estilos de Yoga}
\subsubsection{Viendo la Imagen Completa}
\subsection{El Negocio del Yoga}
\subsubsection{Mercado}
\subsubsection{Yoga en Casa}
\Subsection{Sencillez Voluntaria}
\subsubsection{Principios de Sencillez}
\subsubsection{Acercamiento a Sencillez}
\subsubsection{Un Acercamiento al Lado Financiero}

\section{T\'{e}cnicas de Entrenamiento y Pr\'{a}ctica}
\subsection{Tecnicas de Asana}
\subsubsection{Alineamientos Fundamentales}
\subsubsection{Biomecanicas Holisticas}
\subsubsection{Terapia Estructural}
\subsubsection{Tecnicas de Asana: Categorias de Posturas}
\subsubsection{Forma y Accion}
\subsubsection{Navegando el Tapete}
\subsubsection{Sacro}
\subsubsection{La Practica de Asana}
\subsubsection{Yin y Yang}
\subsubsection{Polaridades de la Energia Fisica}
\subsubsection{Fuerzas Opuestas}
\subsection{Tecnicas de Purificacion}
\subsection{Meditacion}
\subsection{Pranayama}
\subsubsection{Nadi Shodhana}

\section{Anatomia y Psicologia}
\subsection{Los Vayus}
\subsection{Comprension y Tension}
\subsection{Anatomia Funcional}
\subsubsection{Huesos y Articulaciones}
\subsubsection{La Espina}
\subsubsection{Musculos}
\subsubsection{Musculos y Posturas}
\subsubsection{Yoga y Posturas}
\subsection{Los Bandhas}
\subsubsection{Mulabandha}
\subsubsection{Uddiyana Bandha}
\subsubsection{Jalandhara Bandha}
\subsection{La Respiracion}
\subsubsection{Respiracion Ujjayi}
\subsection{Elementos de la Naturaleza}
\subsubsection{Cualidades Caracteristicas de los Cinco Elementos}
\subsubsection{Ayurveda}
\subsection{Los Cinco Koshas}

\section{Metodologia de Enseñanza}
\subsection{Enseñando a dirigir}
\subsection{Agregando Contenido}
\subsubsection{Nivel 1 - Respiracion}
\subsubsection{Nivel 2 - Movimiento del Cuerpo Exterior}
\subsubsection{Nivel 3 - Alineamiento Fisico/Movimiento Energetico}
\subsubsection{Nivel 4 - Incorporando Intencion}
\subsubsection{Enseñando lo que Observas}
\subsubsection{Temporizando una Clase}
\subsubsection{Saludando/Centrando}
\subsubsection{Secuencia}
\subsubsection{Trabajo en Equipo}
\subsection{Tematizando}
\subsubsection{Tematizando a una Postura Especifica}
\subsection{Organizacion del Salon de Clase}
\subsubsection{Lineas Visuales}
\subsubsection{Diseño de Clase}
\subsubsection{Ofreciendo Props}
\subsection{Demostracion}
\subsubsection{Demostracion Silenciosa}
\subsection{Cuestiones de Salud}
\subsubsection{Estudiantes Lesionados}
\subsubsection{Usando Props}
\subsubsection{Cuidados Especificos de Salud}
\subsubsection{Yoga para Artritis}
\subsubsection{Fibromalgia}
\subsection{Lenguaje}
\subsubsection{Tono de Voz}
\subsubsection{Que Tanto Decir?}
\subsubsection{Se Conciso}
\subsubsection{Volumen/Contenido}
\subsection{Modificacion de Postura}
\subsection{Observacion: Estudiante Individual}
\subsubsection{Fundacion}
\subsubsection{Estado de Animo General}
\subsection{El Rol del Maestro}
\subsection{Fundamentos de Secuenciado}
\subsubsection{Secuenciado Variable y Establecido}
\subsubsection{Principios de Secuenciado}
\subsubsection{Entretenimiento - Centrando la Clase}
\subsubsection{Secuenciando la Clase a Nivel Mixto}
\subsection{Creando Intencion}
\subsubsection{La Intencion de "Liberar Tension"}
\subsection{La Practica y el Servicio de Enseñar Yoga}

\section{Practico}
\subsection{Tarea}
\subsubsection{Clases en Secuencia}
\subsubsection{Desarrollar Intencion para Clases}
\subsubsection{Autoevaluacion}

\section{Terminos de Sanscrito}
\subsubsection{Glosario y Terminos en Sanscrito}

\section{Un Entrenamiento Ejemplo}
\subsection{Un Entrenamiento Ejemplo - Yoga en Sillas}

\section{Ilustraciones}
\subsection{Flujo de Posturas}
\subsubsection{Surya Namaskara}
\subsubsection{Todos los Niveles de Practica de Asana}
\subsection{Posturas Syllabus}
\subsection{Ajustes Hands-on}

\pagenumbering{arabic}

\begin{equation*} 
	f(x)=x^2
\end{equation*}

$\begin{matrix}
 1 & 0\\
 0 & 1
\end{matrix}$

\begin{align*}
	1 &+ 2 = 3\\
	1 = 3 &- 2\\
	\int^a_b \frac{1}{3}x^3
 \end{align*}


\section{Section}

Hola

\subsection{Subsection}

Probando punto sin comando. Y ahora con comando\@. Ahora con minuscula sin. ahora con.\ fin.
\LaTeX
\today

\subsubsection{Subsubsection}
woah

\paragraph{Paragraph}
Mas texto D=

\subparagraph{Subparagraph}

Mas texto D=!!!!
 
\section{Another section}

\end{document}
