\chapter{Metodología de Enseñanza}
\section{Enseñando a dirigir}
Antes de hablar acerca de la filosofía de yoga dentro de una clase o incluso enseñar una postura, debes ser capaz de dirigir el movimiento de un estudiante con claridad y unas cuantas palabras.

\textbf{Ejercicio:}

Elige diariamente una actividad como abrir una puerta, quitarte un zapato, o rascar tu pierna. Escribe un guión para esa acción de tal manera que pueda ser llevada a cabo sin interpretación. Un ejemplo de una instrucción con escasez de claridad sería:

``Camina y toma la perilla de la puerta y abre la puerta.''

Esta instrucción funcionaría, pero sólo porque la persona a qui\'en instruyes ha abierto muchas puertas antes y sabe como, una mejor instrucción sería:

``Colocado a un brazo de distancia de la puerta, coloca tu pie izquierdo al frente y toma la perilla de la puerta con tu mano derecha. Rota la perilla en el sentido del reloj hasta que se detenga, luego colocando un poco más de peso en tu pi\'e derecho, suavemente jala la perilla hacia ti, abriendo la puerta.''

Enseña a un amigo a usar distintos objetos y movimientos diariamente, asegurándote de que tu amigo no interprete las instrucciones a su manera, sino que haga exactamente lo que instruíste. La experiencia puede ilustrar qu\'e tan difícil puede ser mantener tu voz con claridad y fuerza.

\section{Agregando Contenido}
Comienza con sencillez. Simple es claro. Claro es bueno, Practica tus enseñanzas simplemente instruyendo la respirazión de tu propia práctica del Saludo al Sol, una respiración por movimiento. Desde ahí, utiliza instrucciones fundamentales que est\'en tan bien establecidas que te permitan tener libertad creativa.

\subsection{Nivel 1 - Respiración}
Conforme instruyes verbalmente la respiración en tu propia práctica, escucho el tono de tu voz. Observa la sincronización y el paso de la instrucción simple de inhalar, exhalar. Crea un flujo tranquilo en tu cuerpo y tus palabras. Di la palabra ``inhala'' justo un poco antes de iniciar el movimiento en Urdhva Hastasana, y ``exhala'' justo antes del segundo movimiento hacia Uttasana. Observa tus propias tendencias a acortar la respiración.

Cuando te sientas cómodo dirigiendo tu propia respiración, visualiza  la secuencia de movimientos de Surya Namaskara en lugar de realizarlos, y dirige la respiración audiblemente. Cuando te encuentres cómodo con ese nivel de instrucción, camina alrededor del cuarto y dirige la respiración, manteniendo un paso tranquilo de movimiento sincronizando con tu propia inhalación y exhalación. Incorpora las enseñanzas de la respiración tan completamente de tal forma que si tu congelaras un instante en cualquier parte del saludo al sol, sabrías que parte de la respiración (inhalación o exhalación) se conecta.

\subsection{Nivel 2 - Movimiento del Cuerpo Exterior}
Lo siguiente es enseñar el movimiento del cuerpo exterior. La duración dentro de una postura es dependiente del estilo de clase que est\'es enseñando (como alentadora, restaurativa, meditativa, etc). Como punto inicial, cuenta tu número de respiraciones en un minuto, utilizando tus respiraciones para marcar la duración. Enseña posturas por 45 segundos por lado para una clase de paso medio, un minuto para un paso más lento.

Necesitarás escribir un guión para instruír la entrada a una postura. Comenzando tu entrada a la mayoría de las posturas de pie desde un desplante te proporciona un punto de referencia ``base''. Aquí está un ejemplo de instrucción meramente física a Parsvakonasana (postura de lado) entrando desde un desplante:

``Desde un desplante (realizado con el pi\'e derecho al frente, pi\'e izquierdo atrás) rotando tu pi\'e trasero 90 grados y presionando por completo las cuatro esquinas del pi\'e a la tierra. Coloca tu antebrazo derecho en tu muslo derecho y tu mano izquierda en la cadera. Rota tu torso hacia la izquierda.''

Esta es una instrucción muy básica, libre de palabras innecesarias. Esta instrucción tarda alrededor de quince segundos, permitiendo a los estudiantes realizar cada parte de la instrucción. Una vez que ellos han tomado la forma básica, la duración es cinco respiraciones o aproximadamente un minuto. La postura es repetira de nuevo hacia el otro lado con la misma duración. Escribe un guión básico para el cuerpo exterior para todas las posturas de pie que enseñarás y practica decirlas conforme realizas el asana, utilizando los movimientos de tu propio cuerpo como guía para la sincronización. Habla primero, despu\'es mu\'evete.

Cuando te sientes cómodo a este nivel, instruye tu propio cuerpo, intenta levantarte como si estuvieras dirigiendo una clase al frente del cuarto y dirigiendo oralmente una postura de pi\'e a la vez. Luego sincronízate, eliminando cualquier cosa poco clara o innecesaria.

\subsection{Nivel 3 - Alineamiento Fisico/Movimiento Energético}
Construyendo sobre la respiración y la forma básica de la postura, alineando el cuerpo óptimamente es lo siguiente. La alineación básica del cuerpo exterior (longitud de la postura, posición del cuerpo) debe ser tomada en cuenta en tu instrucción de movimiento del cuerpo exterior. Ahora puedes comenzar a describir el movimiento del prana, conectado a la inhalación y exhalación para alinear al estudiante con el pulso de la naturaleza. Conectando la inhalación a energía condensada, la exhalación a energía expansiva. Aquí está un ejemplo, basado en las previas instrucciones a Parsvakonasana:

``Desde un desplante (realizado con el pi\'e derecho al frente, pi\'e izquierdo atrás) rotando tu pi\'e trasero 90 grados y presionando por completo las cuatro esquinas del pi\'e a la tierra. Inhala conforme recoges tu energía de la tierra hacia tu centro. Coloca tu antebrazo derecho en tu muslo derecho y tu mano izquierda en la cadera. Rota tu torso hacia la izquierda. En tu siguiente exhalación, envía la energía hacia atrás a trav\'es de las piernas hacia la tierra''

Esta instrucción mejorada ahora toma alrededor de 25 segundos en ser expresada. A este punto, estás comenzando a introducir la intención filosófica. Al simplemente describir el flujo de la energía la atención del estudiante es recogida a esta pulsación universal de opuestos. Dependiendo del estudiante, esta incorporación física puede ser un momento ``Ah,Ha!'', o puede no resonar del todo. Continúa enseñando, Practica esto con todas las posturas de pie de nuevo, conforme las realizas, y luego mantente quieto.

\subsection{Nivel 4 - Incorporando Intención}
Aquí, tu enseñanza de clase debe referirse a la intención inicial de la clase introducida brevemente al comienzo de la clase. Puedes usar un poema, compartir una experiencia personal, recitar un yoga \textit{sutra} o utilizar cualquier otro material inspirador para introducir una intención. Cualquiera que sea, entrelaza la intención a travs de las instrucciones de postura en tu clase. Debido a lo que experimentamos físicamente en la práctica de asana se encuentra relacionado a una pulsasión mayor de la vida, la esencia de la enseñanza se vuelve evidente, encaminando a una profundización de conciencia en el estudiante. Una enseñanza creativa y hábil eleva la práctica del asana de una rutina física a su potencial como eperiencia que revela la integridad. Aquí está un ejemplo, de nuevo construyendo en previas instrucciones. La intención para esta clase, reflejado en las instrucciones de posturas, es el ``no apego.''

``Desde un desplante (realizado con el pi\'e derecho al frente, pi\'e izquierdo atrás) rotando tu pi\'e trasero 90 grados y presionando por completo las cuatro esquinas del pi\'e a la tierra, recon\'ectate con ella. Inhala conforme recoges tu energía de la tierra hacia tu centro, haciendo conciencia que lo que es tomado debe ser regresado. Coloca tu antebrazo derecho en tu muslo derecho y tu mano izquierda en la cadera. Rota tu torso hacia la izquierda, y con gratitud expande completamente en la inhalación. En tu siguiente exhalación, envía la energía hacia atrás a trav\'es de las piernas hacia la tierra, y permite este pulso de energía moverse a trav\'es de ti como un río, este flujo se mantiene limpio, sin inactividad.''

Este nivel de instrucción puede ser muy inspirador si viene naturalmente de un lugar de verdadera experiencia en un maestro. Debes decidir con qu\'e nivel de instrucción te encuentras cómodo. Asegurate de que tienes los fundamentos de instrucción absolutamente sólidos antes de avanzar. No tiene sentido intentar transmitir aspectos sutiles de nuestra verdadera naturaleza cuando media clase se encuentra en otra postura.

Tu intención funcionará más efectivamente si es una intención que puede ser incororada. El ejemplo anterior funciona en una práctica de asana porque liberanr la respiración es claramente una forma física de no apego. Tu puedes tener una gran intención de clase que es difícil de incorporar. Una intención como ``estudiar una escritura para incrementar la sabiduría'' es una gran cosa para dedicar el esfuerzo personal, pero es difícil desmenuzar este tipo de intención en el cuerpo durante la práctica de asanas.

\subsubsection{Dirigiendo posturas específicas}
Ser un buen maestro de yoga es como ser un buen mesero. Diriges a los estudiantes para tomar su asiento, les describes qu\'e hay en el menú, modificas probablemente alguno que otro platillo para servirlo mejor, y ocacionalmente vas a revisar cómo van las cosas. Cada tiempo de la comida necesita ser llevado y presentado, y entonces el mesero permite a los comenzales disfrutar.

Al utilizar un lenguaje efectivo y claro, los estudiantes a quienes enseñas deberán ser instruídos para entrar y salir de la forma básica de cada postura que enseñas. Los estudiantes avanzados pueden estar familiarizados con los nombres de muchas posturas, pero los principiantes requerirán instrucción de cómo aproximarse a cada postura. Las instrucciones básicas son esenciales antes de detallados o alineaciones que se enseñen. Para hacer esto, necesitarás escribir un guión para cada postura que pretendas enseñar.

Este trabajo se vuelve cada vez más sencillo dado que muchas posturas comparten similaridades. La longitud de apertura para muchas posturas de pie es la misma. Muchas posturas de piso comparten tambin similaridades en su forma general. La maestría de enseñar una vez que una forma básica de una postura es realizada por tus estudiantes consiste en ilustrar las diferencias entre las posturas.

\subsubsection{Algunos ejemplos de posibles guiones:}

\underline{Surya Namaskara - Saludo al Sol}

``Inhala y eleva tus brazos, exhala y flexionate al frente, tocando el suelo.
Inhala y mira al frente.
Exhala, camina hacia atrás y baja a plancha baja, llamada Chataronga Dandasana.
Inhala, desliza tu cuerpo hacia el frente y arriba en Urdhva Mukha Svanasana (perro boca arriba).
Exhala y muevete hacia atrás en una V invertida, o Adho Mukha Svanasana.
Mant\'en esta postura por cinco respiraciones, siente la conexión con la tierra mediante de tus dedos.
Flexiona las rodillas ligeramente, estira hacia atrás desde las manos para alargar la columna.
En tu sexta inhalación, caminas al frente y mira al frente.
Exhala y flexiona al frente de nuevo.
Inhala colócate de pie, eleando tus brazos al cielo.
Exhala manos al pecho y sigues de pie.''

Debido a la velocidad de Surya Namaskara, moverse a una postura por respiración, no es posible instruir mucho más que la entrada básica a cada postura. Surya Namaskara es un lugar excelente para comenzar a instruír posturas con presición debido a esto.

\subsubsection{Ejercicio:}
Escribe un guión para Surya Namaskara en tus propias palabras. Tu propio uso de gramática, el lenguaje coloquial y la elección de palabras es como una firma. Tu forma natural de interpretar la práctica es única de ti. Has la enseñanza propia, no una copia de nadie más. Al mismo tiempo, mantn el número de palabras al mínimo mientras sigues dando instrucciones claras.

\subsubsection{Continuando los ejemplos de guines\ldots}
\underline{Parsvakonasana}\\
``Desde perro boca abajo, da un paso al frente entre las manos. Colocando tu antebrazo derecho en tu rodilla, rota tu torso hacia la izquierda conforme rotas tu pie trasero 90 grados hacia el frente de tu tapete. MAnteniendo tu espalda fuerte, dirige tu rodilla derecha justo sobre tu tobillo. Eleva el brazo izquierdo 45 grados hacia el techo, con la palma mirando hacia abajo. Respira profundo, relajando la mirada. Mantente ahí por cinco respiraciones. En tu siguiente inhalación, mant\'en ambas palmas en el tapete, exhala y camina hacia atrás en perro boca abajo.''
AQUI VA UNA IMAGEN

\underline{Janu Sirsasana}\\
Desde Dandasana,lleva tu pierna izquierda hacia el frente y coloca la planta del pie contra tu muslo. Mant\'en los dedos en el suelo y mant\'en firmes los músculos de tus piernas. Desde la base, extiende la columna hacia arriba y la pelvis hacia el frente. Toma tu pierna, tobillo o pie con la mano derecha. Usando la conexión con la tierra a trav\'es de los dedos izquierdos, inhala y cuadra tus hombros, exhala y flexiona hacia el frente en la postura. Cada inhalación, alarga la columna. Cada exhalación, entr\'egate más profundamente en Janu Sirsasana. Toma cinco largas respiraciones aquí. Mant\'en tu pierna extendida firme hacia el suelo y en tu siguiente inhalación, el\'evate de nuevo de la flexión y coloca las piernas en Dandasana de nuevo.``
AQUI VA UNA IMAGEN

\underline{Rajakapotasana}\\
Desde perro boca abajo, desliza tu pie derecho al frente, cruzándolo por el frente del tapete de tal forma que tu rodilla dercha se encuentre más abierta que tu cadera, y la espinilla exterior y el tobillo traseros descancen en el tapete. Mantn tu pierna izquierda activa mientras apoyas las uñas de los pies en el tapete, estirándose hacia atrás. Los dedos o palmas debajo de los hombros, inhala y alarga la columna, manteniendo tu base firme en ambas piernas. Exhala y arquea hacia atrás, llevando la punta de la garganta atrás. Continúa activando las piernas y llena las costillas con aire. Despus de cinco respiraciones, coloca las palmas hacia abajo y empuja firmememnte en perro boca abajo.``
AQUI VA UNA IMAGEN

\underline{Savasana}\\
``Ahora el esfuerzo de la práctica de asana ha terminado. Acu\'estate en tu tapete, permitiendo que tus palmas se abran y las piernas y pi\'es se relajen. Toma otra respiración profunda por la naríz y exhala por la boca. Permitiendo que los ojos se relajen hacia atrás en sus cavidades con los párpados descanzando sobre ellos. Siente la conexión entre tu cuerpo y el suelo y entra más profundamente en la experiencia de la relajación.
AQUI VA UNA IMAGEN

Cada palabra que utilizes debe ser específica y no fácilmente mal interpretada. Recuerda, tus estudiantes se estarán esforzando físicamente, experimentando alguna incomodidad e intentando escuchar tu voz por encima de su propio diálogo. Utilize un diálogo activo y directo para dirigir las formas básicas de las posturas.

Esos ejemplos son simples y directos, y constituyen las ``tuercas y tornillos'' de instruír posturas. Tu propia energía, palabras e intención para la clase que estás enseñando florecerán de una instrucción clara de fundamentos básicos, y te permitirá verdaderamente llevar tu pripia voz a la clase.

Apegarte a las instrucciones básicas te permitirá tambi\'en relajarte completamente conforme enseñas. Esta fluidez y confianza tangible motivará a tus estudiantes a relajarse y tambi\'en aceptar las enseñanzas más completamente. Cuando ya dominas instruír con sencillez, entonces puedes tambi\'en comenzar a enseñar lo que observas, agregando los ajustes apropiados físicos y verbales cuando sean necesarios.

\subsubsection{Ejercicio:}
Escribe un guión para cinco posturas de pie y cinco posturas de piso. Diríge la entrada en posturas de pie desde perro boca abajo. Dirige las posturas de piso desde Dandasana. Se tan corto como sea posible por ahora, recordando que el estudiante necesita encontrar la postura y luego pasar cinco largas respiraciones experimentándola.

Despu\'es de escribir tu guión, leelo en voz alta en una voz natural y observa si suena como algo que dirías. Evita descripciones anatómicas no familiares o jerga que la población general no entendería.

\subsection{Enseñando lo que Observas}
Ahora que tienes las herramientas para enseñar a un nivel de sutileza que es apropiado para ti, debes enseñar la clase que estás dirigiendo. Observa lo siguiente:

\begin{itemize}
	\item la base de la postura (usualmente los pies) para alineación y conexión;
	\item la calidad de respiración - escucha la velocidad, la suavidad, la profundidad; y
	\item la forma básica de la postura - posicionamiento, alineación, flujo de energía.
\end{itemize}

Prepárate para ajustar tu clase dependiendo de lo que veas. Si la secuencia que estás enseñando es claramente muy difícil o físicamente muy demandante para la mayoría de la clase, ajusta acorde sin bajar el nivel completamente.

\subsubsection{Ajustes verbales}
Despu\'es de observar, puedes dar ajustes a las posturas verbalmente. Recuerda que cada estudiante es distinto, desde la estructura ósea hasta sus habilidades. No toda postura es apropiada para todo estudiante, y dos estudiantes no se verán igual en la misma postura. No hay postura ``perfecta'' en t\'erminos de cómo se ve en el exterior. La postura perfecto es aquella que se encuentra balanceada en acción, con una completa luz de conciencia que ilumina el cuerpo del practicante, y resulta en una expresión de creatividad y júvilo.

Hay formas de crear un flujo más abierto y profundo del prana y seguro para las articulaciones en una postura. Cuando observas a un estudiante y has determinado que su postura puede ser ajustada para verelar más conciencia, o alinearlos más para una práctica más segura, debes hacer una elección para verbalizar esto.


\subsection{Temporizando una Clase}

La mayoría de las clases de yoga duran entre 1 y 1.5 horas, generalmente 1 hora 15 minutos. Tus clases deben consistir en:

\begin{itemize}
	\item Saludar y centrar/tomar un lugar
	\item Introducir intención
	\item Calentamiento básico
	\item Posturas de pie
	\item Posturas de piso
	\item Meditación
	\item Savasana
	\item Cerrar
\end{itemize}

Necesitas distribuir el tiempo entre esas categorías. Utilizando la plantilla básica para práctica personal descrita en \hyperref[sec:fundSec]{Fundamentos de una Secuencia}, podemos establecer la cantidad de tiempo que una clase tomará. Mant\'en en mente:

\begin{itemize}
	\item La duración por postura puede ser entre 30 segundos a 1 minuto. Posturas que tienen un lado izquierdo y uno derecho tomarán entre 60 segundos y 2 minutos para enseñar. Puedes usar un reloj para medir tu respiración por 1 minutos, y luego usar tu respiración para marcar la clse.
	\item Tu saludo e introducción de intención deben ser de entre 2 y 5 minutos.
	\item Los calentamientos básicos pueden incluír posturas sentado, aperturas de pecho y hombros, trabajo de respiración, saludos al sol, o algún otro movimiento repetitivo designado para llevar a los alumnos ``hacia'' su cuerpo, antes de que una instrucción detallada de posturas comience (5-10 minutos)
	\item Las posturas de pie pueden tomar entre la mitad del tiempo restante asignado para asanas y posturas de piso la siguiente mitad, dejando tiempo para meditación y relajación.
	\item Savasana (relajación) debe ser entre 5 y 10 minutos.
	\item Cerrar la clase puede incluír una oración final, agradeciendo a los alumnos por su atención, t\item  y responder preguntas.
\end{itemize}

\subsection{Saludando/Centrando}
\begin{itemize}
	\item Posturas de pie (piernas cruzadas, medio o loto completo)
	\item Introducir intención. Comenzar respiración Ujjai. (5 minutos)
\end{itemize}

\subsection{Secuenciando}
\begin{itemize}
	\item Supta Padangustasana (manos detrás de los muslos) (2 min)
	\item Adho Mukha Svanasana (perro boca abajo) (1 min)
	\item Lunge (dedos al suelo) (2 min)
	\item Surya Namaskara (modificado) x3 (6 min)
	\item Surya Namaskara (completo) x2 (5 min)
	\item Trikonasana (postura de triángulo) (2 min)
	\item Parsvakonasana (de lado) (2 min)
	\item Parsvattonasana (2 min)
	\item Prasarita Padottanasana (flexión frontal de pie ancha), A y C (2 min)
	\item Utthita Hasta Padangustasana (equilibrio mano al dedo gordo) (2 min)
	\item Rajakapotasana (postura de paloma rey) (3 min)
	\item Bakasana (cuervo) (2 min)
	\item Balasana (postura de niño) (2 min)
	\item Navasana (bote) x3 (3 min)
	\item Virasana (heroe) (1 min)
	\item Supta Virasana (heroe suprino) (1 min)
	\item Ustrasana (camello) (1 min)
	\item Urdhva Dhanurasana (rueda completa) (2 min)
	\item Marichyasana C (torsión) (2 min)
	\item Dandasana (postura de bastón) (1 min)
	\item Upavistha Konasana (flexión frontal con apertura ancha) (1 min)
	\item Janu Sirsasana (flexión frontal con una pierna) (1 min)
	\item Pranayama/meditacíón (4 min)
	\item Savasana (10 min)
\end{itemize}

La secuencia anterior es programada para 65 minutos. Permitiendo alguna demostración llevando la clase a un grupo antes de entrar a otr apostura, el tiempo total de la clase debe ser aproximadamente 75 minutos. Esto sería una clase marcada de forma moderadamente energ\'etica.

Ejercicio:

Utilizando la plantilla básica para la planeación de una clase (ver \hyperref[sec:fundSec]{Fundamentos de una secuencia}), escribe tu propia secuencia de posturas y asigna tiempo dependiendo tu intención de crear un levantamiento de energía, más restaurativa o clase meditativa. Las posturas restaurativas que incluyen posturas suprinas y asistidas toman más tiempo en realizar y la duración es usualmente mayor (2-10 minutos). Por lo tanto, habría pocas posturas en una misma clase.

\subsection{Trabajo en Equipo}
Poner a los estudiantes en pareja puede ser divertido e informativo, aligerar el caracter de la clase, y es una oportunidad para crear comunidad. Los estudiantes necesitan entender cuál es el propósito del trabajo en equipo, y cómo hacerlo con sin peligro. Algunas personas no están interesadas en trabajo en equipo por distintas razones, que pueden incluír lesiones, enfermedades, desinters en comunicarse al final del día, o muchas otras razones. Pon tu juicio en el uso de trabajo en equipo.

Antes de unir estudiantes para un trabajo en equipo, demuestra claramente qu\'e quieres hacer, utilizando a un estudiante para asistir en la demostrasión. Pregunta al estudiante con el que trabajaste si el trabajo ha cumplido su propósito. Antes de que el trabajo en equipo comience, pregunta a la clase si tienen alguna pregunta. Asegúrate de que todos tienen pareja, y si no, únete con un estudiante.

\section{Tematizando}
``Cada clase tiene un tema orientado al corazín, que tiene una conexión significativa a los propósitos espiritual es en una práctica de asana. El tema usualmente se centra en cultivar una virtud (una cualidad de mente o corazón), que es una reflexión microcósmica de nuestra naturaleza divina. Cada tema nos da una dirección para la actitud de la energía que prepara cada acción y respiración en las poses. Efectivamente, todas las posturas en Anusara Yoga están expresadas de ``adentro hacia afuera''. El tema está entrelazado con la instrucción de las posturas a lo largo de la clase.``
-John Friend

Un ejemplo de tema podría ser ``alegría'':

``La alegría es observada en quienes se acercan a la vida con luz, y una actitud sin miedos. Esta actitud se basa en un entendimiento de que estamos, a un nivel fundamental, perectamente bien. Pueden gustarnos o disgustarnos ciertas cualidades que poseemos, pero la práctica de yoga nos permite ver mas allá hacia nuestra verdadera naturaleza. Hoy en día cuando practicamos, nos divertimos con las posturas, permite tu naturaleza creativa desenvolverse a travs de la práctica.''

Existe un sentimiento expansivo cuando nos permitimos ser alegres y un poco descuidados. Muchos temas pueden contener polaridad, pareciendo opuestos, como esfuerzo y entregam estabilidad y libertad, uno que limita al otro y que en realidad trabajan juntos.

\subsection{Tematizando a una Postura Especifíca}
Si decides aplicar un tema a tu clase, será mucho más efectivo si puedes referirte a el dentro del contexto de las posturas, no sólo al inicio y al final de la clase. Por ejemplo:

\textbf{Bakasana:}

``Desde perro boca abajo, flexiona tus rodillas y brinca suavemente hacia tus brazos flexionados, como si quisieras aterrizar en ellos con tus rodillas. Observa esto ocurriendo en tu mente primero. Libera cualquier idea de imposibilidad.''

Este es un ejemplo de temática, realmente solo un palabra. Para mantener las cosas interesantes para los estudiantes es una buena idea escribir algunos sinónimos que describan tu temática: Alegría, ligereza, expansión, sin miedo, con júvilo, con una sonrisa, sin ataduras.

Los temas más efectivos describirán las cualidades del corazon o de la mente: coraje, firmeza, deteminación, compasión, compasión y muchos otros.

Si tienes una historia personal que se relacione a tu tema, esto resonará con los estudiantes con más fuerza que sólo referirse a una emoción o virtud lejana que no resuena contigo.

Existen muchas maneras de comenzar una lluvia de ideas para temáticas, todo lo que tienes que hacer es pensar en una experiencia que has tenido y cómo te relacionas con ella, entonces descubre la virtud o cualidad del corazón que fue necesaria para negociar con habilidad esa circunstancia. Hasla personal, y luego relacionala con la experiencia del estudiante. Por ejemplo:

``Acabo de regresar de un taller de yoga en Tofino. Cuando llegu una noche antes, estaba exausto y tenía que dirigirme a la clase regular de mi anfitrión Natalie. No me sentía realmente con ánimos de enseñar a la mañana siguiente, pero al hablar con Natalie despu\'es de su clase y ver su increible entusiasmo por el taller del día siguiente y todo el trabajo que ha hecho para prepararlo, algo en mi cambió. Me di cuenta de que mi propia fatiga podría esperar hasta que llegara a casa, y la experiencia colectiva de los estudiantes era más importante. La energía que sentí me movió, y tuvimos un gran taller. Cuando llegu\'e a casa dormí por 12 horas.

Hoy, cuando practiques, ve hacia esas reservas internas de energía y poder que tienes. El tiempo que tenemos juntos es corto. Hagamos a esta práctica vibrar con energía, y al final toma un Savasana muy profundo.``

Entonces, cuando enseñes una postura, relaciona la virtud de la que hablas (en este caso energía) y muevela a instrucciones de posturas. Elige posturas que funcionen bien con tu tema; por ejemplo, incluye montones de posturas mantenidas de pie un poco más de lo normal:

``De Tadasana, brinca al lado de tu tapete, con los pies separados lejos para Prasarita Padottanasana (flexión con piernas abiertas). toma una respíración profunda  siente la pulsasión de la fuerza en tu respiración conforme suavisas tu piel. Inhala y toma fuerza a trav\'es de tus pies hacia el centro de tu cuerpo. Junta tus manos detrás de la espalda conforme flexionas ligeramente las rodillas. Con las rodillas flexionadas, recoge tus muslos conforme te flexionas hacia el frente. En tu siguiente exhalación, baja tu cosis y envía esa reserva de energía a travs de tus piernas y brazos. Con determinación, extiende un poco más.''

Si estás teniendo problemas pensando un tema, puedes reflejarte en:
\begin{itemize}
	\item La cosa más difícil que has hecho.
	\item Lo más divertido que has vivido.
	\item Una experiencia que te ha tocado profundamente.
	\item Un momento cuando pudiste actuar mejor, y qu\'e cualidad faltó en ese momento.
\end{itemize}

Escribe un enunciado o dos acerca de estas experiencias, luego encuentra la virtud o emoción central. Por ejemplo, en el enunciado, ``la cosa más difícil que he hecho fue decirle a mi hermano menor, quien tenía 11 años, que nuestro padre murió``, la emoción central o virtud necesaria era compasión de estabilidad.

Tomate un tiempo para escribir el tema para una clase, compártelo con tu grupo y compartan retroalimentación mutuamente. El tema de tu clase debe conectarse con de tu propia experiencia a la experiencia colectiva del grupo, ilustra un concepto que pueda ser incorporado físicamente, que sea simple y memorable.

\section{Organización del Salón de Clase}
Para enseñar con eficiencia y dar ajustes físicos y verbales, el maestro necesita ver a sus estudiantes. Aún cuando esto suena obvio, la observación es una de las cosas más difíciles de enseñar Yoga. Puede ser relativamente fácil observar a un estudiante individual y dar apoyo y claridad de tus instrucciones. Puede ser mucho más difícil cuando se enseña a una clase de veinte o treinta estudiantes. Afortunadamente, hay t\'ecnicas qu epueden hacer tu trabajo ligeramentre más fácil.

\subsection{Líneas Visuales}
Para escanear más rápido alrededor del cuarto y verificar la forma general de la postura que has instruído, colocación de pies y/o manos, y ver ue la postura está siendo realizada en el lado (izquierdo o derecho) que has dirigido, has lo siguiente:

\begin{itemize}
	\item Que los estudiantes dejen sus tapetes alineados perpendicular a la pared más larga en el cuarto.
	\item Si tienes una clase grande, alinea otra fila frente a la primera.
	\item Repite conforme sea necesario.
	\item Acomoda la segunda fila de tal forma que los tapetes se intercalen frente a la primer fila, como los asientos de un cine. Esto te permitirá hacer contacto visual con cada fila de el frente del cuarto, donde comenzarás la clase.
\end{itemize}

La habilidad de observar un grupo de estudiantes te permitirá hacer los ajustes verbales más eficiente al grupo entero o a un individuo dentro del grupo basado en tu observación, El contacto visual (la habilidad de los estudiantes tambi\'en de verte cómodamente) te permitirá comunicarte e inspirarlos a trav\'es de expresiones faciales. Ver las caras de los estudiantes, te permitirá determinar el estado de su mente más fácilmente y notar signos de disconformidad y color de piel.

Al inicio de la clase, enseñarás desde el frente del salón, donde tu tapete se encuentra colocado. Para posturas orientadas hacia el frente como la postura del árbol, postura de silla, etc, colócate hacia el frente del tapete. Cuando muestres posturas que se encuentran orientadas de izquirda a derecha, como triángulo o guerrero dos, muevete hacia el lado del salón que los estudiantes se encuentran viendo. De esta forma, puedes mantener un mejor contacto visual y comunicación. Utiliza esto mismo para posturas de pie y de piso.

Las posturas que se encuentran orientadas hacia atrás, como perro boca abajo, son complicadas (si no es que imposibles) de mantener un contacto visual. Estas posturas pueden ser usadas como una oportunidad para dar más ajustes físicos.

Cuando los estudiantes se encuentran en posición suprina (acostados sobre la espalda) es la forma más fácil de moverse entre el grupo despacio para crear contacto visual, hacer obsercaciones, y ajustes físicos y verbales.

Sin importar el tamaño del salón, mantn a los estudiantes no más de un tapete de distancia el uno del otro. Los estudiantes tienen la tendencia a separaese, haciendo difícil para el maestro en el realizar observaciones y ajustes. Mantener a los estudiantes cerca ayudará tambi\'en a mantener mayor atención y energía de en la clase, y te permitirá moderar la voz ya que no habrá necesidad de proyectarla en más de una dirección. Asegúrate de tener suficiente espacio para moverte entre los tapetes para realizar ajustes físicos, de ser necesario.

\subsection{Organización de la Clase}
%%------------------Aqui va una imagen--------------%%

\subsection{Ofreciendo Props}
Decide de antemano en la planeación de tu clase si deseas realizar posturas que requieran props. De ser así, crea una plantilla, o ejemplo, de los props requeridos para la clase al lado de tu propio tapete al frente del salón. Cuando saludes a los estudiantes conforme lleguen, indica qu\'e props serán requeridos. Si impartirás varias clases, toma el tiempo al inicio de una clase para explicar el uso de los props y permite que los estudiantes observen si se beneficiarán o no usando un block o una cuerda en su práctica.

\begin{itemize}
	\item Las cuerdas pueden ser extendidas para alcanzar al estudiante, permitiendo que la fuerza del tronco superior ayude a lograr una flexión frontal más profunda. En este caso, la cuerda rodea los pies y se sujeta por ambos lados. De esta forma, la columna se mantiene extendida.
	\item Los blocks pueden ser utilizados para llevar a la pelvis a una posición neutral para las posturas de piso. Tambi\item n pueden ser usados para ayudar a la columna en un arco supino.
\end{itemize}

Existe una variedad de uso para los props. Aún cuando son útiles, pueden disminuír el ritmo de la clase considerablemente. Si estás impartiendo una clase flow, elige posturas con una mínima necesidad de props.

\section{Demostración}
Utiliza las demostraciones sabiamente. Demostrar una postura o una apoximación puede ser muy ilustrativo, porque el estudiante no puede ver la forma de su propia postura. Ver la postura desde otra perspectiva es de mucha ayuda y puede explicar con frecuencia lo que las palabras no pueden.

Sin embargo, demostrar tiene el efecto de romper la continuidad del flujo dentro de la clase y alterar la dirección de concentración para el estudiante. Cuando decides demostrar, asegurate que se cumple los iguiente:

\begin{itemize}
	\item Cada estudiante puede verte y escucharte. Pide que se acerquen.
	\item S\'e conciso conforme variante que quieres que realicen, o el objetivo de la postura que estás demostrando. Qu\'edate con uno o dos puntos principales.
	\item Si estás utilizando un estudiante para demostrar, pide permiso discretamente primero.
	\item Cuando termines la demostración, agradece a los estudiantes por su atención, y el estudiante que has usado para demostrar.
	\item Pide que los estudiantes regresen a su tapete y realicen la misma postura de nuevo, utilizando lo que fu\'e remarcado en la demostración.
\end{itemize}

\subsection{Demostración Silenciosa}
Un fenómento interesante y poderoso es que la parte del cerebro usada para controlar una cierta parte del cuerpo responde tambi\'en a una estimulación visual, si es demostrada en silencio. Saltar la zone del lenguaje del cerebro puede resultar en un entendimiento kinest\'esico más profundo. Para realizar una demostración silenciosa, haz lo siguiente:

\begin{itemize}
	\item Dile a tu clase que demostrarás algo en silencio, y explica por qupe.
	\item Pide mantener el silencio.
	\item Entra en la postura demostrando con la respiración.
	\item Apunta a la parte del cuerpo que estás enfocando en tu demostración y realiza la variante que estás remarcando.
	\item Sal de la postura antes de hablar de nuevo.
\end{itemize}

\section{Cuestiones de Salud}
Cuando se enseña a un nuevo estudiante, se debe llenar una forma de admisión por el estudiante, listando cualquier problema de salud conocido. Si la condición es seria, o tienes restricciones sobre los estudiantes de tu clase, pide que visiten a su doctor para obtener un diagnóstico. La práctica de asana puede ser benfica para muchas enfermedades comunes pero contraindicada para algunas. Dirige a los estudiantes que deseen practicar, pero tienen serios problemas de salud, a un terapeuta calificado de yoga, si conoces alguno. Siempre ve a lo seguro.

\subsection{Estudiantes Lesionados}
Cuando enseñas a un estudiante con una lesión epecífica o alineamiento, observa de cerca su respiración, color de piel y habilidad de entrar a las posturas. Perpárate para ofrecer variantes o decrementar la intensidad para ese estudiante y recuerda siempre estar atento a señales de dolor o inconformidad dentro de su cuerpo a la hora de seguir las instrucciones de la clase general. Trabajar con una lesión puede ser frustrante, Tómate el tiempo para asesorar cuándo es y cuándo no realmente benfico para un estudiante específico participar en una clase supervisada, y no permitas que un estudiante lesionado realice un asana con mala alineación que creará dolor y posiblemente una mayor lesión. Sugiere una clase privada como alternativa.

\subsection{Usando Props}
La idea de usar props para sostener el cuerpo superior puede ser tremendamente beneficioso para llevar al cuerpo a una alineación óptima. Por ejemplo, para estudiantes con espalda baja y cadera rígica, sentarse sobre un block firme permitirá que la pelvis se rote hacia el frente lo suficiente para que la columna se alargue por completo hacia arriba, permitiendo una flexión frontal o una torsión más óptima. Mant\'en en mente que el uso de props alentará el ritmo de la clase considerablemente. Si planeas usar props (cobijas, blocks, cuerdas) presenta el prop y explica los beneficios de usarlo. Seguido los estudiantes ven los props como un insulto y crítica de que no son capaces de realizar lo que otros estudiantes. Si entienden que el prop puede ayudar a alinear el cuerpo y avanzar en la práctica, estarán más dispuestos a utilizar un prop de ser necesario.

\subsection{Cuidados Específicos de Salud}
\subsubsection{Fatiga crónica}
Utiliza posturas restaurativas e inversiones suaves para estimular la circulación. Mant\'ente dentro del nivel de energía del estudiante sin sobrepasar sus límites. Algunas posturas activas son ben\'eficas para fortalecer los músculos.
\subsubsection{Dolor de cabeza}
Recuerda al estudiante mantenerse hidratado, ya que este es un factor que suele contribuír a los dolores de cabeza. Suaviza los músculos de hombros y cuello con posturas apropiadas. Cuando se practiquen pranayamas simples, enfatiza las exhalaciones largas y suaves, cerrando los ojos para relajar los nervios del sistema y liberar tensión
\subsubsection{Problemas cardiacos}
Armonizar la respiración y aperturas de pecho simples, como arcos suavez, pueden estimular el flujo de la sangre. Una práctica suave es recomendada para disminuír el esfuerzo en el sistema circulatorio, incrementando la intensidad suavemente.
\subsubsection{Estreñimiento}
Soltar la zona del piso p\'elvico donde naturalmente el flujo inferior del prana se puede atorar es la llave. Aperturas de cadera con \'enfasis en la liberación de energía, y de nuevo, mucha agua es útil.
\subsubsection{Insomnio}
El insomnio es causado frecuentemente por un sistema nervioso agitado. Las posturas que enfatizan enfriamiento y liberación son de gran beneficio para las personas que padecen de esta condición. Sarvangasana (parado de hombros) y Halasana (arado) son de gran ayuda para esto. Si el estudiante es por el contrari capaz, una práctica relativamente vigoroza para liberar la energía nerviosa, seguida por un largo Savangasana para incentivar la relajación es adecuado.
\subsubsection{Dolor de espalda baja/ciatica}
Las piernas, nuestra conexión con la tierra, se encuentran fuertemente relacionadas a la salud de la espalda baja. Los huesos del muslo necesitan aterrizarse y recogerse energ\'eticamente hacia las corvas para colocarse en el centro del hueco de la cadera, y por lo tanto permitir la tensión de la espalda baja liberarse. Realiza flexiones de pie, enfatizando el alargamiento de la columna, y simultaneamente mover la parte superior del f\'emur hacia atrás.
\subsubsection{Dolor de muñecas}
Revisa la alineación de las crestas de las muñecas en perro boca abajo. Asegúrate que el estudiante se encuentre practicando los asanas en una superficie firme. Enfatiza una fuerte conexión con la tierra a travs de todos los dedos y el borde de la mano. Padangustasana (manos bajo los pies, flexión de pie) puede liberar la tensión de las muñecas tambi\'en.

\subsection{Yoga para Artritis}
Artritis, literalmente ``inflamación de articulaciones'' es una de las enfermedades más frecuentes. Para adaptar el yoga para la artritis necesitamos entender primero qu\'e es, y distinguir entre los tipos de artritis.

\subsubsection{Osteoartritis}
Esto es básicamente ``usese y tírese'' en las articulaciones. Puede haber muchas posibles causas para la osteoartritis, incluyendo trauma, sobreuso, desalineación crónica, deficiencias en la estructura química de los tejidos de articulaciones, y más. A lo largo de la vida, el cartílago que cubre las articulaciones puede ser fácil de desgastar. Demasiado actividad puede lastimar las articulaciones, al igual que la mala circulación y falta de movimiento.

Con el tiempo, conforme el cartílago se desgasta, puede ser reemplazado por formas efectivas de cartílago menos efectivas y eventualmente el hueso mismo queda expuesto. Tallar ``hueso con hueso'' es extremadamente doloroso dado que los huesos son sencibles.

Una articulación es el área donde dos huesos se unen, conectados por tendones que unen músculo y hyeso. ligamentos que conectan un hueso con otro, y al final de los huesos mismos. Los extremos de los huesos se encuentran conectados con cartílago articular (un resistente, relleno que absorbe impactos con una cobertura superficial que permite los extremos de los huesos deslizarse fácilmente uno con otro. Este desliz es facilitado por un lubricante viscoso llamado ``fluido sinovial''. Este mecanismo entero es colocado en una articulación ``capsula'', un escudo de tejido conectivo relleno con fluido articular.

Los músculos funcionan en pares opuestos para mover y estabilizar cada articulación. Cada articulación en nuestros cuerpos es, en esencia, una relación. Necesitamos que estas relaciones para poseer la habilidad de movernos en muchas formas distintas. Imagina no tener las articulaciones de las rodillas, sólo un hueso largo de tu cadera a tu robillo. Como cualquier relación, cuidado y atención debe ser brindada a las articulaciones para que funcionen óptimamente a lo largo del tiempo.

\subsubsection{Artritis Reumatoide}
Esta es la siguiente forma de artritis más común despu\'es de la osteoartritis. Esta forma de padecimiento no es causada por desgaste, sino por el propio sistema inmunológico poni\'endose en contra de sí mismo y destruyendo tejido saludable. Imagina limpiar un suelo de madera, puedes tallar con suficiente fuerza y frecuencia para mantener los pisos limpios, pero tambi\'en podrías tallar tan fuerte que arranques la superficie del suelo.

En la artritis reumatoide, el foco principal de ataque son los revestimientos de las articulaciones, que se llenan con c\'elulas blancas que atacan el tejido saludable.

Las causas de la artritis aún no son del todo comprendidas. Si nos enfocamos en los caminos ayurv\'edicos para esta enfermedad, está claro que ambas formas de la enfermedad son inflamatorias (desvalance pitta) pero el dosha vata se encuentra desvalanceado (demasiado ``espacio'') y un decremento en el dosha kapha puede contribuir a decrementar la cantidad de fluido sinovial.

Mantener el cuerpo movi\'endose de una forma indolora es ciertamente una buena idea para cualquiera, especialmente aquellos que sufren de artritis de cualquier forma. La práctica de asanas con un enfoque en las posturas correctas, una práctica que no sobrecaliente el cuerpo o la mente, y se enfoque en alineamiento preciso será ben\'efico para aquellos que padezcan de esta condición. Dieta y estilo de vida ueden jugar un rol tambi\'en.

Trabajar uno a uno con el estudiante es recomendable, al igual que tener props disponibles, tal como silla, blocks, cobijas y cuerdas. Pregunta algunas cosas sobre su objetivo en la práctica de yoga, qu\'e partes del cuerpo son problemáticas, tambi\'en qu\'e disfrutan hacer y no sea doloroso! Comienza enseñando una respiración suave y audible como harías con cualquier estudiante. Trabaja sin dolor, adaptando posturas simples con una observación aguda, tanto visual como auditiva.

La clave para trabajar efectiva y terap\'euticamente es la comunicación. Pregunta a tu cliente cómo se siente cuando trabaja con alguna articulación. Usa ``en escala del 1 al 10'' para dolor, o ``igual, mejor, peor''. La mejor terapia ocurre cuando la articulación soporta tanto peso como es posible, sin dolor ni incomodidad.

Trabaja dentro de las capacidades del cliente, deja ejercicios simples de tarea, termina con una postura restaurativa. asegúrte de que disfrutan el proceso del yoga.

\subsection{Fibromalgia}
Fibromalgia describe un conjunto de síntomas que incluye dolory sencibilidad en el músculo y tejido conectivo, falta de sueño, depresión, falta de concentración, fatiga y rigidez en articulaciones. La medicina occidental aún no encuentra la causa de esta enfermedad. Estr\'es, predisposición gen\'etica y una depresión mayor son consideradas las posibles causas.

Desde una perspectiva Yóguica/ayurv\'edica (Ayur-Veda significa conocimiento de la vida), la causa de los síntomas conocidos como fibromalgia es la desintegración. Esta desintegración tiene componentes mentales, emocionales y físicos. Los tejidos del cuerpo no se encuentran en armonía. La causa puede encontrarse en una circunstancia, memoria o evento que no ha sido completamente recibido, digerido y asimilado. Puede haber falta de credibilidad en que una sanación pueda ocurrir, si esta situación ocurre, las posibilidades de librarse decrementan. Cualquier sanación requiere de la participación del afectado para ser efectiva.

Más específicamente, los elementos de aire y espacio dentro de la constitución individual, puede estar fuera de equilibrio, o incrementado. Ayurveda ve cinco elementos (agua, aire, tierra, fuego y espacio) aterrizados en cinco razgos característicos llamados \textit{doshas}. El dosha Vata contiene aire y espacio. Al ser estos los elementos menos densos, y de más rápido movimiento, las condiciones como fibromalgia pueden surgir de un incremento en este dosha. Esto descarta la idea de algún evento, estilo de vida, lesión o recuerdo mismo pueda ser la causa, sino la interpretación o recepción de estos.

La terapia para la fibromalgia comenzaría con una conversación con el cliente, una historia de eventos que proceden a la condición, y la completa participación y entusiasmo del cliente en su propia sanación, lo qu incluye tarea y práctica personal de asana, pranayama y meditación.

La aproximación a asanas para fibromalgia comienza con aceptación de cero dolor durante la práctica. Cuando el dolor se presenta, el cuerpo/mente se alejará de nuevo en lugar de ingegrarse. La práctica debe comenzar con conciencia en la respiración y movimientos simples, con en un enfoque en una comodidad relativa en la práctica de asanas. Pranayama y meditación deben ser enfocadas en la integración de cuerpo y mente, con una relajación guiada con cobijas para proveer aún más enraizamiento.

\section{Lenguaje}
La voz humana se conforma por muchas cosas: el tono de voz, número de palabras, velocidad de habla, volumen de voz, por supuesto, contenido. Al enseñar yoga, el instructor utiliza todas estas cosas para comunicarse con los estudiantes, affectando cerebro, cuerpo y mente.

\subsection{Tono de Voz}
Posiblemente lo más importante, el tono de voz crea el ánimo de la clase. Modula tu voz para que no sea monótona. Si te encuentras emocionado sobre algo, comunica tu entusiasmo con tu voz, si no te encuentras genuinamente emocionado sobre algo, no pretendas estarlo, tal pretensión será aparente para otros. En una clase acelerada, toma la energía creada por la práctica de asana para avivar tu voz y motivar la clase. En una clase más restaurativa o suave, tu voz y ritmo deben encajar en la intención relajada.

\subsection{Que Tanto Decir?}
Mientras más aprendemos como maestros, más conocimiento queremos transmitir a otros. Últimamente, la mejor enseñanza consiste en facilitar la conexión con la sabiduría interna de los estudiantes. Para estudiantes princiíantes, mant\'en las instrucciones simples. No tengas miedo de repetir la misma instrucción más de una vez. Remarca uno o dos puntos principales en una postura, relacionándolo ocasionalmente con la intención principal de tu clase. Cuando te refieras a una parte del cuerpo o posición, es útil nombrarla de más de una forma. Por ejemplo: ``Desde el desplante, rota tu pi\'e trasero (el izquierdo) hacia afuera 90 grados. Coloca los dedos de la mano derecha bajo tu hombro derecho y rota tu torso hacia la izquierda''.

\subsection{Se Conciso}
Un buen autor dijo una vez a su amigo ``te escribo una carta larga porque no tengo tiempo de escribirte una corta''. Es difícil ser consiso al momento de hablar. Tus la atención enfocada de tus estudiantes es una cosa valiosa. Honra su atención al decir sólo lo necesario en la clase que enseñas. Crea algún silencio ocasional para permitirles reflexionar e incorporar las instrucciones.

\subsection{Volumen/Contenido}
Parate en el salón que enseñarás cuando est vacío. Habla en un tono de voz natural para ti mismo, no distinto de como hablarías a un amigo en algún otro cuarto. Tu puedes practicar leer en voz alra, o ensayar enseñar una postura. Tu voz debe llenar cómodamente el salón sin retumbar. Cuando enseñes te estarás moviendo alrededor del salón, no estático en tu tapete, así que practica esto tambin. Consigue un grabador de sonido, colócalo en la ezquina del salón y practica enseñar por 10 minutos. Escucha la grabación para ver si es entendible y suficientemente fuerte para escuchar cómodamente. Mantente atento a muletillas como ``ok'', ``bien'', ``bueno''. Graba de nuevo y elimina las palabras inecesarias.

Ahora que has escuchado tu estilo de discurso, mantenlo natural. Continúa refinando cómo decir lo que quieres decir de tal forma que sea lo más fácil de escuchar posible. Ahora puedes comenzar a trabajar en el contenido de lo que dirás.

\section{Modificación de Postura}
Cuando se enseña una postura, considera que sea apropiada para la clase particular. Mant\'en en mente que, debido a la diferencia de estructúra ósea, algunos estudiantes serán capaces de realizar posturas avanzadas con poca o ninguna experiencia, pero otros nunca serán capaces de realizarlas, dada la proporción y estructura de sus brazos, piernas y torso. En la planeación de tu clase ua la plantilla de posturas de categorías, incluye posturas fundamentales antes de agregar posturas complejas que requieran torsiones de manos alrededor de piernas o arcos o flexiones demasiado profundas. Al final, el estudiante debe dejar tu clase sinti\'endose revitalizado. La práctica de asanas no es una competencia gimnástica.

Cuando se enseña cuna clase intermedia, remarca las posturas que puedan ser modificadas hacia arriba o hacia abajo en dificultad, como dar la opción de \textit{Balasana} (postura de niño) como alternativa para el perro boca abajo. Observa la respiración completa y la alineación óptima en lugar de incentivar una actividad potencialmente dañina. Cada estudiante debe ser alentado para realizar asanas con sus limitaciones completamente conscientes.

En posturas donde los brazos se encuentran alrededor de las piernas, como en Marichyasana-C, simplemente instruye la postura sin las manos entrelazadas como alternativa. Cuerdas pueden ser utilizadas para hacer las posturas de piso más accesibles para estudiantes rígidos.

Cuando instruyes una clase y das modificaciones, recuerda que una modificación es para facilitar una experiencia más profunda para aquellos que así lo desean. Para obtener una experiencia más profunda, cuida el lenguaje que utilizas.


\section{Observación: Estudiante Individual}
\subsection{Fundación}
\subsection{Estado de Animo General}
\section{El Rol del Maestro}
\section{Fundamentos de una Secuencia}
\label{sec:fundSec}
\subsection{Secuenciado Variable y Establecido}
\subsection{Principios de Secuenciado}
\subsection{Entretenimiento - Centrando la Clase}
\subsection{Secuenciando la Clase a Nivel Mixto}
\section{Creando Intención}
\subsection{La Intención de "Liberar Tensión"}
\section{La Práctica y el Servicio de Enseñar Yoga}


